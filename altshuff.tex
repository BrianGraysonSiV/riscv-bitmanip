\chapter{Alternative to \texttt{shuffle} instruction}

There are a couple of issues with the {\tt shuffle} instruction as currently proposed.

(1) It performs a fairly complex operation (the description of the instruction
spans 4 pages).

(2) It tries to do two things at the same time, namely perform various zip/unzip
operations and perform butterfly operations of different orders.

(3) Even though all butterfly stages should be present in a GREV unit, tapping into
those stages to inject the masks for {\tt shuffle} increases overall complexity of
the core. This is not really necessary: A single butterfly stage would be sufficient
because zip/unzip operations can be used to emulate the other butterfly stages.

(4) Implementing a butterfly+mask operation can easily be done with grevi, zip,
and the mix pattern (and+andc+or).

This chapter proposes to replace the {\tt shuffle} instruction with a generalized
{\tt zip} instruction with an encoding similar to {\tt grevi}.

%%%%%%%%%%%%%%%%%%%%%%%%%%%%%%%%%%%%%%%%%%%%%%%%%%%%%%%%%%%%%%%%%%%%%%%%%%%%%%%%%%%%%%%%%%%

\section{Generalized zip (\texttt{zip})}

{\it Under construction}

{\tt zip} is the third bit permutation instruction in XBitmanip, after {\tt rori}
and {\tt grevi}.

The {\tt zip} instruction uses an I-type encoding similar to {\tt grevi}. There
are XLEN different generalized zip operations, with 0 being the identity. See
Table~\ref{zip-modes}.

Like GREV and rotate shift, the {\tt zip} instruction can be implemented using a short
sequence of ``atomic'' permutations, that are enabled or disabled by the mode (shamt)
bits. Zip has one more stage than a shift or GREV unit, with the first and last stage
both controlled by mode$[0]$. As a special case, mode$=$00001 only activates the
first or last stage.

\begin{table}[h]
\begin{small}
\begin{center}
\begin{tabular}{l l l}
      mode   & Bit index rotations              & Comments / Pseudo instruction \\ \hline

      00000  & no-op                            & \\
      00001  & bit-reversal                     & \\
      00010  & {\tt i[1] -> i[0]}               & \\
\sout{00011} & {\tt i[1] <- i[0]}               & {\it equivalent to 00010} \\
      00100  & {\tt i[2] -> i[1]}               & \\
\sout{00101} & {\tt i[2] <- i[1]}               & {\it equivalent to 00100} \\
      00110  & {\tt i[2] -> i[0]}               & {\tt shuffle.b} \\
      00111  & {\tt i[2] <- i[0]}               & {\tt unshuffle.b} \\

\hline

      01000  & {\tt i[3] -> i[2]}               & \\
\sout{01001} & {\tt i[3] <- i[2]}               & {\it equivalent to 01000} \\
      01010  & {\tt i[3] -> i[2], i[1] -> i[0]} & \\
\sout{01011} & {\tt i[3] <- i[2], i[1] <- i[0]} & {\it equivalent to 01010} \\
      01100  & {\tt i[3] -> i[1]}               & {\tt shuffle4.h} \\
      01101  & {\tt i[3] <- i[1]}               & {\tt unshuffle4.h} \\
      01110  & {\tt i[3] -> i[0]}               & {\tt shuffle.h} \\
      01111  & {\tt i[3] <- i[0]}               & {\tt unshuffle.h} \\

\hline

      10000  & {\tt i[4] -> i[3]}               & \\
\sout{10001} & {\tt i[4] <- i[3]}               & {\it equivalent to 01010} \\
      10010  & {\tt i[4] -> i[3], i[1] -> i[0]} & \\
\sout{10011} & {\tt i[4] <- i[3], i[1] <- i[0]} & {\it equivalent to 10010} \\
      10100  & {\tt i[4] -> i[3], i[2] -> i[1]} & \\
\sout{10101} & {\tt i[4] <- i[3], i[2] <- i[1]} & {\it equivalent to 10100} \\
      10110  & {\tt i[4] -> i[3], i[2] -> i[0]} & \\
      10111  & {\tt i[4] <- i[3], i[2] <- i[0]} & \\

\hline

      11000  & {\tt i[4] -> i[2]}               & {\tt shuffle8} \\
      11001  & {\tt i[4] <- i[2]}               & {\tt unshuffle8} \\
      11010  & {\tt i[4] -> i[2], i[1] -> i[0]} & \\
      11011  & {\tt i[4] <- i[2], i[1] <- i[0]} & \\
      11100  & {\tt i[4] -> i[1]}               & {\tt shuffle4} \\
      11101  & {\tt i[4] <- i[1]}               & {\tt unshuffle4} \\
      11110  & {\tt i[4] -> i[0]}               & {\tt shuffle} \\
      11111  & {\tt i[4] <- i[0]}               & {\tt unshuffle} \\

\end{tabular}
\end{center}
\end{small}
\caption{RV32 modes for {\tt zip} instruction}
\label{zip-modes}
\end{table}

