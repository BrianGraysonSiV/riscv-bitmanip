\chapter{Alternative to \texttt{shuffle} instruction}

There are a couple of issues with the {\tt shuffle} instruction as currently proposed.

(1) It performs a fairly complex operation (the description of the instruction
spans 4 pages).

(2) It tries to do two things at the same time, namely perform various zip/unzip
operations and perform butterfly operations of different orders.

(3) Even though all butterfly stages should be present in a GREV unit, tapping into
those stages to inject the masks for {\tt shuffle} increases overall complexity of
the core. This is not really necessary: A single butterfly stage would be sufficient
because zip/unzip operations can be used to emulate the other butterfly stages.

(4) Implementing a butterfly+mask operation can easily be done with grevi, zip,
and the mix pattern (and+andc+or).

This chapter proposes to replace the {\tt shuffle} instruction with a generalized
{\tt zip} instruction with an encoding similar to {\tt grevi}.

%%%%%%%%%%%%%%%%%%%%%%%%%%%%%%%%%%%%%%%%%%%%%%%%%%%%%%%%%%%%%%%%%%%%%%%%%%%%%%%%%%%%%%%%%%%

\section{Generalized zip (\texttt{zip})}

{\it Under construction}

{\tt zip} is the third bit permutation instruction in XBitmanip, after {\tt rori}
and {\tt grevi}.

The {\tt zip} instruction uses an I-type encoding similar to {\tt grevi}. There
are XLEN different generalized zip operations, with 0 being the identity. See
Table~\ref{zip-modes}.

Like GREV and rotate shift, the {\tt zip} instruction can be implemented using a short
sequence of ``atomic'' permutations, that are enabled or disabled by the mode (shamt)
bits. Zip has one more stage than a shift or GREV unit, with the first and last stage
both controlled by mode$[0]$.

There is no R-type instruction for {\tt zip}. It is an I-type only instruction.

\begin{table}[h]
\begin{small}
\begin{center}
\begin{tabular}{c l l}
      mode   & Bit index rotations               & Instruction \\ \hline

      0000 0  & no-op                            & {\it reserved} \\
      0000 1  & no-op                            & {\it reserved} \\
      0001 0  & {\tt i[1] -> i[0]}               & {\tt zip.n, unzip.n} \\
\sout{0001 1} & {\it equivalent to 0001 0}       & {\it reserved} \\
      0010 0  & {\tt i[2] -> i[1]}               & {\tt zip2.b, unzip2.b} \\
\sout{0010 1} & {\it equivalent to 0010 0}       & {\it reserved} \\
      0011 0  & {\tt i[2] -> i[0]}               & {\tt zip.b} \\
      0011 1  & {\tt i[2] <- i[0]}               & {\tt unzip.b} \\

\hline

      0100 0  & {\tt i[3] -> i[2]}               & {\tt zip4.h, unzip4.h} \\
\sout{0100 1} & {\it equivalent to 0100 0}       & {\it reserved} \\
      0101 0  & {\tt i[3] -> i[2], i[1] -> i[0]} & --- \\
\sout{0101 1} & {\it equivalent to 0101 0}       & {\it reserved} \\
      0110 0  & {\tt i[3] -> i[1]}               & {\tt zip2.h} \\
      0110 1  & {\tt i[3] <- i[1]}               & {\tt unzip2.h} \\
      0111 0  & {\tt i[3] -> i[0]}               & {\tt zip.h} \\
      0111 1  & {\tt i[3] <- i[0]}               & {\tt unzip.h} \\

\hline

      1000 0  & {\tt i[4] -> i[3]}               & {\tt zip8, unzip8} \\
\sout{1000 1} & {\it equivalent to 1000 0}       & {\it reserved} \\
      1001 0  & {\tt i[4] -> i[3], i[1] -> i[0]} & --- \\
\sout{1001 1} & {\it equivalent to 1001 0}       & {\it reserved} \\
      1010 0  & {\tt i[4] -> i[3], i[2] -> i[1]} & --- \\
\sout{1010 1} & {\it equivalent to 1010 0}       & {\it reserved} \\
      1011 0  & {\tt i[4] -> i[3], i[2] -> i[0]} & --- \\
      1011 1  & {\tt i[4] <- i[3], i[2] <- i[0]} & --- \\

\hline

      1100 0  & {\tt i[4] -> i[2]}               & {\tt zip4} \\
      1100 1  & {\tt i[4] <- i[2]}               & {\tt unzip4} \\
      1101 0  & {\tt i[4] -> i[2], i[1] -> i[0]} & --- \\
      1101 1  & {\tt i[4] <- i[2], i[1] <- i[0]} & --- \\
      1110 0  & {\tt i[4] -> i[1]}               & {\tt zip2} \\
      1110 1  & {\tt i[4] <- i[1]}               & {\tt unzip2} \\
      1111 0  & {\tt i[4] -> i[0]}               & {\tt zip} \\
      1111 1  & {\tt i[4] <- i[0]}               & {\tt unzip} \\
\end{tabular}
\end{center}
\end{small}
\caption{RV32 modes for {\tt zip} instruction}
\label{zip-modes}
\end{table}
