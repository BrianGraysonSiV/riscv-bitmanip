\chapter{Discussion and Evaluation}

\section{Frequently Asked Questions}

\subsubsection{Why \texttt{clz} instead of \texttt{ctz}? Why not XYZ?}

The important thing is that we wanted to have a small ISA that only provides
useful atomics for building most of the other "usual" bit manipulation
instructions.

If we chose \texttt{clz} or \texttt{ctz} is not that important. But we want to
pick only one and let the used build the other by combining the supported
instruction with bitwise reverse (which is provided by \texttt{grevi}).

\subsubsection{\texttt{grev[i]} seems to be overly complicated? Do we really need it?}

The \texttt{grevi} instruction can be used to build a wide range of common
bit permutation instructions. For example on RV32:

\begin{longtable}[c]{@{}rl@{}}
\caption{Some GREVI Pseudo-Instructions}\tabularnewline
\toprule
Mode & Meaning on RV32\tabularnewline
\midrule
\endfirsthead
\toprule
Mode & Meaning on RV32\tabularnewline
\midrule
\endhead
4 & swap the nibbles of each byte\tabularnewline
7 & reverse the bits in each byte\tabularnewline
15 & reverse the bits in each halfword\tabularnewline
16 & swap the two half words halfword\tabularnewline
24 & reverse the byte order\tabularnewline
31 & bitwise reverse\tabularnewline
\bottomrule
\end{longtable}

If \texttt{grevi} were removed from this spec we would need to add a few new
instructions in its place for operations such as bitwise reverse or
endianness convertion.

\subsubsection{Do we really need all the \texttt{*W} opcodes for 32 bit ops on RV64?}

I don't know. I think nobody does know at the moment. But they add very little
complexity to the core. So the only question is if it is worth the encoding
space. We need to run proper experiments with compilers that support those
instructions. So they are in for now and if future evaluations show that they
are not worth the encoding space then we can still throw them out.

\subsubsection{Why only \texttt{andc} and not any other complement operators?}

Early versions of this spec also included other \texttt{*c} operators. But
experiments\footnote{http://svn.clifford.at/handicraft/2017/bitcode/} have show that
\texttt{andc} is much more common in bit manipulation code than any other operators.
Especially because it is commonly used in \texttt{mix} and \texttt{mux} operations.

\subsubsection{BEXT/BDEP look like really expensive operations. Do we really need them?}

Yes, they are expensive, but not as expensive as one might expect. A
single-cycle 32 bit BEXT+BDEP+GREV core can be implemented in less space than a
single-cycle 16x16 bit multiplier with 32 bit output.\footnote{https://github.com/cliffordwolf/bextdep}

\section{Analysis of used encoding space}

So how much encoding space is used by the XBitmanip extension?

\begin{longtable}[c]{@{}rr|rr|l@{}}
\caption{XBitmanip encoding space ($log2$, i.e. in equivalent number of bits)}\tabularnewline
\toprule
\multicolumn{2}{c}{RV32} & \multicolumn{2}{c}{RV64} & Instruction\tabularnewline
\midrule
\endfirsthead
\toprule
\multicolumn{2}{c}{RV32} & \multicolumn{2}{c}{RV64} & Instruction\tabularnewline
\midrule
\endhead
1x & 10 & 2x & 10 & CLZ, CLZW\tabularnewline
1x & 10 & 2x & 10 & PCNT, PCNTW\tabularnewline
\midrule
1x & 15 & 2x & 15 & GREV, GREVW\tabularnewline
1x & 15 & 1x & 16 & GREVI\tabularnewline
   &    & 1x & 15 & GREVIW\tabularnewline
\midrule
2x & 15 & 2x & 16 & SLOI, SROI\tabularnewline
   &    & 2x & 15 & SLOIW, SROIW\tabularnewline
2x & 15 & 4x & 15 & SLO, SRO, SLOW, SROW\tabularnewline
\midrule
1x & 15 & 1x & 16 & RORI\tabularnewline
   &    & 1x & 15 & RORIW\tabularnewline
2x & 15 & 4x & 15 & ROR, ROL, RORW, ROLW\tabularnewline
\midrule
3x & 15 & 3x & 15 & ANDC, BEXT, BDEP\tabularnewline
\midrule
4x &  4 & 4x &  4 & C.NEG, C.NOT, C.CLZ, C.PCNT\tabularnewline
\bottomrule
\end{longtable}

In the compressed encoding space XBitmanip needs the equivalent of a 6 bit encoding space,
or $\approx 0.13\%$ of the total $\approx 15.6$ bits available. ($ 100 / (2^{15.6-6}) \approx 0.13 $)

On RV32, XBitmanip requires the equivalent of a $\approx 18.6$ bit encoding space in
the uncompressed encoding space. For comparison: A single standard I-type
instruction (such as \texttt{ADDI} or \texttt{SLTIU}) requires a $22$ bit
encoding space. I.e. the entire RV32 XBitmanip extension needs less than
one-eighth of the encoding space of the \texttt{SLTIU} instruction.

$$ \frac{log(2\cdot2^{10} + 12\cdot2^{15})}{log(2)} \approx 18.6 $$

On RV64, XBitmanip requires the equivalent of a $\approx 19.7$ bit encoding
space in the uncompressed encoding space. I.e. the entire RV64 XBitmanip
extension needs less than one-quarter of the encoding space of the
\texttt{SLTIU} instruction.

$$ \frac{log(4\cdot2^{10} + 17\cdot2^{15} + 4\cdot2^{16})}{log(2)} \approx 19.7 $$

