\chapter{Discussion}

\section{Frequently Asked Questions}

\subsubsection{Which instructions were considered but not included?}

\begin{itemize}
\item A bit-field extract instruction. Unfortunately the encoding space required
by such an instruction would be enourmous (for forward-looking support up to
RV128 this would require a $2 \time 7 = 14$ bits immedate. This was deemed
too expensive considering that the same functionality can already be implemented
using only two shift instructions. However, see Chapter~\ref{bfxp} for an
alternative. (The same encoding space argument applies to instructions for
inserting bits in a bit field, or setting or clearing ranges of bits.)

\item A ``butterfly'' instruction with an XLEN/2 immediate that would run one
butterfly stage (see Section~\ref{butterfly}). The {\tt smartbextdep} implementation
of {\tt bext}/{\tt bdep} (see Section~\ref{cores}) already includes the hardware
for that, all that would be required is exposing a few internal control signals.
However, the encoding cost of such an instruction would be enourmous, the
instruction itself would only be useful in niche applications, and it would require
implementors to implement {\tt bext}/{\tt bdep} in a certain way. So it's is
not worth the effort considering that a butterfly stage can also be implemented in
four instructions using {\tt grevi} and the MIX pattern (see Section~\ref{butterfly}).

\item A reverse-subract-immediate operation that performs the calculation
$\texttt{imm} - \texttt{rs}$.  This comes up often in calculating bit indices
(for example $\texttt{XLEN} - i$ is a very common operation, and compressed
instructions are only of some help here because compressed immediates are in
the range $-32 \dots 31$). However, a reverse-subract-immediate operation is
not very bit-manipulation specific, would require a much larger encoding
space than the rest of XBitmanip (it would have a 12 bit immediate), and
does not add any functionality that can not be emulated easily with {\tt neg} and
{\tt addi}: $\texttt{rsbi}(x, k) = \texttt{neg}(\texttt{addi}(x, -k)) = \texttt{addi}(\texttt{neg}(x), k)$

\item Conditional move and mix instructions. Those would be very helpful! However,
they would require three source operands, and we did not want to add a register
file with three read ports as requirement for XBitmanip. See Section~\ref{mixmux}
for short sequences implementing conditional move (aka mux) and mix operations.

\item Funnel shift instructions. Those are also super helpful! For example when
processing input that is a bit-packed stream of words of arbitrary (variable)
bit width. However, a funnel shift instruction would also require three source
operands, therefore it was not considered for inclusion in XBitmanip.

\item Instructions for testing, setting and clearing bits. This operations can
be performed in few easily macro-op fusable instructions. So implementations
might provide hardware support for them, but we do not reserve dedicated opcodes
for those operations.

\item Instructions for the usual patterns of ALU operations to fiddle with
trailing bits, such as {\tt x \& -x} (extract lowest set bit), or
{\tt x \^{} (x - 1)} (mask up to and including lowest set bit). But those
operations all fit into short easily macro-op fusable instruction sequences
(see Section~\ref{x86comp}).
\end{itemize}

\subsubsection{\texttt{grev} seems to be overly complicated? Do we really need it?}

The \texttt{grev} instruction can be used to build a wide range of common
bit permutation instructions, such as endianess conversion or bit reversal.

If \texttt{grev} were removed from this spec we would need to add a few
new instructions in its place for those operations.

\subsubsection{Do we really need all the \texttt{*W} opcodes for 32 bit ops on RV64?}

I don't know. I think nobody does know at the moment. But they add very little
complexity to the core. So the only question is if it is worth the encoding
space. We need to run proper experiments with compilers that support those
instructions. So they are in for now and if future evaluations show that they
are not worth the encoding space then we can still throw them out.

\subsubsection{Why only \texttt{andc} and not any other complement operators?}

Early versions of this spec also included other \texttt{*c} operators. But
experiments have shown that \texttt{andc} is much more common in bit
manipulation code than any other operators.~\cite{Wolf17A}
Especially because it is commonly used in \texttt{mix} and \texttt{mux} operations.

\subsubsection{Why \texttt{andc}? It can easily be emulated using \texttt{and} and \texttt{not}.}

Yes, and we did not include any other ALU+complement operators. But \texttt{andc}
is so common (mostly because of the \texttt{mix} and \texttt{mux} patterns), and
its implementation is so cheap, that we decided to dedicate an R-type instruction
to the operation.

\subsubsection{The shift-ones instructions can be emulated using {\tt not} and logical shift? Do we really need it?}

Yes, a shift-ones instruction can easily be implemented using the logical shift
instructions, with a bitwise invert before and after it. (This is literally the
code we are using in the reference C implementation of shift-ones.)

We have decided to include it for now so that we can collect benchmark data
before making a final decision on the inclusion or exclusion of those
instructions. The main objection here is instruction encoding space. The
hardware overhead of adding this functionality to a shifter is relatively low.

\subsubsection{BEXT/BDEP look like really expensive operations. Do we really need them?}

Yes, they are expensive, but not as expensive as one might expect. A
single-cycle 32 bit BEXT+BDEP+GREV core can be implemented in less space than a
single-cycle 16x16 bit multiplier with 32 bit output.~\cite{Wolf17B}

It is also important to keep in mind that implementing those operations in
software is very expensive. Hacker's Delight contains a highly optimized
software implementation of 32-bit BEXT that requires $>120$ instructions. Their
BDEP software implementation requires $>160$ instructions. (Please disregard the
``hardware-oriented algorithm'' described in Hacker's Delight. It is
extremely expensive compared to other implementations.~\cite{Wolf17B})

\subsubsection{But do we really need 64-bit BEXT/BDEP?}

Good question. A 64-bit BEXT/BDEP unit certainly is more than 2x the size of a
32-bit unit and in most cases 32-bit would be sufficient. It is also not very difficult
to emulate 64-bit BEXT/BDEP using 32-bit BEXT/BDEP. On RV64 (with data in {\tt a0} and
mask in {\tt a1}):

\begin{multicols}{2}
\begin{minipage}{\linewidth}
\begin{verbatim}
  bext64:
    pcntw a2, a1
    bextw a3, a0, a1
    c.srli a0, 32
    c.srli a1, 32
    bextw a0, a0, a1
    sloi a1, zero, 32
    c.and a3, a1
    c.and a0, a1
    sll a0, a2
    c.or a0, a3
    ret
\end{verbatim}
\end{minipage}

\begin{minipage}{\linewidth}
\begin{verbatim}
  bdep64:
    pcntw a2, a1
    bdepw a3, a0, a1
    srl a0, a0, a2
    c.srli a1, 32
    bdepw a0, a0, a1
    c.slli a0, 32
    c.slli a3, 32
    c.srli a3, 32
    c.or a0, a3
    ret
\end{verbatim}
\end{minipage}
\end{multicols}

However, one solution here would be to still reserve the opcode for 64-bit
BEXT/BDEP and leave it to the implementation to decide whether to implement the
function in hardware or emulating it using a software trap.

\subsubsection{Compressed encoding space is expensive. Do we need {\tt c.neg}, {\tt c.not}, {\tt c.brev}?}

Because they are unary operations and encoded with $\texttt{rd}' = \texttt{rs}'$, they only take
up $\approx 0.05\%$ of the total ``C'' encoding space, and only $\approx 1.1\%$
of the remaining reserved ``C'' encoding space.

They are common building blocks of sequences that are candidates for macro-op fusion,
and keeping these sequences short can help with decoding them efficiently.

{\tt c.neg} can be emulated using {\tt c.not} if the next (or previous) operation
is an {\tt addi}. But commonly {\tt c.neg} is simply used to turn the value 1
into all-bits-set and leave the value 0 at 0. Or it is used stand-alone to
effectively perform the operation $\textrm{XLEN}-i$ to calulate the shift
amount for the 2nd shift in a funnel shift (see Section~\ref{funnel}, XLEN does
not actually need to be added because the shift operations only uses the lower
$log_2(\textrm{XLEN})$ bits of their 2nd argument anyways). In those cases
there is no addition before or after the {\tt c.neg} instruction.

\section{Analysis of used encoding space}

\begin{table}
\begin{center}
\begin{tabular}{rr|rr|l}
\multicolumn{2}{c|}{RV32} & \multicolumn{2}{c|}{RV64} & Instruction \\
\hline
3x &  0 & 6x &  0 & {\tt clz, clzw, ctz, ctzw, pcnt, pcntw} \\
\hline
1x & 15 & 1x & 15 & {\tt grev} \\
2x & 15 & 2x & 16 & {\tt grevi, gzip} \\
\hline
2x & 15 & 6x & 15 & {\tt slo, sro, slow, srow, sloiw, sroiw} \\
2x & 15 & 2x & 16 & {\tt sloi, sroi} \\
\hline
2x & 15 & 5x & 15 & {\tt ror, rol, rorw, rolw, roriw} \\
1x & 15 & 1x & 16 & {\tt rori} \\
\hline
3x & 15 & 3x & 15 & {\tt andc, bext, bdep} \\
   &    & 2x & 15 & {\tt bextw, bdepw} \\
\hline
3x &  4 & 3x &  4 & {\tt c.neg, c.not, c.brev} \\
\end{tabular}
\end{center}
\caption{XBitmanip encoding space ($log2$, i.e. in equivalent number of bits)}
\label{encspace-tab}
\end{table}

Table~\ref{encspace-tab} lists all XBitmanip instructions and the encoding
space needed to implement them.  We do not count any encoding space for the
unary instructions {\tt clz}, {\tt clzw}, {\tt ctz}, {\tt ctzw}, {\tt pcnt},
and {\tt pcntw} because they can be implemented in the reserved modes in {\tt
gzip}.

The compressed encoding space is $\approx 15.6$ bits wide.

$$ log_2(3 \cdot 2^{14}) \approx 15.585 $$

The compressed XBitmanip instructions need the equivalent of a 4.6 bit
encoding space, or $\approx 0.05\%$ of the total $\approx 15.6$ bits available.

$$ log_2(3 \cdot 2^3) \approx 4.585 $$
$$ 100 / (2^{15.585-4.585}) \approx 0.049 $$

The reserved ``C'' encoding space as of Version 2.3 of the RISC-V User-Level
ISA sim summarized in Table~\ref{resctab}. (This is not including RV32-only
or RV64-only reserved encoding space.) According to this information
the reserved space is $\approx 11.1$ bits wide. Therefore the compressed
XBitmanip instructions would use $\approx 1.1\%$ of the remaining
reserved ``C'' encoding space.

$$ log_2(2^3-1 + 2^{11} + 2^5 + 2^7 + 1) \approx 11.114 $$
$$ 100 / (2^{11.114-4.585}) \approx 1.083 $$

\begin{table}[h]
\begin{small}
\begin{center}
\begin{tabular}{p{0in}p{0.05in}p{0.05in}p{0.05in}p{0.05in}p{0.05in}p{0.05in}p{0.05in}p{0.05in}p{0.05in}p{0.05in}p{0.05in}p{0.05in}p{0.05in}p{0.05in}p{0.05in}p{0.05in}l}
& & & & & & & & & & \\ &
\instbit{15} &
\instbit{14} &
\instbit{13} &
\multicolumn{1}{c}{\instbit{12}} &
\instbit{11} &
\instbit{10} &
\instbit{9} &
\instbit{8} &
\instbit{7} &
\instbit{6} &
\multicolumn{1}{c}{\instbit{5}} &
\instbit{4} &
\instbit{3} &
\instbit{2} &
\instbit{1} &
\instbit{0} \\
\cline{2-17}
&
\multicolumn{3}{|c|}{000} &
\multicolumn{8}{c|}{0} &
\multicolumn{3}{c|}{nz} &
\multicolumn{2}{c|}{00} & $2^3-1$ \\
\cline{2-17}
&
\multicolumn{3}{|c|}{100} &
\multicolumn{11}{c|}{---} &
\multicolumn{2}{c|}{00} & $2^{11}$ \\
\cline{2-17}
&
\multicolumn{3}{|c|}{011} &
\multicolumn{1}{c|}{0} &
\multicolumn{5}{c|}{---} &
\multicolumn{5}{c|}{0} &
\multicolumn{2}{c|}{01} & $2^5$ \\
\cline{2-17}
&
\multicolumn{3}{|c|}{100} &
\multicolumn{3}{c|}{111} &
\multicolumn{3}{c|}{---} &
\multicolumn{1}{c|}{1} &
\multicolumn{4}{c|}{---} &
\multicolumn{2}{c|}{01} & $2^7$ \\
\cline{2-17}
&
\multicolumn{3}{|c|}{100} &
\multicolumn{1}{c|}{0} &
\multicolumn{5}{c|}{0} &
\multicolumn{5}{c|}{0} &
\multicolumn{2}{c|}{10} & $1$ \\
\cline{2-17}
\end{tabular}
\end{center}
\end{small}
\caption{Reserved ``C'' encoding space as of Version 2.3 of the RISC-V User-Level ISA}
\label{resctab}
\end{table}

On RV32, XBitmanip requires the equivalent of a $\approx 18.7$ bit encoding space in
the uncompressed encoding space. For comparison: A single standard I-type
instruction (such as \texttt{ADDI} or \texttt{SLTIU}) requires a $22$ bit
encoding space. I.e. the entire RV32 XBitmanip extension needs less than
one-eighth of the encoding space of the \texttt{SLTIU} instruction.

$$ log_2(13\cdot2^{15}) \approx 18.700 $$

On RV64, XBitmanip requires the equivalent of a $\approx 19.8$ bit encoding
space in the uncompressed encoding space. I.e. the entire RV64 XBitmanip
extension needs about one-quarter of the encoding space of the \texttt{SLTIU}
instruction.

$$ log_2(17\cdot2^{15} + 5\cdot2^{16}) \approx 19.755 $$

\section{Area usage of reference implementations}
\label{cores}

\begin{figure}[b]
%%%%%%% BEGIN: verilog/stats.tex %%%%%%%
\begin{center}
\begin{tikzpicture}
\draw[gray!20] (0,-0.119) -- (11.500,-0.119);
\draw[gray!20] (0,-0.238) -- (11.500,-0.238);
\draw[gray!20] (0,-0.356) -- (11.500,-0.356);
\draw[gray!20] (0,-0.475) -- (11.500,-0.475);
\draw[gray!20] (0,-0.594) -- (11.500,-0.594);
\draw[gray!20] (0,0.340) -- (11.500,0.340);
\draw[gray!20] (0,0.680) -- (11.500,0.680);
\draw[gray!20] (0,1.020) -- (11.500,1.020);
\draw[gray!20] (0,1.360) -- (11.500,1.360);
\draw[gray!20] (0,1.700) -- (11.500,1.700);
\draw[gray!20] (0,2.040) -- (11.500,2.040);
\draw[gray!20] (0,2.380) -- (11.500,2.380);
\draw[gray!20] (0,2.720) -- (11.500,2.720);
\draw[gray!20] (0,3.060) -- (11.500,3.060);
\draw[gray!20] (0,3.400) -- (11.500,3.400);
\fill[white] (0,2.5) rectangle (4,4);
\draw[-latex] (0,0) -- (0,4);
\draw (-0.2,4) node[left,rotate=90] {\tiny Logic (LUTs or Gates)};
\draw[-latex] (0,0) -- (0,-1);
\draw (-0.2,-1) node[right,rotate=90] {\tiny FFs};
\draw[pattern=north west lines, pattern color=blue] (0.25,3.5) rectangle (0.5,3.75);
\draw[pattern=crosshatch dots, pattern color=green] (0.25,3.0) rectangle (0.5,3.25);
\draw[pattern=crosshatch, pattern color=magenta] (0.25,2.5) rectangle (0.5,2.75);
\draw (0.5,3.5 + 0.125) node[right] {\tiny ASIC Gates (ror $=$ 458)};
\draw (0.5,3.0 + 0.125) node[right] {\tiny FPGA 4-LUTs (ror $=$ 160)};
\draw (0.5,2.5 + 0.125) node[right] {\tiny D-FFs (ror $=$ 32)};
\draw (0.550,0) node[above right,rotate=90] {\tiny\tt ror};
\fill[white] (0.500,0) rectangle (0.750,0.340);
\fill[white] (0.750,0) rectangle (1.000,0.340);
\fill[white] (0.500,0) rectangle (1.000,-0.119);
\draw[pattern=north west lines, pattern color=blue] (0.500,0) rectangle (0.750,0.340);
\draw[pattern=crosshatch dots, pattern color=green] (0.750,0) rectangle (1.000,0.340);
\draw[pattern=crosshatch, pattern color=magenta] (0.500,0) rectangle (1.000,-0.119);
\draw (1.550,0) node[above right,rotate=90] {\tiny\tt tinygrev};
\fill[white] (1.500,0) rectangle (1.750,0.160);
\fill[white] (1.750,0) rectangle (2.000,0.240);
\fill[white] (1.500,0) rectangle (2.000,-0.160);
\draw[pattern=north west lines, pattern color=blue] (1.500,0) rectangle (1.750,0.160);
\draw[pattern=crosshatch dots, pattern color=green] (1.750,0) rectangle (2.000,0.240);
\draw[pattern=crosshatch, pattern color=magenta] (1.500,0) rectangle (2.000,-0.160);
\draw (2.550,0) node[above right,rotate=90] {\tiny\tt simplegrev};
\fill[white] (2.500,0) rectangle (2.750,0.340);
\fill[white] (2.750,0) rectangle (3.000,0.340);
\fill[white] (2.500,0) rectangle (3.000,-0.119);
\draw[pattern=north west lines, pattern color=blue] (2.500,0) rectangle (2.750,0.340);
\draw[pattern=crosshatch dots, pattern color=green] (2.750,0) rectangle (3.000,0.340);
\draw[pattern=crosshatch, pattern color=magenta] (2.500,0) rectangle (3.000,-0.119);
\draw (3.550,0) node[above right,rotate=90] {\tiny\tt tinygzip};
\fill[white] (3.500,0) rectangle (3.750,0.217);
\fill[white] (3.750,0) rectangle (4.000,0.266);
\fill[white] (3.500,0) rectangle (4.000,-0.156);
\draw[pattern=north west lines, pattern color=blue] (3.500,0) rectangle (3.750,0.217);
\draw[pattern=crosshatch dots, pattern color=green] (3.750,0) rectangle (4.000,0.266);
\draw[pattern=crosshatch, pattern color=magenta] (3.500,0) rectangle (4.000,-0.156);
\draw (4.550,0) node[above right,rotate=90] {\tiny\tt simplegzip};
\fill[white] (4.500,0) rectangle (4.750,0.277);
\fill[white] (4.750,0) rectangle (5.000,0.336);
\fill[white] (4.500,0) rectangle (5.000,-0.119);
\draw[pattern=north west lines, pattern color=blue] (4.500,0) rectangle (4.750,0.277);
\draw[pattern=crosshatch dots, pattern color=green] (4.750,0) rectangle (5.000,0.336);
\draw[pattern=crosshatch, pattern color=magenta] (4.500,0) rectangle (5.000,-0.119);
\draw (5.550,0) node[above right,rotate=90] {\tiny\tt simplebitcnt};
\fill[white] (5.500,0) rectangle (5.750,0.460);
\fill[white] (5.750,0) rectangle (6.000,0.317);
\fill[white] (5.500,0) rectangle (6.000,-0.022);
\draw[pattern=north west lines, pattern color=blue] (5.500,0) rectangle (5.750,0.460);
\draw[pattern=crosshatch dots, pattern color=green] (5.750,0) rectangle (6.000,0.317);
\draw[pattern=crosshatch, pattern color=magenta] (5.500,0) rectangle (6.000,-0.022);
\draw (6.550,0) node[above right,rotate=90] {\tiny\tt simplebfxp};
\fill[white] (6.500,0) rectangle (6.750,0.237);
\fill[white] (6.750,0) rectangle (7.000,0.261);
\fill[white] (6.500,0) rectangle (7.000,-0.119);
\draw[pattern=north west lines, pattern color=blue] (6.500,0) rectangle (6.750,0.237);
\draw[pattern=crosshatch dots, pattern color=green] (6.750,0) rectangle (7.000,0.261);
\draw[pattern=crosshatch, pattern color=magenta] (6.500,0) rectangle (7.000,-0.119);
\draw (7.550,0) node[above right,rotate=90] {\tiny\tt simplebextdep};
\fill[white] (7.500,0) rectangle (7.750,0.586);
\fill[white] (7.750,0) rectangle (8.000,0.663);
\fill[white] (7.500,0) rectangle (8.000,-0.486);
\draw[pattern=north west lines, pattern color=blue] (7.500,0) rectangle (7.750,0.586);
\draw[pattern=crosshatch dots, pattern color=green] (7.750,0) rectangle (8.000,0.663);
\draw[pattern=crosshatch, pattern color=magenta] (7.500,0) rectangle (8.000,-0.486);
\draw (8.550,0) node[above right,rotate=90] {\tiny\tt smartbextdep};
\fill[white] (8.500,0) rectangle (8.750,1.145);
\fill[white] (8.750,0) rectangle (9.000,1.118);
\fill[white] (8.500,0) rectangle (9.000,-0.119);
\draw[pattern=north west lines, pattern color=blue] (8.500,0) rectangle (8.750,1.145);
\draw[pattern=crosshatch dots, pattern color=green] (8.750,0) rectangle (9.000,1.118);
\draw[pattern=crosshatch, pattern color=magenta] (8.500,0) rectangle (9.000,-0.119);
\draw (9.550,0) node[above right,rotate=90] {\tiny\tt XBitmanip};
\fill[white] (9.500,0) rectangle (9.750,2.221);
\fill[white] (9.750,0) rectangle (10.000,2.110);
\fill[white] (9.500,0) rectangle (10.000,-0.379);
\draw[pattern=north west lines, pattern color=blue] (9.500,0) rectangle (9.750,2.221);
\draw[pattern=crosshatch dots, pattern color=green] (9.750,0) rectangle (10.000,2.110);
\draw[pattern=crosshatch, pattern color=magenta] (9.500,0) rectangle (10.000,-0.379);
\draw (10.550,0) node[above right,rotate=90] {\tiny\tt Rocket MulDiv};
\fill[white] (10.500,0) rectangle (10.750,3.697);
\fill[white] (10.750,0) rectangle (11.000,3.066);
\fill[white] (10.500,0) rectangle (11.000,-0.445);
\draw[pattern=north west lines, pattern color=blue] (10.500,0) rectangle (10.750,3.697);
\draw[pattern=crosshatch dots, pattern color=green] (10.750,0) rectangle (11.000,3.066);
\draw[pattern=crosshatch, pattern color=magenta] (10.500,0) rectangle (11.000,-0.445);
\draw (0,0) -- (11.500,0);
\end{tikzpicture}
\end{center}
%%%%%%% END: verilog/stats.tex %%%%%%%
\caption{Relative area of XBitmanip reference cores compared to simple 32-bit rotate
shift and Rocket MulDiv. The height of 1 LUT $\hat{=}$ 2.86 Gates, 1 FF $\hat{=}$ 5.00 Gates.}
\label{area-fig}
\end{figure}

We created RV32 implementations for the different compute cores necessary to
implement XBitmanip (and XBitfield): \url{https://github.com/cliffordwolf/xbitmanip/tree/master/verilog}

We are comparing the area of those cores with the following two references:

\begin{enumerate}
\item A very basic right-rotate shift core ({\tt ror}):
\begin{verbatim}
    module reference_ror (
        input clock,
        input [31:0] din,
        input [4:0] shamt,
        output reg [31:0] dout
    );
        always @(posedge clock)
            dout <= {din, din} >> shamt;
    endmodule
\end{verbatim}
\item A version of the Rocket RV32 {\tt MulDiv} core that performs multiplication
in 5 cycles and division/modulo in 32+ cycles. This is the {\tt MulDiv} default
configuration used in Rocket for small cores.
\end{enumerate}

The implementations of cores performing XBitmanip (and XBitfield) operations are:

\begin{itemize}
\item {\tt tinygrev} --- A small multi-cycle implementation of {\tt grev}/{\tt grevi}.
It takes 6 cycles for one operation.
\item {\tt tinygzip} --- A small multi-cycle implementation of {\tt gzip}.
It takes 5 cycles for one operation.
\item {\tt simplegrev} --- A simple single-cycle implementation of {\tt grev}/{\tt grevi}.
\item {\tt simplegzip} --- A simple single-cycle implementation of {\tt gzip}.
\item {\tt simplebitcnt} --- A simple single-cycle implementation of {\tt clz}, {\tt ctz}, and {\tt pcnt}.
\item {\tt simplebfxp} --- A simple single-cycle implementation of the XBitfield {\tt bfxp}
instruction (see Chapter~\ref{bfxp}). It requires an external rotate-shift implementation.
\item {\tt simplebextdep} --- A straight-forward multi-cycle implementation of {\tt bext}/{\tt bdep}.
It takes up to 32 cycles for one operation (number of set mask bits).
\item {\tt smartbextdep} --- A single-cycle implementation of {\tt bext}/{\tt bdep}
using the method described in~\cite{Hilewitz06}.
\end{itemize}

We implemented those cores with an ASIC cell library containing NOT, NAND, NOR,
AOI3, OAI3, AOI4, and OAI4 gates. For the gate counts below we count NAND and NOR
as 1 gate, NOT as 0.5 gates, AOI3 and OAI3 as 1.5 gates, and AOI4 and OAI4 as 2 gates.

We also implemented the cores using a simple FPGA architecture with 4-LUTs (and no
dedicated CARRY or MUX resources).

\begin{table}[h]
%%%%%%% BEGIN: verilog/stats.tex %%%%%%%
\begin{center}
\begin{tabular}{l|rr|rr|r}
Module & Gates & Depth & 4-LUTs & Depth & FFs \\
\hline
{\tt ror} & 458 & 7 & 160 & 5 & 32 \\
\hline
{\tt tinygrev} & 215 & 5 & 113 & 3 & 43 \\
{\tt simplegrev} & 458 & 7 & 160 & 5 & 32 \\
\hline
{\tt tinygzip} & 292 & 6 & 125 & 4 & 42 \\
{\tt simplegzip} & 373 & 6 & 158 & 4 & 32 \\
\hline
{\tt simplebitcnt} & 619 & 42 & 149 & 13 & 6 \\
\hline
{\tt simplebfxp} & 319 & 13 & 123 & 6 & 32 \\
\hline
{\tt simplebextdep} & 790 & 35 & 312 & 13 & 131 \\
{\tt smartbextdep} & 1542 & 28 & 526 & 10 & 32 \\
\hline
{\tt XBitmanip} & 2992 & 42 & 993 & 13 & 102 \\
{\tt Rocket MulDiv} & 4980 & 53 & 1443 & 24 & 120 \\
\end{tabular}
\end{center}
%%%%%%% END: verilog/stats.tex %%%%%%%
\caption{Area and logic depth of XBitmanip reference cores compared to simple 32-bit rotate
shift and Rocket MulDiv. The entry {\tt XBitmanip} is just the total of {\tt simplegrev} $+$ {\tt
simplegzip} $+$ {\tt simplebitcnt} $+$ {\tt smartbextdep}. Not included in {\tt XBitmanip} is the
cost for adding shift-ones and rotate-shift support to the existing ALU shifter
and the cost for additional decode and control logic.}
\label{area-tab}
\end{table}

Table~\ref{area-tab} and Figure~\ref{area-fig} show the area and logic depth of
our reference cores (and the simple 32-bit rotate shift and Rocket MulDiv for
comparison).

{\tt smartbextdep} could share resources with {\tt simplegrev} (it contains
two butterfly circuits) and {\tt simplebitcnt} (it contains a prefix adder network).
We have not explored those resource sharing options in our reference cores.
