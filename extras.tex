\chapter{Additional Instructions}

This chapter contains additional instructions that are being proposed and/or
currently discussed within the Bitmanip task group.

%%%%%%%%%%%%%%%%%%%%%%%%%%%%%%%%%%%%%%%%%%%%%%%%%%%%%%%%%%%%%%%%%%%%%%%%%%%%%%%%%%%%%%%%%%%

\section{\hsout{Byte-swap with sign extend (\texttt{bswaps.h, bswaps.w})}}

This instructions perform a byte-swap (endianness conversion) on the lower 16 bits
(\texttt{bswaps.h}), 32 bits (\texttt{bswaps.w}, RV64), or 64 bits (\texttt{bswaps.d}, RV128)
of the argument and then sign extend the result. (Byte-swap without sign extend can be
performed using the \texttt{grevi} \texttt{bswap} pseudo-instructions.)

Note that \texttt{bswaps.h}, and \texttt{bswaps.w} can be emulated with
\texttt{bswap} followed by \texttt{srai}. A separate extension with 48-bit big-endian
load/store instructions probably would make more sense than this two instructions.

\input{bextcref-bswaps}

The hardware for this operation can for the most part share resources from
\texttt{grevi} and the instructions can be encoded in the reserved space in \texttt{shfli}/\texttt{unshfli}.

\vspace{-0.3in}
\begin{center}
\begin{tabular}{M@{}R@{}S@{}R@{}O}
\\
\instbitrange{31}{20} &
\instbitrange{19}{15} &
\instbitrange{14}{12} &
\instbitrange{11}{7} &
\instbitrange{6}{0} \\
\hline
\multicolumn{1}{|c|}{imm[11:0]} &
\multicolumn{1}{c|}{rs1} &
\multicolumn{1}{c|}{funct3} &
\multicolumn{1}{c|}{rd} &
\multicolumn{1}{c|}{opcode} \\
\hline
12 & 5 & 3 & 5 & 7 \\
???????????? & src & BSWAPS  & dest & OP-IMM \\
\end{tabular}
\end{center}


%%%%%%%%%%%%%%%%%%%%%%%%%%%%%%%%%%%%%%%%%%%%%%%%%%%%%%%%%%%%%%%%%%%%%%%%%%%%%%%%%%%%%%%%%%%

\section{Min/max instructions (\texttt{min, max, minu, maxu})}

We define 4 R-type instructions \texttt{min, max, minu, maxu} with the
following semantics:

\input{bextcref-minmax}

\vspace{-0.3in}
\begin{center}
\begin{tabular}{S@{}R@{}R@{}S@{}R@{}O}
\\
\instbitrange{31}{25} &
\instbitrange{24}{20} &
\instbitrange{19}{15} &
\instbitrange{14}{12} &
\instbitrange{11}{7} &
\instbitrange{6}{0} \\
\hline
\multicolumn{1}{|c|}{funct7} &
\multicolumn{1}{c|}{rs2} &
\multicolumn{1}{c|}{rs1} &
\multicolumn{1}{c|}{funct3} &
\multicolumn{1}{c|}{rd} &
\multicolumn{1}{c|}{opcode} \\
\hline
7 & 5 & 5 & 3 & 5 & 7 \\
??????? & src2 & src1 & MIN   & dest & OP    \\
??????? & src2 & src1 & MAX   & dest & OP    \\
??????? & src2 & src1 & MINU  & dest & OP    \\
??????? & src2 & src1 & MAXU  & dest & OP    \\
??????? & src2 & src1 & MINW  & dest & OP-32 \\
??????? & src2 & src1 & MAXW  & dest & OP-32 \\
??????? & src2 & src1 & MINUW & dest & OP-32 \\
??????? & src2 & src1 & MAXUW & dest & OP-32 \\
\end{tabular}
\end{center}


%%%%%%%%%%%%%%%%%%%%%%%%%%%%%%%%%%%%%%%%%%%%%%%%%%%%%%%%%%%%%%%%%%%%%%%%%%%%%%%%%%%%%%%%%%%

\section{Bitfield mask instructions (\texttt{clri, maki})}

{\tt clri} and {\tt maki} both take a source register and two small immediates.
The exact size of those immediates is undefined at the moment. (Max 7 bits each
on RV128.) Together they can be used to implement a bitfield extract operation
that extracts the {\tt size} LSB bits from {\tt rs2} and places them in the
output, filling the remaining bits with the values from {\tt rs1}.

\begin{verbatim}
  clri rd, rs1, size, offset
  maki tmp, rs2, size, offset
  or rd, rd, tmp
\end{verbatim}

{\tt clri} returns the source register with the specified bitfield cleared to 0s.

{\tt maki} returns the source register shifted left and ANDed with the mask.

\input{bextcref-clri-maki}

This instructions are inspired by Motorola 88000 instructions with similar semantics.

%%%%%%%%%%%%%%%%%%%%%%%%%%%%%%%%%%%%%%%%%%%%%%%%%%%%%%%%%%%%%%%%%%%%%%%%%%%%%%%%%%%%%%%%%%%

\section{\hsout{Bitfield join instruction (\texttt{join})}}

This is another approach to the bitfield extract problem. {\tt join} takes two
source registers and one small immediate (max 7 bits on RV128). It takes as many
LSB bits as indicated by the immediate argument from the second source register,
and the remaining bits from the first source register.

The following code is equivalent to the example code for {\tt clri} and {\tt maki}:

\begin{verbatim}
  rori rd, rs1, offset
  join rd, rd, rs2, size
  rori rd, rd, -offset
\end{verbatim}

\input{bextcref-join}

Compared to {\tt clri} and {\tt maki}, the join approach scales better to RV128
in terms of instruction encoding. The {\tt clri} and {\tt maki} approach can
utilize a superscalar architecture because the first two instructions can be
executed in parallel. The {\tt join} approach works better with
macro-op-fusion because it does not spill a temporary register.

%%%%%%%%%%%%%%%%%%%%%%%%%%%%%%%%%%%%%%%%%%%%%%%%%%%%%%%%%%%%%%%%%%%%%%%%%%%%%%%%%%%%%%%%%%%

\section{Conditional move mask instructions (\texttt{cseln, cselz})}

Those instructions either return their first source register or zero, depending on the
value in the second source register. This allows for a conditional move in three instructions:

\begin{verbatim}
  cseln tmp, a, cond
  cselz rd, b, cond
  or rd, rd, tmp
\end{verbatim}

\input{bextcref-csel}

%%%%%%%%%%%%%%%%%%%%%%%%%%%%%%%%%%%%%%%%%%%%%%%%%%%%%%%%%%%%%%%%%%%%%%%%%%%%%%%%%%%%%%%%%%%

\section{In-place conditional move instructions (\texttt{mvnez, mveqz})}

This two instructions copy their first source argument to the destination if the second
argument is non-zero (zero), and otherwise leave the destination register untouched.
This requires less encoding space than \texttt{cmov} and can be implemented more
efficiently (on architectures without register renaming).

%%%%%%%%%%%%%%%%%%%%%%%%%%%%%%%%%%%%%%%%%%%%%%%%%%%%%%%%%%%%%%%%%%%%%%%%%%%%%%%%%%%%%%%%%%%

\section{Carry-less multiply (\texttt{clmul, clmulh})}

Calculate the carry-less product~\cite{CarryLessProduct} of the two arguments. \texttt{clmul}
produces the lower half of the carry-less product and \texttt{clmulh} produces the upper half
of the carry-less product.

We need to coordinate with the crypto extension work group if carry-less multiply should
be part of the B-extension or crypto-extension. (One possible option is to have a vectorized
clmul instruction in the crypto-extension and a scalar version in the B-extension.)

\input{bextcref-clmul}

The classic applications for \texttt{clmul} are the GCM and CRC, but more
applications exist, including the following examples.

There are obvious applications in hashing and pseudo random number generations. For
example, it has been reported that hashes based on carry-less multiplications can
outperform Google's CityHash~\cite{CLHASH}.

\texttt{clmul} of a number with itself inserts zeroes between each input bit. This can
be useful for generating Morton code~\cite{MortonCode}.

\texttt{clmul} of a number with -1 calculates the prefix XOR operation. This can
be useful for decoding gray codes.

Carry-less multiply can be used to implement Erasure code efficiently~\cite{ClmulErasureCode}.

SPARC introduced similar instructions (XMULX, XMULXHI) in SPARC T3 in 2010.

%%%%%%%%%%%%%%%%%%%%%%%%%%%%%%%%%%%%%%%%%%%%%%%%%%%%%%%%%%%%%%%%%%%%%%%%%%%%%%%%%%%%%%%%%%%

\section{Bit-matrix operations (\texttt{bmatxor, bmator, bitmatflip})}

These are 64-bit-only instruction that are not available on RV32. On RV128 they
ignore the upper half of operands and set the upper half of the results to zero.

This instructions interpret a 64-bit value as 8x8 binary matrix.

\texttt{bmatxor} performs a matrix-matrix multiply with boolean AND as multiply
operator and boolean XOR as addition operator.

\texttt{bmator} performs a matrix-matrix multiply with boolean AND as multiply
operator and boolean OR as addition operator.

\texttt{bmatflip} is a unary operator that transposes the source matrix. It is
equivalent to \texttt{zip; zip; zip} on RV64.

Among other things, \texttt{bmatxor}/\texttt{bmator} can be used to perform
arbitrary permutations of bits within each byte (permutation matrix as 2nd
operand) or perform arbitrary permutations of bytes within a 64-bit word
(permutation matrix as 1st operand).

There are similar instructions in Cray XMT~\cite{CrayXMT}. The Cray X1
architecture even has a full 64x64 bit matrix multiply unit~\cite{CrayX1}.

The MMIX architecture has MOR and MXOR instructions with the same semantic.~\cite[p.~182f]{Knuth4A}

\input{bextcref-bmat}
