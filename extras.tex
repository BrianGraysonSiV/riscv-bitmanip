\chapter{Additional Instructions}

This chapter contains additional instructions that are being proposed and/or
currently discussed within the Bitmanip task group.

%%%%%%%%%%%%%%%%%%%%%%%%%%%%%%%%%%%%%%%%%%%%%%%%%%%%%%%%%%%%%%%%%%%%%%%%%%%%%%%%%%%%%%%%%%%

\section{Min/max instructions (\texttt{min, max, minu, maxu})}

We define 4 R-type instructions \texttt{min, max, minu, maxu} with the
following semantics:

\input{bextcref-minmax}

Code that performs saturated arithmetic on a word size $<$ \texttt{XLEN} needs to perform
min/max operations frequently. A simple way of performing those operations without branching
can benefit those programs.

SAT solvers spend a lot of time calculating the absolute value of a signed
integer due to the way CNF literals are commonly encoded~\cite{BiereComm}. With
\texttt{max} (or \texttt{minu}) this is a two-instruction operation:

\begin{verbatim}
  neg a1, a0
  max a0, a0, a1
\end{verbatim}

\vspace{-0.3in}
\begin{center}
\begin{tabular}{S@{}R@{}R@{}S@{}R@{}O}
\\
\instbitrange{31}{25} &
\instbitrange{24}{20} &
\instbitrange{19}{15} &
\instbitrange{14}{12} &
\instbitrange{11}{7} &
\instbitrange{6}{0} \\
\hline
\multicolumn{1}{|c|}{funct7} &
\multicolumn{1}{c|}{rs2} &
\multicolumn{1}{c|}{rs1} &
\multicolumn{1}{c|}{funct3} &
\multicolumn{1}{c|}{rd} &
\multicolumn{1}{c|}{opcode} \\
\hline
7 & 5 & 5 & 3 & 5 & 7 \\
??????? & src2 & src1 & MIN   & dest & OP    \\
??????? & src2 & src1 & MAX   & dest & OP    \\
??????? & src2 & src1 & MINU  & dest & OP    \\
??????? & src2 & src1 & MAXU  & dest & OP    \\
??????? & src2 & src1 & MINW  & dest & OP-32 \\
??????? & src2 & src1 & MAXW  & dest & OP-32 \\
??????? & src2 & src1 & MINUW & dest & OP-32 \\
??????? & src2 & src1 & MAXUW & dest & OP-32 \\
\end{tabular}
\end{center}


%%%%%%%%%%%%%%%%%%%%%%%%%%%%%%%%%%%%%%%%%%%%%%%%%%%%%%%%%%%%%%%%%%%%%%%%%%%%%%%%%%%%%%%%%%%

\section{Carry-less multiply (\texttt{clmul, clmulh})}

Calculate the carry-less product~\cite{CarryLessProduct} of the two arguments. \texttt{clmul}
produces the lower half of the carry-less product and \texttt{clmulh} produces the upper half
of the carry-less product.

We need to coordinate with the crypto extension work group if carry-less multiply should
be part of the B-extension or crypto-extension. (One possible option is to have a vectorized
clmul instruction in the crypto-extension and a scalar version in the B-extension.)

\input{bextcref-clmul}

The classic applications for \texttt{clmul} are the GCM and CRC~\cite{FastCRC,Wolf18A}, but more
applications exist, including the following examples.

There are obvious applications in hashing and pseudo random number generations. For
example, it has been reported that hashes based on carry-less multiplications can
outperform Google's CityHash~\cite{CLHASH}.

\texttt{clmul} of a number with itself inserts zeroes between each input bit. This can
be useful for generating Morton code~\cite{MortonCode}.

\texttt{clmul} of a number with -1 calculates the prefix XOR operation. This can
be useful for decoding gray codes.

Carry-less multiply can be used to implement Erasure code efficiently~\cite{ClmulErasureCode}.

SPARC introduced similar instructions (XMULX, XMULXHI) in SPARC T3 in 2010.

%%%%%%%%%%%%%%%%%%%%%%%%%%%%%%%%%%%%%%%%%%%%%%%%%%%%%%%%%%%%%%%%%%%%%%%%%%%%%%%%%%%%%%%%%%%

\section{CRC instructions (\texttt{crc32.[bhwd], crc32c.[bhwd]})}

Unary CRC instructions that interpret the bits of rs1 as a CRC32/CRC32C state
and perform a polynomial reduction of that state shifted left by 8, 16, 32, or
64 bits.

The instructions return the new CRC32/CRC32C state.

The \texttt{crc32w}/\texttt{crc32cw} instructions are equivalent to executing
\texttt{crc32h}/\texttt{crc32ch} twice, and \texttt{crc32h}/\texttt{crc32ch}
instructions are equivalent to executing \texttt{crc32b}/\texttt{crc32cb}
twice.

All 8 CRC instructions operate on bit-reflected data.

\input{bextcref-crc}

Payload data must be XOR'ed into the LSB end of the state before executing the
CRC instruction. The following code demonstrates the use of \texttt{crc32.b}:

\begin{minipage}{\linewidth}
\begin{verbatim}
  uint32_t crc32_demo(const uint8_t *p, int len)
  {
    uint32_t x = 0xffffffff;
    for (int i = 0; i < len; i++) {
      x = x ^ p[i];
      x = crc32_b(x);
    }
    return ~x;
  }
\end{verbatim}
\end{minipage}

In terms of binary polynomial arithmetic those instructions perform the operation
$$ \texttt{rd}'(x) = (\texttt{rs1}'(x) \cdot x^N) \; \textrm{mod} \; \{\texttt{1}, P'\}(x)\textrm, $$
with $N \in \{8, 16, 32, 64\}$,
$P = \texttt{0xEDB8\_8320}$ for CRC32 and $P = \texttt{0x82F6\_3B78}$ for CRC32C,
$a'$ denoting the XLEN bit reversal of $a$,
and $\{a, b\}$ denoting bit concatenation.
Note that for example for CRC32 $\{\texttt{1}, P'\} = \texttt{0x1\_04C1\_1DB7}$
on RV32 and $\{\texttt{1}, P'\} = \texttt{0x1\_04C1\_1DB7\_0000\_0000}$ on RV64.

These dedicated CRC instructions are meant for RISC-V implementations without fast multiplier
and therefore without fast \texttt{clmul[h]}. For implementations with fast \texttt{clmul[h]}
it is recommended to use the methods described in~\cite{FastCRC} and demonstrated in~\cite{Wolf18A}
that can process XLEN input bits using just one carry-less multiply for arbitrary CRC polynomials.

In applications where those methods are not applicable it is possible to emulate the dedicated CRC
instructions using two carry-less multiplies that implement a Barrett reduction. The following example
implements a replacement for \texttt{crc32.w} (RV32).

\begin{minipage}{\linewidth}
\begin{verbatim}
crc32_w:
  li t0, 0x04C11DB7
  li t1, 0x04D101DF
  brev a0, a0
  clmulh a1, a0, t1
  xor a1, a1, a0
  clmul a0, a1, t0
  brev a0, a0
  ret
\end{verbatim}
\end{minipage}

%%%%%%%%%%%%%%%%%%%%%%%%%%%%%%%%%%%%%%%%%%%%%%%%%%%%%%%%%%%%%%%%%%%%%%%%%%%%%%%%%%%%%%%%%%%

\section{Bit-matrix operations (\texttt{bmatxor, bmator, bitmatflip})}

These are 64-bit-only instruction that are not available on RV32. On RV128 they
ignore the upper half of operands and set the upper half of the results to zero.

This instructions interpret a 64-bit value as 8x8 binary matrix.

\texttt{bmatxor} performs a matrix-matrix multiply with boolean AND as multiply
operator and boolean XOR as addition operator.

\texttt{bmator} performs a matrix-matrix multiply with boolean AND as multiply
operator and boolean OR as addition operator.

\texttt{bmatflip} is a unary operator that transposes the source matrix. It is
equivalent to \texttt{zip; zip; zip} on RV64.

Among other things, \texttt{bmatxor}/\texttt{bmator} can be used to perform
arbitrary permutations of bits within each byte (permutation matrix as 2nd
operand) or perform arbitrary permutations of bytes within a 64-bit word
(permutation matrix as 1st operand).

There are similar instructions in Cray XMT~\cite{CrayXMT}. The Cray X1
architecture even has a full 64x64 bit matrix multiply unit~\cite{CrayX1}.

The MMIX architecture has MOR and MXOR instructions with the same semantic.~\cite[p.~182f]{Knuth4A}

\input{bextcref-bmat}
