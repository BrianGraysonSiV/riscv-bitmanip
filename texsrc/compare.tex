\chapter{Comparison with othert ISAs}

\hrulefill

\textbf{IMPORTANT NOTE:} Some of the discussions below refer to an older draft of the
RISC-V Bitmanip extension and are now out-of-date.

\hrulefill

\section{Comparison with x86 Bit Manipulation ISAs}
\label{x86comp}

The following code snippets implement all instructions from the x86 bit manipulation
ISA extensions ABM, BMI1, BMI2, and TBM using RISC-V code that does not spill any
registers and thus could easily be implemented in a single instruction using macro-op
fusion. (Some of them simply map directly to instructions in this spec and so no
macro-op fusion is needed.) Note that shorter RISC-V code sequences are
sometimes possible if we allow spilling to temporary registers.

ABM added x86 encodings for POPCNT, LZCNT, and TZCNT.\footnote{Depending on if
you ask Intel or AMD you will get different opinion regarding whether LZCNT
and/or TZCNT are part of ABM or BMI1.} The difference between LZCNT and the
80386 instruction BSR, and between TZCNT and the 80386 instruction BSF, is that
the new instructions return the operand size when the input operand is zero,
while BSR and BSF were undefined in that case. The ABM instructions map 1:1 to
Bitmanip instructions. Table~\ref{abm-comp} lists ABM instructions and regular
x86 bit manipulation instructions.

\begin{table}[h]
\centering
\begin{tabular}{lrrl}
\multirow{2}{*}{x86 Instruction} & \multicolumn{2}{c}{Bytes} & \multirow{2}{*}{RISC-V Code} \\
& x86 & RV & \\
\hline
POPCNT       &   5 &  4 & {\tt pcnt a0, a0} \\
\hline
LZCNT / BSR  &   5 &  4 & {\tt clz a0, a0} \\
\hline
TZCNT / BSF  &   5 &  4 & {\tt ctz a0, a0} \\
\hline
BSWAP        &   3 &  4 & {\tt bswap} \\
\hline
ROL          &   4 &  4 & {\tt roli} \\
\hline
ROR          &   4 &  4 & {\tt rori} \\
\hline
BT           &   5 &  4 & {\tt c.srl a0, N} \\
             &     &    & {\tt c.andi a0, 1} \\
\hline
BTC          &   5 & 16 & {\tt rori a0, N} \\
             &     &    & {\tt andi a1, a0, 1} \\
             &     &    & {\tt xori a0, a0, 1} \\
             &     &    & {\tt rori a0, XLEN-N} \\
\hline
BTR          &   5 & 16 & {\tt rori a0, N} \\
             &     &    & {\tt andi a1, a0, 1} \\
             &     &    & {\tt andi a0, a0, -2} \\
             &     &    & {\tt rori a0, XLEN-N} \\
\hline
BTS          &   5 & 16 & {\tt rori a0, N} \\
             &     &    & {\tt andi a1, a0, 1} \\
             &     &    & {\tt ori a0, a0, 1} \\
             &     &    & {\tt rori a0, XLEN-N} \\
\end{tabular}
\caption{Comparison of x86+ABM with Bitmanip}
\label{abm-comp}
\end{table}

BMI1 adds some instructions for trailing bit manipulations, an add-complement instruction,
and a bit field extract instruction that expects the length and start position packed in one
register operand. Our version expects the length in a0, start position in a1, and source
value in a2. See Table~\ref{bmi1-comp}.

\begin{table}[h]
\centering
\begin{tabular}{lrrl}
\multirow{2}{*}{x86 Instruction} & \multicolumn{2}{c}{Bytes} & \multirow{2}{*}{RISC-V Code} \\
& x86 & RV & \\
\hline
ANDN    & 5 &  4 & {\tt andn a0, a2, a1} \\
\hline
BEXTR (regs)  & 5 & 12 & {\tt c.add a0, a1} \\
              &   &    & {\tt slo a0, zero, a0} \\
              &   &    & {\tt c.and a0, a2} \\
              &   &    & {\tt srl a0, a0, a1} \\
\hline
BLSI          & 5 &  6 & {\tt neg a0, a1} \\
              &   &    & {\tt c.and a0, a1} \\
\hline
BLSMSK        & 5 &  6 & {\tt addi a0, a1, -1} \\
              &   &    & {\tt c.xor a0, a1} \\
\hline
BLSR          & 5 &  6 & {\tt addi a0, a1, -1} \\
              &   &    & {\tt c.and a0, a1} \\
\end{tabular}
\caption{Comparison of x86 BMI1 with Bitmanip}
\label{bmi1-comp}
\end{table}

BMI2 adds a few \texttt{*X} instructions that just perform the indicated
operation without changing any flags. RISC-V does not use flags, so this
instructions trivially just map to their regular RISC-V counterparts. In
addition to those instructions, BMI2 adds bit extract and deposit instructions
and an instruction to clear high bits above a given bit index. See Table~\ref{bmi2-comp}.

\begin{table}[h]
\centering
\begin{tabular}{lrrl}
\multirow{2}{*}{x86 Instruction} & \multicolumn{2}{c}{Bytes} & \multirow{2}{*}{RISC-V Code} \\
& x86 & RV & \\
\hline
BZHI     & 5 &  6 & {\tt slo a0, zero, a2} \\
         &   &    & {\tt c.and a0, a1} \\
\hline
PDEP     & 5 &  4 & {\tt bdep} \\
\hline
PEXT     & 5 &  4 & {\tt bext} \\
\hline
MULX     & 5 &  4 & {\tt mul} \\
\hline
RORX     & 6 &  4 & {\tt rori} \\
\hline
SARX     & 5 &  4 & {\tt sra} \\
\hline
SHRX     & 5 &  4 & {\tt srl} \\
\hline
SHLX     & 5 &  4 & {\tt sll} \\
\end{tabular}
\caption{Comparison of x86 BMI2 with Bitmanip}
\label{bmi2-comp}
\end{table}

Finally, TBM was a short-lived x86 ISA extension introduced by AMD in
Piledriver processors, complementing the trailing bit manipulation instructions
from BMI1. See Table~\ref{tbm-comp}.

\begin{table}[h]
\centering
\begin{tabular}{lrrl}
\multirow{2}{*}{x86 Instruction} & \multicolumn{2}{c}{Bytes} & \multirow{2}{*}{RISC-V Code} \\
& x86 & RV & \\
\hline
BEXTR (imm)  & 7 &  4 & {\tt c.slli a0, (32-START-LEN)} \\
             &   &    & {\tt c.srli a0, (32-LEN)} \\
\hline
BLCFILL      & 5 &  6 & {\tt addi a0, a1, 1} \\
             &   &    & {\tt c.and a0, a1} \\
\hline
BLCI         & 5 &  8 & {\tt addi a0, a1, 1} \\
             &   &    & {\tt c.not a0} \\
             &   &    & {\tt c.or a0, a1} \\
\hline
BLCIC        & 5 & 10 & {\tt addi a0, a1, 1} \\
             &   &    & {\tt andn a0, a1, a0} \\
             &   &    & {\tt c.not a0} \\
\hline
BLCMSK       & 5 &  6 & {\tt addi a0, a1, 1} \\
             &   &    & {\tt c.xor a0, a1} \\
\hline
BLCS         & 5 &  6 & {\tt addi a0, a1, 1} \\
             &   &    & {\tt c.or a0, a1} \\
\hline
BLSFILL      & 5 &  6 & {\tt addi a0, a1, -1} \\
             &   &    & {\tt c.or a0, a1} \\
\hline
BLSIC        & 5 & 10 & {\tt addi a0, a1, -1} \\
             &   &    & {\tt andn a0, a1, a0} \\
             &   &    & {\tt c.not a0} \\
\hline
T1MSKC       & 5 & 10 & {\tt addi a0, a1, +1} \\
             &   &    & {\tt andn a0, a1, a0} \\
             &   &    & {\tt c.not a0} \\
\hline
T1MSK        & 5 &  8 & {\tt addi a0, a1, -1} \\
             &   &    & {\tt andn a0, a0, a1} \\
\end{tabular}
\caption{Comparison of x86 TBM with Bitmanip}
\label{tbm-comp}
\end{table}

\section{Comparison with RI5CY Bit Manipulation ISA}

The following section compares Bitmanip with the RI5CY bit manipulation
instructions as documented in~\cite{Ri5cy}. All RI5CY bit manipulation
instructions (or something very close to their behavior) can be emulated with
Bitmanip using 3 instructions or less.

\subsubsection{RI5CY Instructions {\tt p.extract}, {\tt p.extractu}, {\tt p.extractr}, and {\tt p.extractur}}

These four RI5CY instructions extract bit-fields. The non-{\tt u}-versions sign-extend
the extracted bit-field. This operations can be performed with two shift-immediate
operations. It even fits in a 32-bit word when using compressed instructions (requires
{\tt rd} $=$ {\tt rs}).

\begin{verbatim}
  p.extract rd, rs, len, pos:
    slli rd, rs, (XLEN-pos-len)
    srai rd, rd, (XLEN-len)

  p.extractu rd, rs, len, pos:
    slli rd, rs, (XLEN-pos-len)
    srli rd, rd, (XLEN-len)
\end{verbatim}

The {\tt r}-versions expect the bit-field size in bits 9:5 of the second source
register and the bit-field start in bits 4:0. Instead we use two registers,
$\texttt{rx} = XLEN-pos-len$ and $\texttt{ry} = XLEN-len$.

\begin{verbatim}
  p.extractr:
    sll rd, rs, rx
    sra rd, rd, ry

  p.extractur:
    sll rd, rs, rx
    srl rd, rd, ry
\end{verbatim}

Alternatively, instead of packing length and position into a register, we
can create a mask in a register and then use this mask with {\tt bext}. This
has the advantage over the {\tt sll}+{\tt srl} sequence that the mask only needs
to be generated once and can then be re-used many times, effectively implementing
{\tt p.extractur} in a single instruction.

\begin{verbatim}
  p.extractur:
    slo rMask, zero, rLen
    sll rMask, rMask, rPos
    bext rd, rs, rMask
\end{verbatim}

{\tt p.extractu} can be efficiently emulated with a single XBitfield {\tt bfxp}
instruction (see Chapter~\ref{bfxp}):

\begin{verbatim}
  p.extract rd, rs, len, pos:
    bfxp rd, rs, zero, pos, len, 0
\end{verbatim}

\subsubsection{RI5CY Instructions {\tt p.insert} and {\tt p.insertr}}

These instructions OR the destination register with the {\tt len} LSB bits
from the source register, shifted up by {\tt pos} bits. This can easily
be achieved using three instructions and a temporary register {\tt rt}:

\begin{verbatim}
  p.insert rd, rs, len, pos, rt:
    slli rt, rs, (XLEN-len)
    srli rt, rt, (XLEN-len-pos)
    or rd, rd, rt
\end{verbatim}

The {\tt r}-version of the instruction expects {\tt len} in bits 9:5 of the
second source register and {\tt pos} in bits 4:0. Instead we use two registers,
$\texttt{rx} = XLEN-pos-len$ and $\texttt{ry} = XLEN-len$.

\begin{verbatim}
  p.insertr:
    slli rt, rs, ry
    srli rt, rt, rx
    or rd, rd, rt
\end{verbatim}

Alternatively, instead of packing length and position into a register, we
can create a mask in a register and then use this mask with {\tt bdep}. This
has the advantage over the {\tt sll}+{\tt srl} sequence that the mask only needs
to be generated once and can then be re-used many times.

\begin{verbatim}
  p.extractur:
    slo rMask, zero, rLen
    sll rMask, rMask, rPos
    bdep rt, rs, rMask
    or rd, rd, rt
\end{verbatim}

{\tt p.insert} can be efficiently emulated with a single XBitfield {\tt bfxp}
instruction (see Chapter~\ref{bfxp}), if target region in {\tt rd} contains
only zeros ({\tt bfxp} clears the target region before performing the OR):

\begin{verbatim}
  p.insert rd, rs, len, pos:
    bfxp rd, rs, rd, 0, len, pos
\end{verbatim}

\subsubsection{RI5CY Instructions {\tt p.bclr} and {\tt p.bclrr}}

These instructions clear {\tt len} bits starting with bit {\tt pos}. Using a
temporary register {\tt rt}:

\begin{verbatim}
  p.bclr rd, rs, len, pos, rt:
    sloi rt, zero, len
    slli rt, rt, pos
    andn rd, rs, rt
\end{verbatim}

Or using two registers {\tt rLen} and {\tt rPos}:

\begin{verbatim}
  p.bclrr rd, rs, rLen, rPos, rt:
    slo rt, zero, rLen
    sll rt, rt, rPos
    andn rd, rs, rt
\end{verbatim}

If the mask in {\tt rt} can be pre-computed then a single {\tt andn} instruction
can emulate {\tt p.bclrr}, or a single {\tt and}/{\tt c.and} instruction if the
mask is already inverted.

Or using {\tt bfxp} with $\texttt{rd} = \texttt{rs}$ (see Chapter~\ref{bfxp}):

\begin{verbatim}
  p.bclr rd, len, pos:
    bfxp rd, zero, rd, 0, len, pos
\end{verbatim}

\subsubsection{RI5CY Instructions {\tt p.bset} and {\tt p.bsetr}}

These instructions set {\tt len} bits starting with bit {\tt pos}. They can be
implemented similar to {\tt p.bclr} and {\tt p.bclrr} by replacing {\tt andn}
with {\tt or}:

\begin{verbatim}
  p.bset rd, rs, len, pos, rt:
    sloi rt, zero, len
    slli rt, rt, pos
    or rd, rs, rt

  p.bsetr rd, rs, rLen, rPos, rt:
    slo rt, zero, rLen
    sll rt, rt, rPos
    or rd, rs, rt
\end{verbatim}

If the mask in {\tt rt} can be pre-computed then a single {\tt or}/{\tt c.or} instruction
can emulate {\tt p.bsetr}.

Or using {\tt bfxpc} with $\texttt{rd} = \texttt{rs}$ (see Chapter~\ref{bfxp}):

\begin{verbatim}
  p.bset rd, len, pos:
    bfxpc rd, zero, rd, 0, len, pos
\end{verbatim}

\subsubsection{RI5CY Instructions {\tt p.ff1}, {\tt p.cnt}, and {\tt p.ror}}

These instructions map directly to the Bitmanip instructions {\tt ctz}, {\tt pcnt}, and {\tt ror}.

\subsubsection{RI5CY Instructions {\tt p.fl1}}

This instruction returns the index of the last set bit in {\tt rs}. If {\tt rs} is 0, {\tt rd} will be 0.

Using the arguably more useful definition that the operation should return -1 when {\tt rs} is 0:

\begin{verbatim}
  p.fl1 rd, rs:
    clz rd, rs
    neg rd, rd
    addi rd, rd, 31
\end{verbatim}

Converting a -1 result to 0 to match the exact {\tt p.fl1} behavior:

\begin{verbatim}
    slt rt, rd, zero
    add rd, rd, rt
\end{verbatim}

\subsubsection{RI5CY Instructions {\tt p.clb}}

This instruction counts the number of consecutive 1s or 0s from MSB. If {\tt rs} is 0, {\tt rd} will be 0.

Using the arguably more useful definition that the operation should return XLEN when {\tt rs} is 0 or -1,
and assuming {\tt rd} $\ne$ {\tt rs1}:

\begin{verbatim}
  p.clb:
    srai rd, rs, XLEN-1
    xor rd, rd, rs
    clz rd, rd
\end{verbatim}

Simply add {\tt andi rd, rd, XLEN-1} if {\tt rd} should be 0 when {\tt rs} is 0 or -1.

\section{Comparison with Cray XMT bit operations}

Cray XMT is the 3rd generation of the Cray MTA architecture, a supercomputer
using a barrel processor architecture. The Cray XMT instruction set contains
a few bit manipulation instructions~\cite{CrayXMT}. In this section we compare
the Cray XMT bit manipulation instructions with Bitmanip.

\subsubsection{Bitwise boolean operations}

Cray XMT provides the following instructions for bitwise boolean operations:
{\tt BIT\_AND}, {\tt BIT\_IMP}, {\tt BIT\_NAND}, {\tt BIT\_NIMP}, {\tt
BIT\_NOR}, {\tt BIT\_OR}, {\tt BIT\_XNOR}, and {\tt BIT\_XOR}.

These can trivially emulated using basic RISC-V instructions. (Some of those
XMT instructions have a direct RISC-V equivalent. The others can be emulated
by combining the {\tt not} pseudoinstruction with {\tt and}, {\tt or}, or {\tt xor}.)

\subsubsection{Count Leading/Trailing Zeros/Ones}

The Cray XMT instructions {\tt BIT\_LEFT\_ZEORS} and {\tt BIT\_RIGHT\_ZEORS} are
equivalent to the Bitmanip {\tt clz} and {\tt ctz} instructions.

The {\tt BIT\_LEFT\_ONES} and {\tt BIT\_RIGHT\_ONES} instructions can be emulated
by combining the {\tt not} pseudoinstruction with {\tt clz} and {\tt ctz}.

\subsubsection{Mask generation}

The Cray XMT instruction {\tt BIT\_MASK dest top bot} generates a bitmask
that has the bits in the range $[\texttt{bot} \dots \texttt{top}]$ set
and the rest cleared iff $\texttt{bot} \le \texttt{top}$, and those
bits cleared and the rest set otherwise.

Similar masks can be generated using two instructions in Bitmanip,
using the regular shift instructions and the ``shift ones'' instructions.

\subsubsection{Cmix equivalent}

The Cray XMT {\tt BIT\_MERGE} instruction is equivalent to the XTernarybits
{\tt cmix} instruction, or the {\tt and-andn-or} MIX pattern.

\subsubsection{Population count}

The Cray XMT {\tt BIT\_TALLY} instruction and the Bitmanip {\tt pcnt}
instruction are equivalent.

\subsubsection{Parity instructions}

The Cray XMT {\tt BIT\_ODD\_AND}, {\tt BIT\_ODD\_NIMP}, {\tt BIT\_ODD\_OR},
and {\tt BIT\_ODD\_XOR} instruction perform the indicated bitwise boolean
operation and then compute the parity of the result.

With Bitmanip the parity can be calculated with {\tt pcnt dst, src} followed
by {\tt andi dst, dst, 1}.

\subsubsection{Bit pack/unpack instruction}

The Cray XMT {\tt BIT\_PACK} instruction and the Bitmanip {\tt bext}
instruction are equivalent.

The Cray XMT {\tt BIT\_UNPACK\_1}, {\tt BIT\_UNPACK\_2}, {\tt BIT\_UNPACK\_3},
instruction sequence, when used as intended, is equivalent to the Bitmanip
{\tt bdep} instruction.

\subsubsection{Bit matrix instructions}

The Cray XMT {\tt BIT\_MAT\_} instructions treat a 64-bit value as a 8x8 bit
matrix.

{\tt BIT\_MAT\_TRANSPOSE} is used to transpose such a bit matrix. With
Bitmanip instructions on RV64 this bit permutation can be performed by
applying the {\tt zip} operation three times to the register holding the
matrix (see also~\ref{transposebitboard}).

The Cray XMT instructions {\tt BIT\_MAT\_OR} and {\tt BIT\_MAT\_XOR} perform a
matrix-matrix multiply between such bit matrices, where AND replaces scalar
multiply and OR/XOR replaces scalar addition.

Bitmanip does not provide a similar operation. However, the two example
applications given in the Cray XMT documentation~\cite[p.~81]{CrayXMT}
are reversing the byte order in a word and reversing the bit order in
each byte. In Bitmanip those operations are performed by the {\tt bswap}
and the {\tt brev.b} {\tt grevi}-pseudoinstructions.

