\chapter{RISC-V Bitmanip Extension}
\label{bext}

In the proposals provided in this chapter, the C code examples are for
illustration purposes only. They are not optimal implementations, but are
intended to specify the desired functionality.
% See Section~\ref{fastc} for fast C code for use in emulators.

The final standard will likely define a range of Z-extensions for different bit
manipulation instructions, with the ``B'' extension itself being a mix of
instructions from those Z-extensions. It is unclear as of yet what this will
look like exactly, but it will probably look something like table~\ref{zbexts}.

\begin{table}[!h]
\begin{center}
\begin{tabular}{lll}
Extension & RV32/RV64 & RV64 only \\
\hline
Zbb (*)
 & {\tt clz, ctz, pcnt            } & {\tt clzw, ctzw, pcntw         } \\
 & {\tt min, minu, max, maxu      } & {\tt                           } \\
 & {\tt sext.b, sext.h            } & {\tt zext.h, zext.w            } \\
 & {\tt andn, orn, xnor           } & {\tt                           } \\
 & {\tt rol, ror, rori            } & {\tt rolw, rorw, roriw         } \\
 & {\tt rev8, rev, orc.b          } & {\tt                           } \\
\hline
Zbp
 & {\tt andn, orn, xnor           } & {\tt                           } \\
 & {\tt pack, packu, packh        } & {\tt packw, packuw             } \\
 & {\tt rol, ror, rori            } & {\tt rolw, rorw, roriw         } \\
 & {\tt grev, grevi               } & {\tt grevw, greviw             } \\
 & {\tt gorc, gorci               } & {\tt gorcw, gorciw             } \\
 & {\tt shfl, shfli               } & {\tt shflw                     } \\
 & {\tt unshfl, unshfli           } & {\tt unshflw                   } \\
 & {\tt xperm.n, xperm.b, xperm.h } & {\tt xperm.w                   } \\
\hline
Zbs (*)
 & {\tt sbset, sbseti             } & {\tt sbsetw, sbsetiw           } \\
 & {\tt sbclr, sbclri             } & {\tt sbclrw, sbclriw           } \\
 & {\tt sbinv, sbinvi             } & {\tt sbinvw, sbinviw           } \\
 & {\tt sbext, sbexti             } & {\tt sbextw                    } \\
\hline
Zba (*)
 & {\tt sh1add                    } & {\tt sh1addu.w                 } \\
 & {\tt sh2add                    } & {\tt sh2addu.w                 } \\
 & {\tt sh3add                    } & {\tt sh3addu.w                 } \\
 & {\tt                           } & {\tt addu.w, slliu.w           } \\
\hline
Zbe
 & {\tt bext, bdep                } & {\tt bextw, bdepw              } \\
 & {\tt pack, packh               } & {\tt packw                     } \\
\hline
Zbf
 & {\tt bfp                       } & {\tt bfpw                      } \\
 & {\tt pack, packh               } & {\tt packw                     } \\
\hline
Zbc (*)
 & {\tt clmul, clmulh, clmulr     } & {\tt                           } \\
\hline
Zbm
 & {\tt                           } & {\tt bmator, bmatxor, bmatflip } \\
 & {\tt                           } & {\tt unzip16, unzip8           } \\
 & {\tt                           } & {\tt pack, packu               } \\
\hline
Zbr
 & {\tt crc32.b, crc32c.b         } & {\tt                           } \\
 & {\tt crc32.h, crc32c.h         } & {\tt                           } \\
 & {\tt crc32.w, crc32c.w         } & {\tt                           } \\
 & {\tt                           } & {\tt crc32.d, crc32c.d         } \\
\hline
Zbt
 & {\tt cmov, cmix                } & {\tt                           } \\
 & {\tt fsl, fsr, fsri            } & {\tt fslw, fsrw, fsriw         } \\
\hline
B
 & \multicolumn{2}{l}{All of the above except Zbr and Zbt} \\
\hline
Notes:\\
\multicolumn{3}{l}{- * means the extensions are expected to be unchanged in the official version.} \\
\end{tabular}
\caption{{\tt Zb*} extensions instruction listings}
\end{center}
\label{zbexts}
\end{table}

The main open questions are:
\begin{itemize}
\item Which of the ``Zb*'' extensions should be included in ``B''?
\item Which ``Zbp'' pseudo-ops should be included in ``Zbb''?
\end{itemize}

These decisions will be informed in big part by evaluations of the cost and
added value for the individual instructions.

For the purpose of tool-chain development ``B'' is currently everything.

For extensions that only implement certain pseudo-instructions (such as ``Zbb''
implements {\tt rev8} and {\tt rev}, which are pseudo-instructions for {\tt grevi rd, rs1, -8}
and {\tt grevi rd, rs1, -1} respectively, the same binary encoding is used for those
instructions as are used on a core with full support for the {\tt grev[i]} instruction.

Like in the base ISA, we add {\tt *W} instruction variants on RV64 with the
semantic of the matching RV32 instruction. Those instructions ignore the upper
32 bit of their input and sign-extend their 32 bit output values. The {\tt *W}
instruction is omitted when the 64-bit instruction produces the same result as
the {\tt *W} instruction would when the 64-bit instruction is fed sign-extended
32 bit values.

%%%%%%%%%%%%%%%%%%%%%%%%%%%%%%%%%%%%%%%%%%%%%%%%%%%%%%%%%%%%%%%%%%%%%%%%%%%%%%%%%%%%%%%%%%%

\section{Basic bit manipulation instructions}

%%%%%%%%%%%%%%%%%%%%%%%%%%%%%%%%%%%%%%%%%%%%%%%%%%%%%%%%%%%%%%%%%%%%%%%%%%%%%%%%%%%%%%%%%%%

\subsection{Count Leading/Trailing Zeros (\texttt{clz, ctz})}

\begin{rvb}
  RV32, RV64:
    clz rd, rs
    ctz rd, rs

  RV64 only:
    clzw rd, rs
    ctzw rd, rs
\end{rvb}

The {\tt clz} operation counts the number of 0 bits at the MSB end of the
argument.  That is, the number of 0 bits before the first 1 bit counting from
the most significant bit. If the input is 0, the output is XLEN. If the input
is -1, the output is 0.

The {\tt ctz} operation counts the number of 0 bits at the LSB end of the
argument. If the input is 0, the output is XLEN. If the input is -1, the
output is 0.

\input{bextcref-clz-ctz}

The expression {\tt XLEN-1-clz(x)} evaluates to the index of the most significant
set bit, also known as integer base-2 logarithm, or -1 if {\tt x} is zero.

These instructions are commonly used for scanning bitmaps for set bits, for
example in {\tt malloc()}, in binary GCD, or in priority queues such as the
{\tt sched\_find\_first\_bit()} function used in the Linux kernel real-time
scheduler.

Another common applications include normalization in fixed-point code and soft
float libraries, null suppression in data compression.

% \subsection{References}
%
% https://en.wikipedia.org/wiki/Find\_first\_set\#CLZ
%
% https://fgiesen.wordpress.com/2013/10/18/bit-scanning-equivalencies/

%%%%%%%%%%%%%%%%%%%%%%%%%%%%%%%%%%%%%%%%%%%%%%%%%%%%%%%%%%%%%%%%%%%%%%%%%%%%%%%%%%%%%%%%%%%

\subsection{Count Bits Set (\texttt{pcnt})}

\begin{rvb}
  RV32, RV64:
    pcnt rd, rs

  RV64 only:
    pcntw rd, rs
\end{rvb}

This instruction counts the number of 1 bits in a register. This operations is known as
population count, popcount, sideways sum, bit summation, or Hamming weight.~\cite{HammingWeight,Warren12}

\input{bextcref-pcnt}

%%%%%%%%%%%%%%%%%%%%%%%%%%%%%%%%%%%%%%%%%%%%%%%%%%%%%%%%%%%%%%%%%%%%%%%%%%%%%%%%%%%%%%%%%%%

\subsection{Logic-with-negate (\texttt{andn}, \texttt{orn}, \texttt{xnor})}

\begin{rvb}
  RV32, RV64:
    andn rd, rs1, rs2
    orn  rd, rs1, rs2
    xnor rd, rs1, rs2
\end{rvb}

This instructions implement AND, OR, and XOR with the 2nd arument inverted.

\input{bextcref-andn}

This can use the existing inverter on rs2 in the ALU that's already there to
implement subtract.

Among other things, those instructions allow implementing the ``trailing bit
manipulation'' code patterns in two instructions each. For example, {\tt (x -
1) \& \textasciitilde{}x} produces a mask from trailing zero bits in {\tt x}.

%%%%%%%%%%%%%%%%%%%%%%%%%%%%%%%%%%%%%%%%%%%%%%%%%%%%%%%%%%%%%%%%%%%%%%%%%%%%%%%%%%%%%%%%%%%

\subsection{Pack two words in one register (\texttt{pack, packu, packh})}

\begin{rvb}
  RV32, RV64:
    pack  rd, rs1, rs2
    packu rd, rs1, rs2
    packh rd, rs1, rs2

  RV64 only:
    packw  rd, rs1, rs2
    packuw rd, rs1, rs2
\end{rvb}

The {\tt pack} instruction packs the XLEN/2-bit lower halves of rs1 and rs2 into
rd, with rs1 in the lower half and rs2 in the upper half.

\input{bextcref-pack}

The {\tt packu} instruction packs the upper halves of rs1 and rs2 into rd.

\input{bextcref-packu}

And the {\tt packh} instruction packs the LSB bytes of rs1 and rs2 into the 16 LSB bits
of rd, zero extending the rest of rd.

\input{bextcref-packh}

Applications include XLEN/2-bit funnel shifts, zero-extend XLEN/2 bit values, duplicate the lower
XLEN/2 bits (e.g. for mask creation), loading unsigned 32 constants on RV64, and packing
C structs that fit in a register and are therefore passed in a register according to the RISC-V
calling convention.

\begin{minipage}{\linewidth}
\begin{verbatim}
  ; Constructing a 32-bit int from four bytes (RV32)
  packh a0, a0, a1
  packh a1, a2, a3
  pack a0, a0, a1
\end{verbatim}
\end{minipage}

\begin{minipage}{\linewidth}
\begin{verbatim}
  ; Load 0xffff0000ffff0000 on RV64
  lui rd, 0xffff0
  pack rd, rd, rd
\end{verbatim}
\end{minipage}

\begin{minipage}{\linewidth}
\begin{verbatim}
  ; Same as FSLW on RV64
  pack rd, rs1, rs3
  rol rd, rd, rs2
  addiw rd, rd, 0
\end{verbatim}
\end{minipage}

\begin{minipage}{\linewidth}
\begin{verbatim}
  ; Clear the upper half of rd
  pack rd, rd, zero
\end{verbatim}
\end{minipage}

Paired with {\tt shfli/unshfli} and the other bit permutation instructions,
pack can interleave arbitrary power-of-two chunks of {\tt rs1} and {\tt rs2}. For
example, interleaving the bytes in the lower halves of {\tt rs1} and {\tt rs2}:

\begin{minipage}{\linewidth}
\begin{verbatim}
  pack rd, rs1, rs2
  zip8 rd, rd
\end{verbatim}
\end{minipage}

{\tt pack} is most commonly used to zero-extend words $<$XLEN.
For this purpose we define the following assembler pseudo-ops:

\begin{minipage}{\linewidth}
\begin{verbatim}
  RV32:
    zext.b rd, rs   ->    andi  rd, rs, 255
    zext.h rd, rs   ->    pack  rd, rs, zero

  RV64:
    zext.b rd, rs   ->    andi  rd, rs, 255
    zext.h rd, rs   ->    packw rd, rs, zero
    zext.w rd, rs   ->    pack  rd, rs, zero

  RV128:.
    zext.b rd, rs   ->    andi  rd, rs, 255
    zext.h rd, rs   ->    packw rd, rs, zero
    zext.w rd, rs   ->    packd rd, rs, zero
    zext.d rd, rs   ->    pack  rd, rs, zero
\end{verbatim}
\end{minipage}

%%%%%%%%%%%%%%%%%%%%%%%%%%%%%%%%%%%%%%%%%%%%%%%%%%%%%%%%%%%%%%%%%%%%%%%%%%%%%%%%%%%%%%%%%%%

\subsection{Min/max instructions (\texttt{min, max, minu, maxu})}

\begin{rvb}
  RV32, RV64:
    min  rd, rs1, rs2
    max  rd, rs1, rs2
    minu rd, rs1, rs2
    maxu rd, rs1, rs2
\end{rvb}

We define 4 R-type instructions \texttt{min, max, minu, maxu} with the
following semantics:

\input{bextcref-minmax}

Code that performs saturated arithmetic on a word size $<$ \texttt{XLEN} needs to perform
min/max operations frequently. A simple way of performing those operations without branching
can benefit those programs.

Some applications spend a lot of time on calculating the absolute values of
signed integers. One example of that would be SAT solvers, due to the way CNF
literals are commonly encoded~\cite{BiereComm}. With \texttt{max} (or
\texttt{minu}) this is a two-instruction operation:

\begin{minipage}{\linewidth}
\begin{verbatim}
  neg a1, a0
  max a0, a0, a1
\end{verbatim}
\end{minipage}

%%%%%%%%%%%%%%%%%%%%%%%%%%%%%%%%%%%%%%%%%%%%%%%%%%%%%%%%%%%%%%%%%%%%%%%%%%%%%%%%%%%%%%%%%%%

\subsection{Sign-extend instructions (\texttt{sext.b, sext.h})}

\begin{rvb}
  RV32, RV64:
    sext.b rd, rs
    sext.h rd, rs
\end{rvb}

For sign-extending a byte or half-word we define two unary instructions:

\input{bextcref-sext}

Additionally, we define pseudo-instructions for zero extending bytes or
half-words and for zero- and sign-extending words:

\begin{minipage}{\linewidth}
\begin{verbatim}
  RV32:
    zext.b rd, rs   ->   andi rd, rs, 255
    zext.h rd, rs   ->   pack rd, rs, zero

  RV64:
    zext.b rd, rs   ->   andi rd, rs, 255
    zext.h rd, rs   ->   packw rd, rs, zero
    zext.w rd, rs   ->   pack rd, rs, zero
    sext.w rd, rs   ->   addiw rd, rs, 0
\end{verbatim}
\end{minipage}

Sign extending 8-bit and 16-bit values is needed when calling a function that
accepts an 8-bit or 16-bit signed argument, because the RISC-V calling
conventions dictates that such an argument must be passed in sign-extended
form.

%%%%%%%%%%%%%%%%%%%%%%%%%%%%%%%%%%%%%%%%%%%%%%%%%%%%%%%%%%%%%%%%%%%%%%%%%%%%%%%%%%%%%%%%%%%

\subsection{Single-bit instructions (\texttt{sbset, sbclr, sbinv, sbext})}

\begin{rvb}
  RV32, RV64:
    sbset  rd, rs1, rs2
    sbclr  rd, rs1, rs2
    sbinv  rd, rs1, rs2
    sbext  rd, rs1, rs2
    sbseti rd, rs1, imm
    sbclri rd, rs1, imm
    sbinvi rd, rs1, imm
    sbexti rd, rs1, imm

  RV64:
    sbsetw  rd, rs1, rs2
    sbclrw  rd, rs1, rs2
    sbinvw  rd, rs1, rs2
    sbextw  rd, rs1, rs2
    sbsetiw rd, rs1, imm
    sbclriw rd, rs1, imm
    sbinviw rd, rs1, imm
\end{rvb}

We define 4 single-bit instructions \texttt{sbset} (set), \texttt{sbclr} (clear),
\texttt{sbinv} (invert), and \texttt{sbext} (extract), and their immediate-variants,
with the following semantics:

\input{bextcref-sbx}

%%%%%%%%%%%%%%%%%%%%%%%%%%%%%%%%%%%%%%%%%%%%%%%%%%%%%%%%%%%%%%%%%%%%%%%%%%%%%%%%%%%%%%%%%%%

\subsection{Shift Ones (Left/Right) (\texttt{slo,\ sloi,\ sro,\ sroi})}

\begin{rvb}
  RV32, RV64:
    slo  rd, rs1, rs2
    sro  rd, rs1, rs2
    sloi rd, rs1, imm
    sroi rd, rs1, imm

  RV64 only:
    slow  rd, rs1, rs2
    srow  rd, rs1, rs2
    sloiw rd, rs1, imm
    sroiw rd, rs1, imm
\end{rvb}

These instructions are similar to shift-logical operations from the base
spec, except instead of shifting in zeros, they shift in ones.

\input{bextcref-sxo}

ISAs with flag registers often have a "Shift in Carry" or "Rotate through Carry" instruction.
Arguably a "Shift Ones" is an equivalent on an ISA like RISC-V that avoids such flag registers.

The main application for the Shift Ones instruction is mask generation.

When implementing this circuit, the only change in the ALU over a
standard logical shift is that the value shifted in is not zero, but is
a 1-bit register value that has been forwarded from the high bit of the
instruction decode. This creates the desired behavior on both logical
zero-shifts and logical ones-shifts.

%%%%%%%%%%%%%%%%%%%%%%%%%%%%%%%%%%%%%%%%%%%%%%%%%%%%%%%%%%%%%%%%%%%%%%%%%%%%%%%%%%%%%%%%%%%

\section{Bit permutation instructions}

The following sections describe 3 types of bit permutation instructions: Rotate
shift, generalized reverse, and generalized shuffle.

A bit permutation essentially is applying an invertible function to the bit addresses. (Bit
addresses are 5 bit values on RV32 and 6 bit values on RV64.)

Rotate shift by $k$ is simply addition ({\tt rol}) or subtraction ({\tt ror}) modulo XLEN.
$$ i'_\mathrm{rot} := i \pm k \mod \mathrm{XLEN} $$

Generalized reverse with control argument $k$ is simply XOR-ing the bit addresses with $k$:
$$ i'_\mathrm{grev} := i \oplus k $$

And generalized shuffle is performing a bit permutation on the bits of the bit addresses:
$$ i'_\mathrm{shfl} := \mathrm{perm}_k(i) $$

With the caveat that a single {\tt shfl}/{\tt unshfl} instruction can only
perferm a certain sub-set of bit address permutations, but a sequence of 4 {\tt
shfl}/{\tt unshfl} instructions can perform any of the 120 such permutations on
RV32, and a sequence of 5 {\tt shfl}/{\tt unshfl} instructions can perform any
of the 720 such permutations on RV64.

Combining those three types of operations makes a vast number of bit
permutations accessible within only a few instructions~\cite{Wolf19A} (see
Table~\ref{numperms}).

\begin{table}[!h]
\begin{center}
\begin{tabular}{r|rrr|r|r}
N & ROT-only & GREV-only & SHFL-only & ROT+GREV & ROT+GREV+SHFL \\
\hline
  0 &   1 &   1 &   1 &           1 &                           1 \\
  1 &  32 &  32 &  24 &          62 &                          85 \\
  2 & --- & --- &  86 &         864 &                      3\,030 \\
  3 & --- & --- & 119 &      4\,640 &                     78\,659 \\
  4 & --- & --- & 120 &     23\,312 &                 2\,002\,167 \\
  5 & --- & --- & --- &     92\,192 &                50\,106\,844 \\
  6 & --- & --- & --- &    294\,992 &            1\,234\,579\,963 \\
  7 & --- & --- & --- &    703\,744 & est.      30\,000\,000\,000 \\
  8 & --- & --- & --- & 1\,012\,856 & est.     700\,000\,000\,000 \\
  9 & --- & --- & --- & 1\,046\,224 & est. 15\,000\,000\,000\,000 \\
 10 & --- & --- & --- & 1\,048\,576 &  $\cdots\;\cdots\;\cdots\;\cdots\;\cdots$ \\
 11 & --- & --- & --- &         --- &  $\cdots\;\cdots\;\cdots\;\cdots\;\cdots$ \\
\end{tabular}
\end{center}
\caption{Number of permutations reachable with N permutation instructions on RV32. \hbox{``---''}~indicates
that additional instructions don't increase the space of reachable permutations.}
\label{numperms}
\end{table}

Sequences of {\tt ror}, {\tt grev}, and {\tt [un]shfl} instructions can
generate any arbitrary bit permutation. Often in surprising ways. For example,
the following sequence swaps the two LSB bits of {\tt a0}:

\begin{table}[t]
\begin{center}
\begin{tabular}{l|c|c}
Instruction & State (XLEN=8) & Bit-Index Op \\
\hline
{\it initial value}      & 7 6 5 4 3 2 1 0 & --- \\
{\tt rori a0, a0, 2}     & 1 0 7 6 5 4 3 2 & $i' := i-2$ \\
{\tt unshfli a0, a0, -1} & 1 7 5 3 0 6 4 2 & $i' := \mathrm{ror}(i)$ \\
{\tt roli a0, a0, 1}     & 7 5 3 0 6 4 2 1 & $i' := i+1$ \\
{\tt shfli a0, a0, -1}   & 7 6 5 4 3 2 0 1 & $i' := \mathrm{rol}(i)$ \\
\end{tabular}
\end{center}
\caption{Breakdown of the {\tt ror}+{\tt [un]shfl} sequence for swapping the
two LSB bits of a word, using XLEN=8 for simplicity.}
\label{permdemo}
\end{table}

\begin{minipage}{\linewidth}
\begin{verbatim}
  rori a0, a0, 2
  unshfli a0, a0, -1
  roli a0, a0, 1
  shfli a0, a0, -1
\end{verbatim}
\end{minipage}

The mechanics of this sequence is closely related to the fact that {\tt
rol(ror(x-2)+1)} is a function that maps 1 to 0 and 0 to 1 and every other
number to itself. (With {\tt rol} and {\tt ror} denoting 1-bit rotate left and
right shifts respectively.) See Table~\ref{permdemo} for details.

The numbers in the right column of Table~\ref{numperms} might appear large,
but they are tiny in comparison to the total number of 32-bit permutations
($2.63\cdot10^{35}$). Fortunately the {\tt [un]shfl} instruction ``explores''
permutations that involve fields that have a size that's a power-of-two,
and/or moves that are a power-of-two first, which means we get shorter sequences
for permutations we more often care about in real-world applications.

For example, there are 24 ways of arranging the four bytes in a 32-bit word.
{\tt ror}, {\tt grev}, and {\tt [un]shfl} can perform any of those permutations
in at most 3 instructions. See Table~\ref{permbytes} for a list of those 24
sequences.

There are 40320 ways of arranging the eight bytes in a 64-bit word or the eight
nibbles in a 32-bit word. {\tt ror}, {\tt grev}, and {\tt [un]shfl} can perform
any of those permutations in at most 9 instructions.~\cite{Wolf19A}

Besides the more-or-less arbitrary permutations we get from combining {\tt
ror}, {\tt grev}, and {\tt [un]shfl} in long sequences, there are of course
many cases where just one of these instructions does exactly what we need.
For {\tt grev} and {\tt [un]shfl} we define {\tt rev*} and {\tt [un]zip*}
pseudo instructions for the most common use-cases.

%%%%%%%%%%%%%%%%%%%%%%%%%%%%%%%%%%%%%%%%%%%%%%%%%%%%%%%%%%%%%%%%%%%%%%%%%%%%%%%%%%%%%%%%%%%

\subsection{Rotate (Left/Right) (\texttt{rol,\ ror,\ rori})}

\begin{rvb}
  RV32, RV64:
    ror  rd, rs1, rs2
    rol  rd, rs1, rs2
    rori rd, rs1, imm

  RV64 only:
    rorw  rd, rs1, rs2
    rolw  rd, rs1, rs2
    roriw rd, rs1, imm
\end{rvb}

These instructions are similar to shift-logical operations from the base
spec, except they shift in the values from the opposite side of the
register, in order. This is also called `circular shift'.

\input{bextcref-rox}

%%%%%%%%%%%%%%%%%%%%%%%%%%%%%%%%%%%%%%%%%%%%%%%%%%%%%%%%%%%%%%%%%%%%%%%%%%%%%%%%%%%%%%%%%%%

\begin{figure}[t]
\begin{center}
\input{bextcref-printperm-ror}
\end{center}
\caption{\texttt{ror} permutation network}
\label{permnet-ror}
\end{figure}

\subsection{Generalized Reverse (\texttt{grev}, \texttt{grevi}, \texttt{rev})}
\label{grev}

\begin{rvb}
  RV32, RV64:
    grev  rd, rs1, rs2
    grevi rd, rs1, imm

  RV64 only:
    grevw  rd, rs1, rs2
    greviw rd, rs1, imm
\end{rvb}

This instruction provides a single hardware instruction that can implement all
of byte-order swap, bitwise reversal, short-order-swap, word-order-swap
(RV64), nibble-order swap, bitwise reversal in a byte, etc, all from a single
hardware instruction.

The Generalized Reverse (GREV) operation iteratively checks each bit $i$ in the
2nd argument from $i=0$ to $log_2(\textrm{XLEN})-1$, and if the corresponding bit is
set, swaps each adjacent pair of $2^i$ bits.

\begin{figure}[t]
\begin{center}
\input{bextcref-printperm-grev}
\end{center}
\caption{\texttt{grev} permutation network}
\label{permnet-grev}
\end{figure}

\input{bextcref-grev}

The above pattern should be intuitive to understand in order to extend
this definition in an obvious manner for RV128.

\begin{table}[t]
\begin{small}
\begin{center}
\begin{tabular}{r l p{0.5in} r l p{0.3in} r l}

\multicolumn{2}{c}{RV32} & &
\multicolumn{5}{c}{RV64} \\

\cline{1-2}
\cline{4-8}

\multicolumn{1}{c}{shamt} & Instruction & &
\multicolumn{1}{c}{shamt} & Instruction & &
\multicolumn{1}{c}{shamt} & Instruction \\

\cline{1-2}
\cline{4-5}
\cline{7-8}

 0: 00000 & ---           &   &  0: 000000 & ---           &   & 32: 100000 & {\tt rev32} \\
 1: 00001 & {\tt rev.p}   &   &  1: 000001 & {\tt rev.p}   &   & 33: 100001 & ---         \\
 2: 00010 & {\tt rev2.n}  &   &  2: 000010 & {\tt rev2.n}  &   & 34: 100010 & ---         \\
 3: 00011 & {\tt rev.n}   &   &  3: 000011 & {\tt rev.n}   &   & 35: 100011 & ---         \\
 4: 00100 & {\tt rev4.b}  &   &  4: 000100 & {\tt rev4.b}  &   & 36: 100100 & ---         \\
 5: 00101 & ---           &   &  5: 000101 & ---           &   & 37: 100101 & ---         \\
 6: 00110 & {\tt rev2.b}  &   &  6: 000110 & {\tt rev2.b}  &   & 38: 100110 & ---         \\
 7: 00111 & {\tt rev.b}   &   &  7: 000111 & {\tt rev.b}   &   & 39: 100111 & ---         \\

\cline{1-2}
\cline{4-5}
\cline{7-8}

 8: 01000 & {\tt rev8.h}  &   &  8: 001000 & {\tt rev8.h}  &   & 40: 101000 & ---         \\
 9: 01001 & ---           &   &  9: 001001 & ---           &   & 41: 101001 & ---         \\
10: 01010 & ---           &   & 10: 001010 & ---           &   & 42: 101010 & ---         \\
11: 01011 & ---           &   & 11: 001011 & ---           &   & 43: 101011 & ---         \\
12: 01100 & {\tt rev4.h}  &   & 12: 001100 & {\tt rev4.h}  &   & 44: 101100 & ---         \\
13: 01101 & ---           &   & 13: 001101 & ---           &   & 45: 101101 & ---         \\
14: 01110 & {\tt rev2.h}  &   & 14: 001110 & {\tt rev2.h}  &   & 46: 101110 & ---         \\
15: 01111 & {\tt rev.h}   &   & 15: 001111 & {\tt rev.h}   &   & 47: 101111 & ---         \\

\cline{1-2}
\cline{4-5}
\cline{7-8}

16: 10000 & {\tt rev16}   &   & 16: 010000 & {\tt rev16.w} &   & 48: 110000 & {\tt rev16} \\
17: 10001 & ---           &   & 17: 010001 & ---           &   & 49: 110001 & ---         \\
18: 10010 & ---           &   & 18: 010010 & ---           &   & 50: 110010 & ---         \\
19: 10011 & ---           &   & 19: 010011 & ---           &   & 51: 110011 & ---         \\
20: 10100 & ---           &   & 20: 010100 & ---           &   & 52: 110100 & ---         \\
21: 10101 & ---           &   & 21: 010101 & ---           &   & 53: 110101 & ---         \\
22: 10110 & ---           &   & 22: 010110 & ---           &   & 54: 110110 & ---         \\
23: 10111 & ---           &   & 23: 010111 & ---           &   & 55: 110111 & ---         \\

\cline{1-2}
\cline{4-5}
\cline{7-8}

24: 11000 & {\tt rev8}    &   & 24: 011000 & {\tt rev8.w}  &   & 56: 111000 & {\tt rev8}  \\
25: 11001 & ---           &   & 25: 011001 & ---           &   & 57: 111001 & ---         \\
26: 11010 & ---           &   & 26: 011010 & ---           &   & 58: 111010 & ---         \\
27: 11011 & ---           &   & 27: 011011 & ---           &   & 59: 111011 & ---         \\
28: 11100 & {\tt rev4}    &   & 28: 011100 & {\tt rev4.w}  &   & 60: 111100 & {\tt rev4}  \\
29: 11101 & ---           &   & 29: 011101 & ---           &   & 61: 111101 & ---         \\
30: 11110 & {\tt rev2}    &   & 30: 011110 & {\tt rev2.w}  &   & 62: 111110 & {\tt rev2}  \\
31: 11111 & {\tt rev}     &   & 31: 011111 & {\tt rev.w}   &   & 63: 111111 & {\tt rev}   \\
\end{tabular}
\end{center}
\end{small}
\caption{Pseudo-instructions for {\tt grevi} instruction}
\label{grevi-modes}
\end{table}

The {\tt grev} operation can easily be implemented using a permutation
network with $log_2(\textrm{XLEN})$ stages. Figure~\ref{permnet-ror}
shows the permutation network for {\tt ror} for reference.
Figure~\ref{permnet-grev} shows the permutation network for {\tt grev}.

Pseudo-instructions are provided for the most common GREVI use-cases. Their names
consist of a prefix and and optional suffix. Each prefix and suffix corresponds
to a bit mask (see Table~\ref{grevi-names}). The GREVI control word is obtained
by AND-ing the two masks together.

\begin{table}[!h]
\begin{center}
\begin{tabular}{lcp{1cm}rcl}
Prefix & Mask & & Suffix & Mask \\
\hline
{\tt rev}   & 111111 & &      --- & 111111 \\
{\tt rev2}  & 111110 & & {\tt .w} & 011111 & ({\tt w} = word)\\
{\tt rev4}  & 111100 & & {\tt .h} & 001111 & ({\tt h} = half word)\\
{\tt rev8}  & 111000 & & {\tt .b} & 000111 & ({\tt b} = byte)\\
{\tt rev16} & 110000 & & {\tt .n} & 000011 & ({\tt n} = nibble)\\
{\tt rev32} & 100000 & & {\tt .p} & 000001 & ({\tt p} = pair)\\
\end{tabular}
\end{center}
\caption{Naming scheme for {\tt grevi} pseudo-instructions. The prefix and
suffix masks are ANDed to compute the immediate argument.}
\label{grevi-names}
\end{table}

In other words, the prefix controls the number of zero bits at the LSB end of the
control word, and the suffix controls the number of zeros at the MSB end of the control
word.

{\tt rev8} reverses the order of bytes in a word, thus performs endianness conversion. This
is equivalent to the ARM {\tt REV} instructions or {\tt BSWAP} on x86. ARM also has instructions
for swapping the bytes in 16-bit and 32-bit words, and reversing the bit order (see table~\ref{grevi-cmp}).

\begin{table}[h]
\begin{center}
\begin{tabular}{l|l|l}
RISC-V & ARM & X86 \\
\hline
{\tt rev}    & {\tt RBIT}  & --- \\
{\tt rev8.h} & {\tt REV16} & --- \\
{\tt rev8.w} & {\tt REV32} & --- \\
{\tt rev8}   & {\tt REV}   & {\tt BSWAP} \\
\end{tabular}
\end{center}
\caption{Comparison of bit/byte reversal instructions}
\label{grevi-cmp}
\end{table}

% \subsection{References}
%
% Hackers Delight, Chapter 7.1, ``Generalized Bit Reversal'' in
%
% https://books.google.com/books?id=iBNKMspIlqEC\&lpg=PP1\&pg=RA1-SL20-PA2\#v=onepage\&q\&f=false
%
% http://hackersdelight.org/

%%%%%%%%%%%%%%%%%%%%%%%%%%%%%%%%%%%%%%%%%%%%%%%%%%%%%%%%%%%%%%%%%%%%%%%%%%%%%%%%%%%%%%%%%%%

\subsection{Generalized Shuffle (\texttt{shfl}, \texttt{unshfl}, \texttt{shfli}, \texttt{unshfli}, \texttt{zip}, \texttt{unzip})}
\label{gzip}

\begin{rvb}
  RV32, RV64:
    shfl    rd, rs1, rs2
    unshfl  rd, rs1, rs2
    shfli   rd, rs1, imm
    unshfli rd, rs1, imm

  RV64 only:
    shflw    rd, rs1, rs2
    unshflw  rd, rs1, rs2
\end{rvb}

Shuffle is the third bit permutation instruction in the RISC-V Bitmanip
extension, after rotary shift and generalized reverse. It implements a
generalization of the operation commonly known as perfect outer shuffle and its
inverse (shuffle/unshuffle), also known as zip/unzip or interlace/uninterlace.

Bit permutations can be understood as reversible functions on bit indices (i.e.
5 bit functions on RV32 and 6 bit functions on RV64).

\begin{center}
\begin{tabular}{l l}
Operation & Corresponding function on bit indices \\
\hline
Rotate shift & Addition modulo {\rm XLEN} \\
Generalized reverse & XOR with bitmask \\
Generalized shuffle & Bitpermutation \\
\end{tabular}
\end{center}

A generalized (un)shuffle operation has $log_2(\textrm{XLEN})-1$ control bits,
one for each pair of neighbouring bits in a bit index. When the bit is set,
generalized shuffle will swap the two index bits. The {\tt shfl} operation
performs this swaps in MSB-to-LSB order (performing a rotate left shift on
contiguous regions of set control bits), and the {\tt unshfl} operation performs
the swaps in LSB-to-MSB order (performing a rotate right shift on contiguous
regions of set control bits). Combining up to $log_2(\textrm{XLEN})$ of those
{\tt shfl}/{\tt unshfl} operations can implement any bitpermutation on the
bit indices.

The most common type of shuffle/unshuffle operation is one on an immediate
control value that only contains one contiguous region of set bits. We call
those operations zip/unzip and provide pseudo-instructions for them. The naming
scheme for those pseudo-instructions is similar to the naming scheme for the
{\tt grevi} pseudo-instructions (see Tables~\ref{grevi-modes} and
\ref{grevi-names}), except that the LSB bit of the masks in Table~\ref{grevi-names}
is not used for zip/unzip.

Shuffle/unshuffle operations that only have individual bits set (not a contiguous
region of two or more bits) are their own inverse.

\begin{table}[h]
\begin{small}
\begin{center}
\begin{tabular}{r c l l}
\multicolumn{1}{c}{shamt} &
\multicolumn{1}{c}{inv} &
Bit index rotations &
Pseudo-Instruction \\

\hline

 0: 0000 & 0 & no-op                            & ---                    \\
    0000 & 1 & no-op                            & ---                    \\
 1: 0001 & 0 & {\tt i[1] -> i[0]}               & {\tt zip.n, unzip.n}   \\
    0001 & 1 & {\it equivalent to 0001 0}       & ---                    \\
 2: 0010 & 0 & {\tt i[2] -> i[1]}               & {\tt zip2.b, unzip2.b} \\
    0010 & 1 & {\it equivalent to 0010 0}       & ---                    \\
 3: 0011 & 0 & {\tt i[2] -> i[0]}               & {\tt zip.b}            \\
    0011 & 1 & {\tt i[2] <- i[0]}               & {\tt unzip.b}          \\

\hline

 4: 0100 & 0 & {\tt i[3] -> i[2]}               & {\tt zip4.h, unzip4.h} \\
    0100 & 1 & {\it equivalent to 0100 0}       & ---                    \\
 5: 0101 & 0 & {\tt i[3] -> i[2], i[1] -> i[0]} & ---                    \\
    0101 & 1 & {\it equivalent to 0101 0}       & ---                    \\
 6: 0110 & 0 & {\tt i[3] -> i[1]}               & {\tt zip2.h}           \\
    0110 & 1 & {\tt i[3] <- i[1]}               & {\tt unzip2.h}         \\
 7: 0111 & 0 & {\tt i[3] -> i[0]}               & {\tt zip.h}            \\
    0111 & 1 & {\tt i[3] <- i[0]}               & {\tt unzip.h}          \\

\hline

 8: 1000 & 0 & {\tt i[4] -> i[3]}               & {\tt zip8, unzip8}     \\
    1000 & 1 & {\it equivalent to 1000 0}       & ---                    \\
 9: 1001 & 0 & {\tt i[4] -> i[3], i[1] -> i[0]} & ---                    \\
    1001 & 1 & {\it equivalent to 1001 0}       & ---                    \\
10: 1010 & 0 & {\tt i[4] -> i[3], i[2] -> i[1]} & ---                    \\
    1010 & 1 & {\it equivalent to 1010 0}       & ---                    \\
11: 1011 & 0 & {\tt i[4] -> i[3], i[2] -> i[0]} & ---                    \\
    1011 & 1 & {\tt i[4] <- i[3], i[2] <- i[0]} & ---                    \\

\hline

12: 1100 & 0 & {\tt i[4] -> i[2]}               & {\tt zip4}             \\
    1100 & 1 & {\tt i[4] <- i[2]}               & {\tt unzip4}           \\
13: 1101 & 0 & {\tt i[4] -> i[2], i[1] -> i[0]} & ---                    \\
    1101 & 1 & {\tt i[4] <- i[2], i[1] <- i[0]} & ---                    \\
14: 1110 & 0 & {\tt i[4] -> i[1]}               & {\tt zip2}             \\
    1110 & 1 & {\tt i[4] <- i[1]}               & {\tt unzip2}           \\
15: 1111 & 0 & {\tt i[4] -> i[0]}               & {\tt zip}              \\
    1111 & 1 & {\tt i[4] <- i[0]}               & {\tt unzip}            \\

\end{tabular}
\end{center}
\end{small}
\caption{RV32 modes and pseudo-instructions for {\tt shfli}/{\tt unshfli} instruction}
\label{gzip32-modes}
\end{table}

\begin{table}[h]
\begin{small}
\begin{center}
\begin{tabular}{r c l p{1in} r c l}
\multicolumn{1}{c}{shamt} &
\multicolumn{1}{c}{inv} &
Pseudo-Instruction & &
\multicolumn{1}{c}{shamt} &
\multicolumn{1}{c}{inv} &
Pseudo-Instruction \\

\cline{1-3}
\cline{5-7}

 0: 00000 & 0 & ---                      &   &   16: 10000 & 0 & {\tt zip16, unzip16}    \\
    00000 & 1 & ---                      &   &       10000 & 1 & ---                     \\
 1: 00001 & 0 & {\tt zip.n, unzip.n}     &   &   17: 10001 & 0 & ---                     \\
    00001 & 1 & ---                      &   &       10001 & 1 & ---                     \\
 2: 00010 & 0 & {\tt zip2.b, unzip2.b}   &   &   18: 10010 & 0 & ---                     \\
    00010 & 1 & ---                      &   &       10010 & 1 & ---                     \\
 3: 00011 & 0 & {\tt zip.b}              &   &   19: 10011 & 0 & ---                     \\
    00011 & 1 & {\tt unzip.b}            &   &       10011 & 1 & ---                     \\

\cline{1-3}
\cline{5-7}

 4: 00100 & 0 & {\tt zip4.h, unzip4.h}   &   &   20: 10100 & 0 & ---                     \\
    00100 & 1 & ---                      &   &       10100 & 1 & ---                     \\
 5: 00101 & 0 & ---                      &   &   21: 10101 & 0 & ---                     \\
    00101 & 1 & ---                      &   &       10101 & 1 & ---                     \\
 6: 00110 & 0 & {\tt zip2.h}             &   &   22: 10110 & 0 & ---                     \\
    00110 & 1 & {\tt unzip2.h}           &   &       10110 & 1 & ---                     \\
 7: 00111 & 0 & {\tt zip.h}              &   &   23: 10111 & 0 & ---                     \\
    00111 & 1 & {\tt unzip.h}            &   &       10111 & 1 & ---                     \\

\cline{1-3}
\cline{5-7}

 8: 01000 & 0 & {\tt zip8.w, unzip8.w}   &   &   24: 11000 & 0 & {\tt zip8}              \\
    01000 & 1 & ---                      &   &       11000 & 1 & {\tt unzip8}            \\
 9: 01001 & 0 & ---                      &   &   25: 11001 & 0 & ---                     \\
    01001 & 1 & ---                      &   &       11001 & 1 & ---                     \\
10: 01010 & 0 & ---                      &   &   26: 11010 & 0 & ---                     \\
    01010 & 1 & ---                      &   &       11010 & 1 & ---                     \\
11: 01011 & 0 & ---                      &   &   27: 11011 & 0 & ---                     \\
    01011 & 1 & ---                      &   &       11011 & 1 & ---                     \\

\cline{1-3}
\cline{5-7}

12: 01100 & 0 & {\tt zip4.w}             &   &   28: 11100 & 0 & {\tt zip4}              \\
    01100 & 1 & {\tt unzip4.w}           &   &       11100 & 1 & {\tt unzip4}            \\
13: 01101 & 0 & ---                      &   &   29: 11101 & 0 & ---                     \\
    01101 & 1 & ---                      &   &       11101 & 1 & ---                     \\
14: 01110 & 0 & {\tt zip2.w}             &   &   30: 11110 & 0 & {\tt zip2}              \\
    01110 & 1 & {\tt unzip2.w}           &   &       11110 & 1 & {\tt unzip2}            \\
15: 01111 & 0 & {\tt zip.w}              &   &   31: 11111 & 0 & {\tt zip}               \\
    01111 & 1 & {\tt unzip.w}            &   &       11111 & 1 & {\tt unzip}             \\

\end{tabular}
\end{center}
\end{small}
\caption{RV64 modes and pseudo-instructions for {\tt shfli}/{\tt unshfli} instruction}
\label{gzip64-modes}
\end{table}

\begin{figure}[t]
\begin{center}
\input{bextcref-printperm-gzip-noflip}
\end{center}
\caption{(un)shuffle permutation network without ``flip'' stages}
\label{permnet-gzip-noflip}
\end{figure}

Like GREV and rotate shift, the (un)shuffle instruction can be implemented using a short
sequence of elementary permutations, that are enabled or disabled by the shamt
bits. But (un)shuffle has one stage fewer than GREV. Thus shfli+unshfli together require
the same amount of encoding space as grevi.

\input{bextcref-gzip32}

Or for RV64:

\input{bextcref-gzip64}

The above pattern should be intuitive to understand in order to extend
this definition in an obvious manner for RV128.

Alternatively (un)shuffle can be implemented in a single network with one more
stage than GREV, with the additional first and last stage executing a
permutation that effectively reverses the order of the inner stages. However,
since the inner stages only mux half of the bits in the word each, a hardware
implementation using this additional ``flip'' stages might actually be more
expensive than simply creating two networks.

\input{bextcref-gzip32-alt}

Figure~\ref{permnet-gzip-flip} shows the (un)shuffle permutation network with
``flip'' stages and Figure~\ref{permnet-gzip-noflip} shows the (un)shuffle
permutation network without ``flip'' stages.

\begin{figure}[t]
\begin{center}
\input{bextcref-printperm-gzip-flip}
\end{center}
\caption{(un)shuffle permutation network with ``flip'' stages}
\label{permnet-gzip-flip}
\end{figure}

The \texttt{zip} instruction with the upper half of its input cleared performs
the commonly needed ``fan-out'' operation. (Equivalent to {\tt bdep} with a
0x55555555 mask.) The \texttt{zip} instruction applied twice fans out the bits
in the lower quarter of the input word by a spacing of 4 bits.

For example, the following code calculates the bitwise prefix sum of the bits
in the lower byte of a 32 bit word on RV32:

\begin{minipage}{\linewidth}
\begin{verbatim}
  andi a0, a0, 0xff
  zip a0, a0
  zip a0, a0
  slli a1, a0, 4
  c.add a0, a1
  slli a1, a0, 8
  c.add a0, a1
  slli a1, a0, 16
  c.add a0, a1
\end{verbatim}
\end{minipage}

The final prefix sum is stored in the 8 nibbles of the {\tt a0} output word.

Similarly, the following code stores the indices of the set bits in the LSB
nibbles of the output word (with the LSB bit having index 1), with the unused
MSB nibbles in the output set to zero:

\begin{minipage}{\linewidth}
\begin{verbatim}
  andi a0, a0, 0xff
  zip a0, a0
  zip a0, a0
  orc.n a0, a0
  li a1, 0x87654321
  and a1, a0, a1
  bext a0, a1, a0
\end{verbatim}
\end{minipage}

Other {\tt zip} modes can be used to ``fan-out'' in blocks of 2, 4, 8, or 16 bit.
{\tt zip} can be combined with {\tt grevi} to perform inner shuffles. For example
on RV64:

\begin{minipage}{\linewidth}
\begin{verbatim}
  li a0, 0x0000000012345678
  zip4 t0, a0    ; <- 0x0102030405060708
  rev4.b t1, t0  ; <- 0x1020304050607080
  zip8 t2, a0    ; <- 0x0012003400560078
  rev8.h t3, t2  ; <- 0x1200340056007800
  zip16 t4, a0   ; <- 0x0000123400005678
  rev16.w t5, t4 ; <- 0x1234000056780000
\end{verbatim}
\end{minipage}

Another application for the zip instruction is generating Morton
code~\cite{MortonCode}.

The x86 {\tt PUNPCK[LH]*} MMX/SSE/AVX instructions perform similar operations
as {\tt zip8} and {\tt zip16}.

%%%%%%%%%%%%%%%%%%%%%%%%%%%%%%%%%%%%%%%%%%%%%%%%%%%%%%%%%%%%%%%%%%%%%%%%%%%%%%%%%%%%%%%%%%%

\subsection{Crossbar Permutation Instructions (\texttt{xperm.[nbhw]})}

\begin{rvb}
  RV32, RV64:
    xperm.n rd, rs1, rs2
    xperm.b rd, rs1, rs2
    xperm.h rd, rs1, rs2

  RV64 only:
    xperm.w rd, rs1, rs2
\end{rvb}

These instructions operate on nibbles/bytes/half-words/words. {\tt rs1} is
a vector of data words and {\tt rs2} is a vector of indices into {\tt rs1}.
The result of the instruction is the vector {\tt rs2} with each element replaced
by the corresponding data word from {\tt rs1}, or zero then the index in {\tt rs2}
is out of bounds.

\input{bextcref-xperm}

The \texttt{xperm.[nbhw]} instructions can be implemented with an $XLEN/4$-lane
nibble-wide crossbar switch.

The \texttt{xperm.n} instruction can be used to implement an arbitrary 64-bit
bit permutation in 15 instructions, using 8 control words and 3 constant masks~\cite{Wolf20A}:

\begin{minipage}{\linewidth}
\begin{verbatim}
  uint64_t perm64_bitmanip_cmix(perm64t &p, uint64_t v)
  {
      uint64_t v0 = _rv64_gorc(_rv64_xperm_n(v, p.ctrl[0]) & p.mask[0], 3);   //  3 insns
      uint64_t v1 = _rv64_gorc(_rv64_xperm_n(v, p.ctrl[1]) & p.mask[1], 3);   //  6 insns
      uint64_t v2 = _rv64_gorc(_rv64_xperm_n(v, p.ctrl[2]) & p.mask[2], 3);   //  9 insns
      uint64_t v3 = _rv64_gorc(_rv64_xperm_n(v, p.ctrl[3]) & p.mask[3], 3);   // 12 insns

      v0 = _rv_cmix(0x1111111111111111LL, v0, v1);                            // 13 insns
      v2 = _rv_cmix(0x4444444444444444LL, v2, v3);                            // 14 insns
      return _rv_cmix(0x3333333333333333LL, v0, v2);                          // 15 insns
  }
\end{verbatim}
\end{minipage}

%%%%%%%%%%%%%%%%%%%%%%%%%%%%%%%%%%%%%%%%%%%%%%%%%%%%%%%%%%%%%%%%%%%%%%%%%%%%%%%%%%%%%%%%%%%

\begin{table}[h]
\begin{small}
\begin{center}
\begin{tabular}{r l p{0.5in} r l p{0.3in} r l}

\multicolumn{2}{c}{RV32} & &
\multicolumn{5}{c}{RV64} \\

\cline{1-2}
\cline{4-8}

\multicolumn{1}{c}{shamt} & Instruction & &
\multicolumn{1}{c}{shamt} & Instruction & &
\multicolumn{1}{c}{shamt} & Instruction \\

\cline{1-2}
\cline{4-5}
\cline{7-8}

 0: 00000 & ---           &   &  0: 000000 & ---           &   & 32: 100000 & {\tt orc32} \\
 1: 00001 & {\tt orc.p}   &   &  1: 000001 & {\tt orc.p}   &   & 33: 100001 & ---         \\
 2: 00010 & {\tt orc2.n}  &   &  2: 000010 & {\tt orc2.n}  &   & 34: 100010 & ---         \\
 3: 00011 & {\tt orc.n}   &   &  3: 000011 & {\tt orc.n}   &   & 35: 100011 & ---         \\
 4: 00100 & {\tt orc4.b}  &   &  4: 000100 & {\tt orc4.b}  &   & 36: 100100 & ---         \\
 5: 00101 & ---           &   &  5: 000101 & ---           &   & 37: 100101 & ---         \\
 6: 00110 & {\tt orc2.b}  &   &  6: 000110 & {\tt orc2.b}  &   & 38: 100110 & ---         \\
 7: 00111 & {\tt orc.b}   &   &  7: 000111 & {\tt orc.b}   &   & 39: 100111 & ---         \\

\cline{1-2}
\cline{4-5}
\cline{7-8}

 8: 01000 & {\tt orc8.h}  &   &  8: 001000 & {\tt orc8.h}  &   & 40: 101000 & ---         \\
 9: 01001 & ---           &   &  9: 001001 & ---           &   & 41: 101001 & ---         \\
10: 01010 & ---           &   & 10: 001010 & ---           &   & 42: 101010 & ---         \\
11: 01011 & ---           &   & 11: 001011 & ---           &   & 43: 101011 & ---         \\
12: 01100 & {\tt orc4.h}  &   & 12: 001100 & {\tt orc4.h}  &   & 44: 101100 & ---         \\
13: 01101 & ---           &   & 13: 001101 & ---           &   & 45: 101101 & ---         \\
14: 01110 & {\tt orc2.h}  &   & 14: 001110 & {\tt orc2.h}  &   & 46: 101110 & ---         \\
15: 01111 & {\tt orc.h}   &   & 15: 001111 & {\tt orc.h}   &   & 47: 101111 & ---         \\

\cline{1-2}
\cline{4-5}
\cline{7-8}

16: 10000 & {\tt orc16}   &   & 16: 010000 & {\tt orc16.w} &   & 48: 110000 & {\tt orc16} \\
17: 10001 & ---           &   & 17: 010001 & ---           &   & 49: 110001 & ---         \\
18: 10010 & ---           &   & 18: 010010 & ---           &   & 50: 110010 & ---         \\
19: 10011 & ---           &   & 19: 010011 & ---           &   & 51: 110011 & ---         \\
20: 10100 & ---           &   & 20: 010100 & ---           &   & 52: 110100 & ---         \\
21: 10101 & ---           &   & 21: 010101 & ---           &   & 53: 110101 & ---         \\
22: 10110 & ---           &   & 22: 010110 & ---           &   & 54: 110110 & ---         \\
23: 10111 & ---           &   & 23: 010111 & ---           &   & 55: 110111 & ---         \\

\cline{1-2}
\cline{4-5}
\cline{7-8}

24: 11000 & {\tt orc8}    &   & 24: 011000 & {\tt orc8.w}  &   & 56: 111000 & {\tt orc8}  \\
25: 11001 & ---           &   & 25: 011001 & ---           &   & 57: 111001 & ---         \\
26: 11010 & ---           &   & 26: 011010 & ---           &   & 58: 111010 & ---         \\
27: 11011 & ---           &   & 27: 011011 & ---           &   & 59: 111011 & ---         \\
28: 11100 & {\tt orc4}    &   & 28: 011100 & {\tt orc4.w}  &   & 60: 111100 & {\tt orc4}  \\
29: 11101 & ---           &   & 29: 011101 & ---           &   & 61: 111101 & ---         \\
30: 11110 & {\tt orc2}    &   & 30: 011110 & {\tt orc2.w}  &   & 62: 111110 & {\tt orc2}  \\
31: 11111 & {\tt orc}     &   & 31: 011111 & {\tt orc.w}   &   & 63: 111111 & {\tt orc}   \\
\end{tabular}
\end{center}
\end{small}
\caption{Pseudo-instructions for {\tt gorci} instruction}
\label{gorci-modes}
\end{table}

\section{Generalized OR-Combine (\texttt{gorc, gorci})}

\begin{rvb}
  RV32, RV64:
    gorc  rd, rs1, rs2
    gorci rd, rs1, imm

  RV64 only:
    gorcw  rd, rs1, rs2
    gorciw rd, rs1, imm
\end{rvb}

The GORC operation is similar to GREV, except that instead of swapping pairs of bits,
GORC ORs them together, and writes the new value in both positions.

\input{bextcref-gorc}

GORC can be useful for copying naturally-aligned fields in a word, and testing such
fields for being equal to zero.

{\tt gorci} pseudo-instructions follow the same naming scheme as {\tt grevi}
pseudo-instructions (see Tables~\ref{grevi-modes} and \ref{grevi-names}),
except the prefix {\tt orc} is used instead of {\tt rev}. See
Table~\ref{gorci-modes} for a full list of {\tt gorci} pseudo-instructions.

An important use-case is {\tt strlen()} and {\tt strcpy()}, which can utilize
{\tt orc.b} for testing for zero bytes, and counting trailing non-zero bytes
in a word.

%%%%%%%%%%%%%%%%%%%%%%%%%%%%%%%%%%%%%%%%%%%%%%%%%%%%%%%%%%%%%%%%%%%%%%%%%%%%%%%%%%%%%%%%%%%

\section{Bit-Field Place (\texttt{bfp})}

\begin{rvb}
  RV32, RV64:
    bfp rd, rs1, rs2

  RV64 only:
    bfpw rd, rs1, rs2
\end{rvb}

The bit field place ({\tt bfp}) instruction places up to $\mathrm{XLEN}/2$ LSB bits from {\tt rs2} into
the value in {\tt rs1}. The upper bits of {\tt rs2} control the length of the bit field
and target position. The layout of {\tt rs2} is chosen in a way that makes it possible to construct
{\tt rs2} easily using {\tt pack[h]} instructions and/or {\tt andi}/{\tt lui}.

\input{bextcref-bfp}

The layout of the control word in rs2 is as follows for RV32. {\tt LEN=0} encodes for {\tt LEN=16}.

\begin{minipage}{\linewidth}
\begin{verbatim}
    |  3                   2                   1                    |
    |1 0 9 8 7 6 5 4 3 2 1 0 9 8 7 6 5 4 3 2 1 0 9 8 7 6 5 4 3 2 1 0|
    |---------------|---------------|---------------|---------------|
    |       |  LEN  |     |   OFF   |             DATA              |
    |---------------|---------------|---------------|---------------|
\end{verbatim}
\end{minipage}

And on RV64 (with {\tt LEN=0} encoding for {\tt LEN=32}):

\begin{minipage}{\linewidth}
\begin{verbatim}
    |      6                   5                   4                   3             |
    |3 2 1 0 9 8 7 6 5 4 3 2 1 0 9 8 7 6 5 4 3 2 1 0 9 8 7 6 5 4 3 2 1 0 9 .... 2 1 0|
    |---------------|---------------|---------------|---------------|------ -- ------|
    |SEL| |   LEN   |   |    OFF    |     |   LEN'  |   |    OFF'   |      DATA      |
    |---------------|---------------|---------------|---------------|------ -- ------|
\end{verbatim}
\end{minipage}

When {\tt SEL=10} then {\tt LEN} and {\tt OFF} are used, otherwise {\tt LEN'} and {\tt OFF'} are used.

Placing bits from {\tt a0} in {\tt a1}, with results in {\tt t0} on RV32:

\begin{minipage}{\linewidth}
\begin{verbatim}
  addi t0, zero, {length[3:0], offset[7:0]}
  pack t0, a0, t0
  bfp t0, a1, t0
\end{verbatim}
\end{minipage}

And on RV64:

\begin{minipage}{\linewidth}
\begin{verbatim}
  lui t0, zero, {3'b 100, length[4:0], offset[7:0], 4'b 0000}
  pack t0, a0, t0
  bfp t0, a1, t0
\end{verbatim}
\end{minipage}

Or with {\tt a2=length} and {\tt a3=offset}:

\begin{minipage}{\linewidth}
\begin{verbatim}
  packh t0, a3, a2
  pack t0, a0, t0
  bfp t0, a1, t0
\end{verbatim}
\end{minipage}

Placing up to 16 constant bits in any contiguous region:

\begin{minipage}{\linewidth}
\begin{verbatim}
  lui t0, ...
  addi t0, t0, ...
  bfp[w] t0, a1, t0
\end{verbatim}
\end{minipage}

Note that above sequences only modify one register ({\tt t0}), which makes them
fuse-able sequences.

%%%%%%%%%%%%%%%%%%%%%%%%%%%%%%%%%%%%%%%%%%%%%%%%%%%%%%%%%%%%%%%%%%%%%%%%%%%%%%%%%%%%%%%%%%%

\section{Bit Extract/Deposit (\texttt{bext,\ bdep})}

\begin{rvb}
  RV32, RV64:
    bext rd, rs1, rs2
    bdep rd, rs1, rs2

  RV64 only:
    bextw rd, rs1, rs2
    bdepw rd, rs1, rs2
\end{rvb}

This instructions implement the generic bit extract and bit deposit functions.
This operation is also referred to as bit gather/scatter, bit pack/unpack,
parallel extract/deposit, compress/expand, or right\_compress/right\_expand.

\texttt{bext} collects LSB justified bits to rd from rs1 using extract mask in rs2.

\texttt{bdep} writes LSB justified bits from rs1 to rd using deposit mask in rs2.

\input{bextcref-bext}

Implementations may choose to use smaller multi-cycle implementations of
\texttt{bext} and \texttt{bdep}, or even emulate the instructions in software.

Even though multi-cycle \texttt{bext} and \texttt{bdep} often are not fast
enough to outperform algorithms that use sequences of shifts and bit masks,
dedicated instructions for those operations can still be of great advantage in
cases where the mask argument is not constant.

For example, the following code efficiently calculates the index of the tenth
set bit in {\tt a0} using \texttt{bdep}:

\begin{minipage}{\linewidth}
\begin{verbatim}
  li a1, 0x00000200
  bdep a0, a1, a0
  ctz a0, a0
\end{verbatim}
\end{minipage}

For cases with a constant mask an optimizing compiler would decide when to use
\texttt{bext} or \texttt{bdep} based on the optimization profile for the
concrete processor it is optimizing for. This is similar to the decision
whether to use MUL or DIV with a constant, or to perform the same operation
using a longer sequence of much simpler operations.

The \texttt{bext} and \texttt{bdep} instructions are equivalent to the x86 BMI2
instructions PEXT and PDEP. But there is much older prior art. For example, the
soviet BESM-6 mainframe computer, designed and built in the 1960s, had APX/AUX
instructions with almost the same semantics.~\cite{BESM6} (The BESM-6 APX/AUX
instructions packed/unpacked at the MSB end instead of the LSB end. Otherwise
it is the same instruction.)

Efficient hardware implementations of \texttt{bext} and \texttt{bdep} are described
in~\cite{Hilewitz06} and demonstrated in~\cite{Wolf17B}.

% \subsection{Justification}
%
% http://svn.clifford.at/handicraft/2017/permsyn/
%
% \subsection{References}
%
% http://programming.sirrida.de/bit\_perm.html\#gather\_scatter
%
% Hackers Delight, Chapter 7.1, ``Compress, Generalized Extract'' in
%
% https://books.google.com/books?id=iBNKMspIlqEC\&lpg=PP1\&pg=RA1-SL20-PA2\#v=onepage\&q\&f=false
%
% http://hackersdelight.org/
%
% https://github.com/cliffordwolf/bextdep
%
% http://palms.ee.princeton.edu/system/files/Hilewitz_JSPS_08.pdf

%%%%%%%%%%%%%%%%%%%%%%%%%%%%%%%%%%%%%%%%%%%%%%%%%%%%%%%%%%%%%%%%%%%%%%%%%%%%%%%%%%%%%%%%%%%

\section{Carry-Less Multiply (\texttt{clmul, clmulh, clmulr})}

\begin{rvb}
  RV32, RV64:
    clmul  rd, rs1, rs2
    clmulh rd, rs1, rs2
    clmulr rd, rs1, rs2
\end{rvb}

Calculate the carry-less product~\cite{CarryLessProduct} of the two arguments. \texttt{clmul}
produces the lower half of the carry-less product and \texttt{clmulh} produces the upper half
of the 2$\cdot$XLEN carry-less product.

Carry-less multiplication is equivalent to multiplication in the polynomial ring over GF(2).

\texttt{clmulr} produces bits 2$\cdot$XLEN$-2$:XLEN-1 of the 2$\cdot$XLEN carry-less product.
That means \texttt{clmulh} is equivalent to \texttt{clmulr} followed by a 1-bit right shift.
(The MSB of a \texttt{clmulh} result is always zero.) Another equivalent definition of
\texttt{clmulr} is \texttt{clmulr(a,b) := rev(clmul(rev(a), rev(b)))}. (The ``r''
in \texttt{clmulr} means reversed.)

Unlike {\tt mulh[[s]u]}, we add a *W variant of {\tt clmulh}. This is because we expect
some code to use 32-bit clmul intrisics, even on 64-bit architectures. For example in cases
where data is processed in 32-bit chunks.

\input{bextcref-clmul}

The classic applications for \texttt{clmul} are Cyclic Redundancy Check (CRC)~\cite{FastCRC,Wolf18A}
and Galois/Counter Mode (GCM), but more applications exist, including the following examples.

There are obvious applications in hashing and pseudo random number generations. For
example, it has been reported that hashes based on carry-less multiplications can
outperform Google's CityHash~\cite{CLHASH}.

\texttt{clmul} of a number with itself inserts zeroes between each input bit. This can
be useful for generating Morton code~\cite{MortonCode}.

\texttt{clmul} of a number with -1 calculates the prefix XOR operation. This can
be useful for decoding gray codes.

Another application of XOR prefix sums calculated with \texttt{clmul} is
branchless tracking of quoted strings in high-performance parsers.~\cite{ParseJSON}

Carry-less multiply can also be used to implement Erasure code efficiently.~\cite{ClmulErasureCode}

SPARC introduced similar instructions (XMULX, XMULXHI) in SPARC T3 in 2010.~\cite{sparct3}

TI C6000 introduced a similar instruction (XORMPY) in C64x+.~\cite{c64xp}

%%%%%%%%%%%%%%%%%%%%%%%%%%%%%%%%%%%%%%%%%%%%%%%%%%%%%%%%%%%%%%%%%%%%%%%%%%%%%%%%%%%%%%%%%%%

\section{CRC Instructions (\texttt{crc32.[bhwd], crc32c.[bhwd]})}

\begin{rvb}
  RV32, RV64:
    crc32.b rd, rs
    crc32.h rd, rs
    crc32.w rd, rs
    crc32c.b rd, rs
    crc32c.h rd, rs
    crc32c.w rd, rs

  RV64 only:
    crc32.d rd, rs
    crc32c.d rd, rs
\end{rvb}

Unary Cyclic Redundancy Check (CRC) instructions that interpret the bits of rs1
as a CRC32/CRC32C state and perform a polynomial reduction of that state
shifted left by 8, 16, 32, or 64 bits.

The instructions return the new CRC32/CRC32C state.

The \texttt{crc32.w}/\texttt{crc32c.w} instructions are equivalent to executing
\texttt{crc32.h}/\texttt{crc32c.h} twice, and \texttt{crc32.h}/\texttt{crc32c.h}
instructions are equivalent to executing \texttt{crc32.b}/\texttt{crc32c.b}
twice.

All 8 CRC instructions operate on bit-reflected data.

\input{bextcref-crc}

Payload data must be XOR'ed into the LSB end of the state before executing the
CRC instruction. The following code demonstrates the use of \texttt{crc32.b}:

\begin{minipage}{\linewidth}
\begin{verbatim}
  uint32_t crc32_demo(const uint8_t *p, int len)
  {
    uint32_t x = 0xffffffff;
    for (int i = 0; i < len; i++) {
      x = x ^ p[i];
      x = crc32_b(x);
    }
    return ~x;
  }
\end{verbatim}
\end{minipage}

In terms of binary polynomial arithmetic those instructions perform the operation
$$ \texttt{rd}'(x) = (\texttt{rs1}'(x) \cdot x^N) \; \textrm{mod} \; \{\texttt{1}, P'\}(x)\textrm, $$
with $N \in \{8, 16, 32, 64\}$,
$P = \texttt{0xEDB8\_8320}$ for CRC32 and $P = \texttt{0x82F6\_3B78}$ for CRC32C,
$a'$ denoting the XLEN bit reversal of $a$,
and $\{a, b\}$ denoting bit concatenation.
Note that for example for CRC32 $\{\texttt{1}, P'\} = \texttt{0x1\_04C1\_1DB7}$
on RV32 and $\{\texttt{1}, P'\} = \texttt{0x1\_04C1\_1DB7\_0000\_0000}$ on RV64.

These dedicated CRC instructions are meant for RISC-V implementations without fast multiplier
and therefore without fast \texttt{clmul[h]}. For implementations with fast \texttt{clmul[h]}
it is recommended to use the methods described in~\cite{FastCRC} and demonstrated in~\cite{Wolf18A}
that can process XLEN input bits using just one carry-less multiply for arbitrary CRC polynomials.

In applications where those methods are not applicable it is possible to emulate the dedicated CRC
instructions using two carry-less multiplies that implement a Barrett reduction. The following example
implements a replacement for \texttt{crc32.w} (RV32).

\begin{minipage}{\linewidth}
\begin{verbatim}
crc32_w:
  li t0, 0xF7011641
  li t1, 0xEDB88320
  clmul a0, a0, t0
  clmulr a0, a0, t1
  ret
\end{verbatim}
\end{minipage}

%%%%%%%%%%%%%%%%%%%%%%%%%%%%%%%%%%%%%%%%%%%%%%%%%%%%%%%%%%%%%%%%%%%%%%%%%%%%%%%%%%%%%%%%%%%

\section{Bit-Matrix Instructions (\texttt{bmatxor, bmator, bmatflip}, RV64 only)}

\begin{rvb}
  RV64 only:
    bmator rd, rs1, rs2
    bmatxor rd, rs1, rs2
    bmatflip rd, rs
\end{rvb}

These are 64-bit-only instruction that are not available on RV32. On RV128 they
ignore the upper half of operands and sign extend the results.

This instructions interpret a 64-bit value as 8x8 binary matrix.

\texttt{bmatxor} performs a matrix-matrix multiply with boolean AND as multiply
operator and boolean XOR as addition operator.

\texttt{bmator} performs a matrix-matrix multiply with boolean AND as multiply
operator and boolean OR as addition operator.

\texttt{bmatflip} is a unary operator that transposes the source matrix. It is
equivalent to \texttt{zip; zip; zip} on RV64.

\input{bextcref-bmat}

Among other things, \texttt{bmatxor}/\texttt{bmator} can be used to perform
arbitrary permutations of bits within each byte (permutation matrix as 2nd
operand) or perform arbitrary permutations of bytes within a 64-bit word
(permutation matrix as 1st operand).

There are similar instructions in Cray XMT~\cite{CrayXMT}. The Cray X1
architecture even has a full 64x64 bit matrix multiply unit~\cite{CrayX1}.
(See Section~\ref{bmat64} for how to implement 64x64 bit matix operations
with {\tt bmat[x]or}.)

The MMIX architecture has MOR and MXOR instructions with the same semantic.~\cite[p.~182f]{Knuth4A}

The x86 EVEX/VEX/SSE instruction GF2P8AFFINEQB is equivalent to {\tt bmatxor}.

The {\tt bmm.8} instruction proposed in \cite{Hilewitz08} is also equivalent to {\tt bmatxor}.

%%%%%%%%%%%%%%%%%%%%%%%%%%%%%%%%%%%%%%%%%%%%%%%%%%%%%%%%%%%%%%%%%%%%%%%%%%%%%%%%%%%%%%%%%%%

\section{Ternary Bit-Manipulation Instructions}

%%%%%%%%%%%%%%%%%%%%%%%%%%%%%%%%%%%%%%%%%%%%%%%%%%%%%%%%%%%%%%%%%%%%%%%%%%%%%%%%%%%%%%%%%%%

\subsection{Conditional Mix ({\tt cmix})}

\begin{rvb}
  RV32, RV64:
    cmix rd, rs2, rs1, rs3
\end{rvb}

(Note that the assembler syntax of {\tt cmix} has the {\tt rs2} argument first
to make assembler code more readable. But the reference C code code below uses
the ``architecturally correct'' argument order {\tt rs1, rs2, rs3}.)

The {\tt cmix rd, rs2, rs1, rs3} instruction selects bits from {\tt rs1} and {\tt rs3} based
on the bits in the control word {\tt rs2}.

\input{bextcref-cmix}

It replaces sequences like the following.

\begin{minipage}{\linewidth}
\begin{verbatim}
  and rd, rs1, rs2
  andn t0, rs3, rs2
  or rd, rd, t0
\end{verbatim}
\end{minipage}

Using {\tt cmix} a single butterfly stage can be implemented in only two
instructions. Thus, arbitrary bit-permutations can be implemented using only
18 instruction (32 bit) or 22 instructions (64 bits).

\subsection{Conditional Move ({\tt cmov})}

\begin{rvb}
  RV32, RV64:
    cmov rd, rs2, rs1, rs3
\end{rvb}

(Note that the assembler syntax of {\tt cmov} has the {\tt rs2} argument first
to make assembler code more readable. But the reference C code code below uses
the ``architecturally correct'' argument order {\tt rs1, rs2, rs3}.)

The {\tt cmov rd, rs2, rs1, rs3} instruction selects {\tt rs1} if the control
word {\tt rs2} is non-zero, and {\tt rs3} if the control word is zero.

\input{bextcref-cmov}

The {\tt cmov} instruction helps avoiding branches, which can lead to better
performance, and helps with constant-time code as used in some cryptography
applications.

\subsection{Funnel Shift ({\tt fsl}, {\tt fsr}, {\tt fsri})}

\begin{rvb}
  RV32, RV64:
    fsl  rd, rs1, rs3, rs2
    fsr  rd, rs1, rs3, rs2
    fsri rd, rs1, rs3, imm

  RV64 only:
    fslw  rd, rs1, rs3, rs2
    fsrw  rd, rs1, rs3, rs2
    fsriw rd, rs1, rs3, imm
\end{rvb}

(Note that the assembler syntax for funnel shifts has the {\tt rs2} argument
last to make assembler code more readable. But the reference C code code below
uses the ``architecturally correct'' argument order {\tt rs1, rs2, rs3}.)

The {\tt fsl rd, rs1, rs3, rs2} instruction creates a $2\cdot\textrm{XLEN}$ word
by concatenating rs1 and rs3 (with rs1 in the MSB half), rotate-left-shifts that
word by the amount indicated in the $log_2(\textrm{XLEN})+1$ LSB bits in rs2, and
then writes the MSB half of the result to rd.

The {\tt fsr rd, rs1, rs3, rs2} instruction creates a $2\cdot\textrm{XLEN}$ word
by concatenating rs1 and rs3 (with rs1 in the LSB half), rotate-right-shifts that
word by the amount indicated in the $log_2(\textrm{XLEN})+1$ LSB bits in rs2, and
then writes the LSB half of the result to rd.

\input{bextcref-fsl}

\input{bextcref-fsr}

A shift unit capable of either {\tt fsl} or {\tt fsr} is capable of performing all
the other shift functions, including the other funnel shift, with only minimal additional
logic.

For any values of {\tt A}, {\tt B}, and {\tt C}:

\begin{minipage}{\linewidth}
\begin{verbatim}
  fsl(A, B, C) = fsr(A, -B, C)
\end{verbatim}
\end{minipage}

And for any values {\tt x} and $0 \le \texttt{shamt} < \texttt{XLEN}$:

\begin{minipage}{\linewidth}
\begin{verbatim}
  sll(x, shamt) == fsl(x, shamt, 0)
  srl(x, shamt) == fsr(x, shamt, 0)
  sra(x, shamt) == fsr(x, shamt, sext_x)
  slo(x, shamt) == fsl(x, shamt, ~0)
  sro(x, shamt) == fsr(x, shamt, ~0)
  ror(x, shamt) == fsr(x, shamt, x)
  rol(x, shamt) == fsl(x, shamt, x)
\end{verbatim}
\end{minipage}

Furthermore an RV64 implementation of either {\tt fsl} or {\tt fsr} is capable
of performing the *W versions of all shift operations with only a few gates
of additional control logic.

On RV128 there is no {\tt fsri} instruction. But there is {\tt fsriw} and {\tt fsrid}.

%%%%%%%%%%%%%%%%%%%%%%%%%%%%%%%%%%%%%%%%%%%%%%%%%%%%%%%%%%%%%%%%%%%%%%%%%%%%%%%%%%%%%%%%%%%

\section{Address calculation instructions}

\begin{rvb}
  RV32, RV64:
    sh1add rd, rs1, rs2
    sh2add rd, rs1, rs2
    sh3add rd, rs1, rs2

  RV64 only:
    sh1addu.w rd, rs1, rs2
    sh2addu.w rd, rs1, rs2
    sh3addu.w rd, rs1, rs2
\end{rvb}

These instructions shift {\tt rs1} left by 1, 2, or 3 bits, then add the result
to {\tt rs2}. The {\tt sh?addu.w} instructions are identical to {\tt sh?add}, except
that bits XLEN-1:32 of the {\tt rs1} argument are cleared before the shift.

\input{bextcref-shadd}

An opcode for {\tt sh4add}/{\tt sh4addu.w} for RV128 and/or RVQ is reserved.

%%%%%%%%%%%%%%%%%%%%%%%%%%%%%%%%%%%%%%%%%%%%%%%%%%%%%%%%%%%%%%%%%%%%%%%%%%%%%%%%%%%%%%%%%%%

\section{Add/shift with prefix zero-extend ({\tt addu.w}, {\tt slliu.w})}

\begin{rvb}
  RV64:
    addu.w rd, rs1, rs2
    slliu.w rd, rs1, imm
\end{rvb}

{\tt slliu.w} and {\tt addu.w} are identical to {\tt slli} and {\tt add}, respectively,
except that bits XLEN-1:32 of the {\tt rs1} argument are cleared before the shift or add.

\input{bextcref-slliuw}
\input{bextcref-adduw}

%%%%%%%%%%%%%%%%%%%%%%%%%%%%%%%%%%%%%%%%%%%%%%%%%%%%%%%%%%%%%%%%%%%%%%%%%%%%%%%%%%%%%%%%%%%

\section{Opcode Encodings}
\label{opcodes}

This chapter contains proposed encodings for most of the instructions described
in this document. {\bf DO NOT IMPLEMENT THESE OPCODES YET.} We are trying to get
official opcodes assigned and will update this chapter soon with the official
opcodes.

The {\tt andn}, {\tt orn}, and {\tt xnor} instruction are encoded the same way
as {\tt and}, {\tt or}, and {\tt xor}, but with {\tt op[30]} set, mirroring the
encoding scheme used for {\tt add} and {\tt sub}.

All shift instructions use {\tt funct3=001} for left shifts and {\tt funct3=101}
for right shifts. Just like in the RISC-V integer base ISA, the shift-immediate
instructions have a 5 bit immediate on RV32, and a 6 bit immediate on RV64, and we
reserve encoding space for a 7 bit immediate for RV128.  The same sizes apply
to {\tt sbseti}, {\tt sbclri}, {\tt sbinvi}, and {\tt sbexti}.

The immediate for {\tt shfli}/{\tt unshufli} is one bit smaller than the immediate
for shift instructions, that is 4 bits on RV32, 5 bits on RV64, and we reserve 6
bits for RV128.

{\tt op[26]=1} selects funnel shifts. For funnel shifts {\tt op[30:29]} is part
if the 3rd operand and therefore unused for encoding the operation. For all other
shift operations {\tt op[26]=0}.

{\tt fsri} is also encoded with {\tt op[26]=1}, leaving a 6 bit immediate. The 7th
bit, that is necessary to perform a 128 bit funnel shift on RV64, can be
emulated by swapping rs1 and rs3.

There is no {\tt shfliw} instruction. The {\tt slliu.w} instruction occupies
the encoding slot that would be occupied by {\tt shfliw}.

On RV128 {\tt op[26]} contains the MSB of the immediate for the shift instructions.
Therefore there is no FSRI instruction on RV128. (But there is FSRIW/FSRID.)

\begin{minipage}{\linewidth}
\begin{verbatim}
         | SLL  SRL  SRA | SLO SRO | ROL ROR | FSL FSR
  op[30] |   0    0    1 |   0   0 |   1   1 |   -   -
  op[29] |   0    0    0 |   1   1 |   1   1 |   -   -
  op[26] |   0    0    0 |   0   0 |   0   0 |   1   1
  funct3 | 001  101  101 | 001 101 | 001 101 | 001 101
\end{verbatim}
\end{minipage}

Only an encoding for RORI exists, as ROLI can be implemented with RORI by negating
the immediate. Unary functions are encoded in the spot that would correspond to ROLI,
with the function encoded in the 5 LSB bits of the immediate.

The CRC instructions are encoded as unary instructions with {\tt op[24]} set. The
polynomial is selected via {\tt op[23]}, with {\tt op[23]=0} for CRC32 and
{\tt op[23]=1} for CRC32C. The width is selected with {\tt op[22:20]}, using
the same encoding as is used in {\tt funct3} for load/store operations.

{\tt cmix} and {\tt cmov} are encoded using the two remaining ternary operator
encodings in {\tt funct3=001} and {\tt funct3=101}. (There are two ternary
operator encodings per minor opcode using the {\tt op[26]=1} scheme for
marking ternary OPs.)

The single-bit instructions are also encoded within the shift opcodes, with
{\tt op[27]} set, and using {\tt op[30]} and {\tt op[29]} to select the operation:

\begin{minipage}{\linewidth}
\begin{verbatim}
         | SBCLR  SBSET  SBINV | SBEXT   GORC   GREV
  op[30] |     1      0      1 |     1      0      1
  op[29] |     0      1      1 |     0      1      1
  op[27] |     1      1      1 |     1      1      1
  funct3 |   001    001    001 |   101    101    101
\end{verbatim}
\end{minipage}

There is no {\tt sbextiw} instruction as it can be emulated trivially using
{\tt sbexti}. However, there is {\tt sbsetiw}, {\tt sbclriw}, and {\tt sbinviw}
as changing bit 31 would change the sign extend. There are non-immediate *W
instructions of all single-bit instructions, including {\tt sbextw}, because
the number of used bits in rs2 is different in {\tt sbext} and {\tt sbextw}.

GORC and GREV are encoded in the two remaining slots in the single-bit
instruction encoding space.

The remaining instructions are encoded within {\tt funct7=0000100} and
{\tt funct7=0000101}.

The {\tt funct7=0000101} block contains {\tt clmul[hr]},
{\tt min[u]}, and {\tt max[u]}.

The encoding of {\tt clmul, clmulr, clmulh} is identical to the encoding of
{\tt mulh, mulhsu, mulhu}, except that {\tt op[27]=1}.

The encoding of {\tt min[u]}/{\tt max[u]} uses {\tt funct3=100..111}. The
{\tt funct3} encoding matches {\tt op[31:29]} of the AMO min/max functions.

The remaining instructions are encoded within {\tt funct7=0000100}. The
shift-like {\tt shfl}/{\tt unshfl} instructions uses the same {\tt funct3}
values as the shift operations. {\tt bdep} and {\tt bext} are encoded in a
way so that {\tt funct3[2]} selects the ``direction'', similar to shift
operations.

{\tt bmat[x]or} use {\tt funct3=011} and {\tt funct3=111} in {\tt funct7=0000100}.

{\tt pack} occupies {\tt funct3=100} in {\tt funct7=0000100}.

{\tt addu.w} is encoded like {\tt addw}, except that {\tt op[27]=1}.

Finally, RV64 has {\tt *W} instructions for all bitmanip instructions, with the
following exceptions:

{\tt andn}, {\tt cmix}, {\tt cmov}, {\tt min[u]}, {\tt max[u]} have no {\tt *W}
variants because they already behave in the way a {\tt *W} instruction would
when presented with sign-exteded 32-bit arguments.

{\tt bmatflip}, {\tt bmatxor}, {\tt bmator} have no {\tt *W} variants because
they are 64-bit only instructions.

{\tt crc32.[bhwd]}, {\tt crc32c.[bhwd]} have no {\tt *W} variants because {\tt
crc32[c].w} is deemed sufficient.

There is no {\tt [un]shfliw}, as a perfect outer shuffle always preserves the
MSB bit, thus {\tt [un]shfli} preserves proper sign extension when the
upper bit in the control word is set. There's still {\tt [un]shflw} that
masks that upper control bit and sign-extends the output.

Relevant instruction encodings from the base ISA are included in the table below
and are marked with a {\tt *}.

% Opcodes:
% 0010011 OP-IMM
% 0110011 OP
% 0011011 OP-IMM-32
% 0111011 OP-32

\begin{minipage}{\linewidth}
\begin{verbatim}
|  3                   2                   1                    |
|1 0 9 8 7 6 5 4 3 2 1 0 9 8 7 6 5 4 3 2 1 0 9 8 7 6 5 4 3 2 1 0|
|---------------------------------------------------------------|
|    funct7   |   rs2   |   rs1   |  f3 |    rd   |    opcode   |  R-type
|   rs3   | f2|   rs2   |   rs1   |  f3 |    rd   |    opcode   |  R4-type
|          imm          |   rs1   |  f3 |    rd   |    opcode   |  I-type
|===============================================================|
|  0000000    |   rs2   |   rs1   | 111 |    rd   |   0110011   |  AND*
|  0000000    |   rs2   |   rs1   | 110 |    rd   |   0110011   |  OR*
|  0000000    |   rs2   |   rs1   | 100 |    rd   |   0110011   |  XOR*
|  0100000    |   rs2   |   rs1   | 111 |    rd   |   0110011   |  ANDN
|  0100000    |   rs2   |   rs1   | 110 |    rd   |   0110011   |  ORN
|  0100000    |   rs2   |   rs1   | 100 |    rd   |   0110011   |  XNOR
|---------------------------------------------------------------|
|  0000000    |   rs2   |   rs1   | 001 |    rd   |   0110011   |  SLL*
|  0000000    |   rs2   |   rs1   | 101 |    rd   |   0110011   |  SRL*
|  0100000    |   rs2   |   rs1   | 101 |    rd   |   0110011   |  SRA*
|  0010000    |   rs2   |   rs1   | 001 |    rd   |   0110011   |  SLO
|  0010000    |   rs2   |   rs1   | 101 |    rd   |   0110011   |  SRO
|  0110000    |   rs2   |   rs1   | 001 |    rd   |   0110011   |  ROL
|  0110000    |   rs2   |   rs1   | 101 |    rd   |   0110011   |  ROR
|---------------------------------------------------------------|
|  0010000    |   rs2   |   rs1   | 010 |    rd   |   0110011   |  SH1ADD
|  0010000    |   rs2   |   rs1   | 100 |    rd   |   0110011   |  SH2ADD
|  0010000    |   rs2   |   rs1   | 110 |    rd   |   0110011   |  SH3ADD
|---------------------------------------------------------------|
|  0100100    |   rs2   |   rs1   | 001 |    rd   |   0110011   |  SBCLR
|  0010100    |   rs2   |   rs1   | 001 |    rd   |   0110011   |  SBSET
|  0110100    |   rs2   |   rs1   | 001 |    rd   |   0110011   |  SBINV
|  0100100    |   rs2   |   rs1   | 101 |    rd   |   0110011   |  SBEXT
|  0010100    |   rs2   |   rs1   | 101 |    rd   |   0110011   |  GORC
|  0110100    |   rs2   |   rs1   | 101 |    rd   |   0110011   |  GREV
|---------------------------------------------------------------|
|  00000  |     imm     |   rs1   | 001 |    rd   |   0010011   |  SLLI*
|  00000  |     imm     |   rs1   | 101 |    rd   |   0010011   |  SRLI*
|  01000  |     imm     |   rs1   | 101 |    rd   |   0010011   |  SRAI*
|  00100  |     imm     |   rs1   | 001 |    rd   |   0010011   |  SLOI
|  00100  |     imm     |   rs1   | 101 |    rd   |   0010011   |  SROI
|  01100  |     imm     |   rs1   | 101 |    rd   |   0010011   |  RORI
|---------------------------------------------------------------|
|  01001  |     imm     |   rs1   | 001 |    rd   |   0010011   |  SBCLRI
|  00101  |     imm     |   rs1   | 001 |    rd   |   0010011   |  SBSETI
|  01101  |     imm     |   rs1   | 001 |    rd   |   0010011   |  SBINVI
|  01001  |     imm     |   rs1   | 101 |    rd   |   0010011   |  SBEXTI
|  00101  |     imm     |   rs1   | 101 |    rd   |   0010011   |  GORCI
|  01101  |     imm     |   rs1   | 101 |    rd   |   0010011   |  GREVI
|---------------------------------------------------------------|
|   rs3   | 11|   rs2   |   rs1   | 001 |    rd   |   0110011   |  CMIX
|   rs3   | 11|   rs2   |   rs1   | 101 |    rd   |   0110011   |  CMOV
|   rs3   | 10|   rs2   |   rs1   | 001 |    rd   |   0110011   |  FSL
|   rs3   | 10|   rs2   |   rs1   | 101 |    rd   |   0110011   |  FSR
|   rs3   |1|    imm    |   rs1   | 101 |    rd   |   0010011   |  FSRI
|---------------------------------------------------------------|
\end{verbatim}
\end{minipage}

\begin{minipage}{\linewidth}
\begin{verbatim}
|  3                   2                   1                    |
|1 0 9 8 7 6 5 4 3 2 1 0 9 8 7 6 5 4 3 2 1 0 9 8 7 6 5 4 3 2 1 0|
|===============================================================|
|  0110000    |  00000  |   rs1   | 001 |    rd   |   0010011   |  CLZ
|  0110000    |  00001  |   rs1   | 001 |    rd   |   0010011   |  CTZ
|  0110000    |  00010  |   rs1   | 001 |    rd   |   0010011   |  PCNT
|  0110000    |  00011  |   rs1   | 001 |    rd   |   0010011   |  BMATFLIP
|  0110000    |  00100  |   rs1   | 001 |    rd   |   0010011   |  SEXT.B
|  0110000    |  00101  |   rs1   | 001 |    rd   |   0010011   |  SEXT.H
|---------------------------------------------------------------|
|  0110000    |  10000  |   rs1   | 001 |    rd   |   0010011   |  CRC32.B
|  0110000    |  10001  |   rs1   | 001 |    rd   |   0010011   |  CRC32.H
|  0110000    |  10010  |   rs1   | 001 |    rd   |   0010011   |  CRC32.W
|  0110000    |  10011  |   rs1   | 001 |    rd   |   0010011   |  CRC32.D
|  0110000    |  11000  |   rs1   | 001 |    rd   |   0010011   |  CRC32C.B
|  0110000    |  11001  |   rs1   | 001 |    rd   |   0010011   |  CRC32C.H
|  0110000    |  11010  |   rs1   | 001 |    rd   |   0010011   |  CRC32C.W
|  0110000    |  11011  |   rs1   | 001 |    rd   |   0010011   |  CRC32C.D
|---------------------------------------------------------------|
|  0000101    |   rs2   |   rs1   | 001 |    rd   |   0110011   |  CLMUL
|  0000101    |   rs2   |   rs1   | 010 |    rd   |   0110011   |  CLMULR
|  0000101    |   rs2   |   rs1   | 011 |    rd   |   0110011   |  CLMULH
|  0000101    |   rs2   |   rs1   | 100 |    rd   |   0110011   |  MIN
|  0000101    |   rs2   |   rs1   | 101 |    rd   |   0110011   |  MINU
|  0000101    |   rs2   |   rs1   | 110 |    rd   |   0110011   |  MAX
|  0000101    |   rs2   |   rs1   | 111 |    rd   |   0110011   |  MAXU
|---------------------------------------------------------------|
|  0000100    |   rs2   |   rs1   | 001 |    rd   |   0110011   |  SHFL
|  0000100    |   rs2   |   rs1   | 101 |    rd   |   0110011   |  UNSHFL
|  0100100    |   rs2   |   rs1   | 110 |    rd   |   0110011   |  BDEP
|  0000100    |   rs2   |   rs1   | 110 |    rd   |   0110011   |  BEXT
|  0000100    |   rs2   |   rs1   | 100 |    rd   |   0110011   |  PACK
|  0100100    |   rs2   |   rs1   | 100 |    rd   |   0110011   |  PACKU
|  0000100    |   rs2   |   rs1   | 011 |    rd   |   0110011   |  BMATOR
|  0100100    |   rs2   |   rs1   | 011 |    rd   |   0110011   |  BMATXOR
|  0000100    |   rs2   |   rs1   | 111 |    rd   |   0110011   |  PACKH
|  0100100    |   rs2   |   rs1   | 111 |    rd   |   0110011   |  BFP
|---------------------------------------------------------------|
|  000010   |    imm    |   rs1   | 001 |    rd   |   0010011   |  SHFLI
|  000010   |    imm    |   rs1   | 101 |    rd   |   0010011   |  UNSHFLI
|===============================================================|
|  00001  |     imm     |   rs1   | 001 |    rd   |   0011011   |  SLLIU.W
|  0000100    |   rs2   |   rs1   | 000 |    rd   |   0111011   |  ADDU.W
|---------------------------------------------------------------|
\end{verbatim}
\end{minipage}

\begin{minipage}{\linewidth}
\begin{verbatim}
|  3                   2                   1                    |
|1 0 9 8 7 6 5 4 3 2 1 0 9 8 7 6 5 4 3 2 1 0 9 8 7 6 5 4 3 2 1 0|
|===============================================================|
|  0010000    |   rs2   |   rs1   | 001 |    rd   |   0111011   |  SLOW
|  0010000    |   rs2   |   rs1   | 101 |    rd   |   0111011   |  SROW
|  0110000    |   rs2   |   rs1   | 001 |    rd   |   0111011   |  ROLW
|  0110000    |   rs2   |   rs1   | 101 |    rd   |   0111011   |  RORW
|---------------------------------------------------------------|
|  0010000    |   rs2   |   rs1   | 010 |    rd   |   0111011   |  SH1ADDU.W
|  0010000    |   rs2   |   rs1   | 100 |    rd   |   0111011   |  SH2ADDU.W
|  0010000    |   rs2   |   rs1   | 110 |    rd   |   0111011   |  SH3ADDU.W
|---------------------------------------------------------------|
|  0100100    |   rs2   |   rs1   | 001 |    rd   |   0111011   |  SBCLRW
|  0010100    |   rs2   |   rs1   | 001 |    rd   |   0111011   |  SBSETW
|  0110100    |   rs2   |   rs1   | 001 |    rd   |   0111011   |  SBINVW
|  0100100    |   rs2   |   rs1   | 101 |    rd   |   0111011   |  SBEXTW
|  0010100    |   rs2   |   rs1   | 101 |    rd   |   0111011   |  GORCW
|  0110100    |   rs2   |   rs1   | 101 |    rd   |   0111011   |  GREVW
|---------------------------------------------------------------|
|  0010000    |   imm   |   rs1   | 001 |    rd   |   0011011   |  SLOIW
|  0010000    |   imm   |   rs1   | 101 |    rd   |   0011011   |  SROIW
|  0110000    |   imm   |   rs1   | 101 |    rd   |   0011011   |  RORIW
|---------------------------------------------------------------|
|  0100100    |   imm   |   rs1   | 001 |    rd   |   0011011   |  SBCLRIW
|  0010100    |   imm   |   rs1   | 001 |    rd   |   0011011   |  SBSETIW
|  0110100    |   imm   |   rs1   | 001 |    rd   |   0011011   |  SBINVIW
|  0010100    |   imm   |   rs1   | 101 |    rd   |   0011011   |  GORCIW
|  0110100    |   imm   |   rs1   | 101 |    rd   |   0011011   |  GREVIW
|---------------------------------------------------------------|
|   rs3   | 10|   rs2   |   rs1   | 001 |    rd   |   0111011   |  FSLW
|   rs3   | 10|   rs2   |   rs1   | 101 |    rd   |   0111011   |  FSRW
|   rs3   | 10|   imm   |   rs1   | 101 |    rd   |   0011011   |  FSRIW
|---------------------------------------------------------------|
|  0110000    |  00000  |   rs1   | 001 |    rd   |   0011011   |  CLZW
|  0110000    |  00001  |   rs1   | 001 |    rd   |   0011011   |  CTZW
|  0110000    |  00010  |   rs1   | 001 |    rd   |   0011011   |  PCNTW
|---------------------------------------------------------------|
|  0000100    |   rs2   |   rs1   | 001 |    rd   |   0111011   |  SHFLW
|  0000100    |   rs2   |   rs1   | 101 |    rd   |   0111011   |  UNSHFLW
|  0100100    |   rs2   |   rs1   | 110 |    rd   |   0111011   |  BDEPW
|  0000100    |   rs2   |   rs1   | 110 |    rd   |   0111011   |  BEXTW
|  0000100    |   rs2   |   rs1   | 100 |    rd   |   0111011   |  PACKW
|  0100100    |   rs2   |   rs1   | 100 |    rd   |   0111011   |  PACKUW
|  0100100    |   rs2   |   rs1   | 111 |    rd   |   0111011   |  BFPW
|---------------------------------------------------------------|
\end{verbatim}
\end{minipage}

\begin{minipage}{\linewidth}
\begin{verbatim}
|  3                   2                   1                    |
|1 0 9 8 7 6 5 4 3 2 1 0 9 8 7 6 5 4 3 2 1 0 9 8 7 6 5 4 3 2 1 0|
|===============================================================|
|  0010100    |   rs2   |   rs1   | 010 |    rd   |   0110011   |  XPERM.N
|  0010100    |   rs2   |   rs1   | 100 |    rd   |   0110011   |  XPERM.B
|  0010100    |   rs2   |   rs1   | 110 |    rd   |   0110011   |  XPERM.H
|  0010100    |   rs2   |   rs1   | 000 |    rd   |   0110011   |  XPERM.W
|---------------------------------------------------------------|
\end{verbatim}
\end{minipage}


Encoding changes in v0.93 of the RISC-V Bitmanip Spec ({\tt +} for addition,
{\tt -} for removal):

\begin{minipage}{\linewidth}
\begin{verbatim}
|  3                   2                   1                    |
|1 0 9 8 7 6 5 4 3 2 1 0 9 8 7 6 5 4 3 2 1 0 9 8 7 6 5 4 3 2 1 0|
|===============================================================|
|  0010000    |   rs2   |   rs1   | 010 |    rd   |   0110011   |  + SH1ADD
|  0010000    |   rs2   |   rs1   | 100 |    rd   |   0110011   |  + SH2ADD
|  0010000    |   rs2   |   rs1   | 110 |    rd   |   0110011   |  + SH3ADD
|---------------------------------------------------------------|
|  0010000    |   rs2   |   rs1   | 010 |    rd   |   0111011   |  + SH1ADDU.W
|  0010000    |   rs2   |   rs1   | 100 |    rd   |   0111011   |  + SH2ADDU.W
|  0010000    |   rs2   |   rs1   | 110 |    rd   |   0111011   |  + SH3ADDU.W
|---------------------------------------------------------------|
|  0000101    |   rs2   |   rs1   | 101 |    rd   |   0110011   |  - MAX
|  0000101    |   rs2   |   rs1   | 110 |    rd   |   0110011   |  - MINU
|  0000101    |   rs2   |   rs1   | 110 |    rd   |   0110011   |  + MAX
|  0000101    |   rs2   |   rs1   | 101 |    rd   |   0110011   |  + MINU
|---------------------------------------------------------------|
|  0100100    |   rs2   |   rs1   | 000 |    rd   |   0111011   |  - SUBU.W
|---------------------------------------------------------------|
|  0010100    |   rs2   |   rs1   | 010 |    rd   |   0110011   |  + XPERM.N
|  0010100    |   rs2   |   rs1   | 100 |    rd   |   0110011   |  + XPERM.B
|  0010100    |   rs2   |   rs1   | 110 |    rd   |   0110011   |  + XPERM.H
|  0010100    |   rs2   |   rs1   | 000 |    rd   |   0110011   |  + XPERM.W
|---------------------------------------------------------------|
|       immediate       |   rs1   | 100 |    rd   |   0011011   |  - ADDIWU
|  0000101    |   rs2   |   rs1   | 000 |    rd   |   0111011   |  - ADDWU
|  0100101    |   rs2   |   rs1   | 000 |    rd   |   0111011   |  - SUBWU
|---------------------------------------------------------------|
|  0000101    |   rs2   |   rs1   | 001 |    rd   |   0111011   |  - CLMULW
|  0000101    |   rs2   |   rs1   | 010 |    rd   |   0111011   |  - CLMULRW
|  0000101    |   rs2   |   rs1   | 011 |    rd   |   0111011   |  - CLMULHW
|---------------------------------------------------------------|
\end{verbatim}
\end{minipage}

\begin{figure}[t]
\begin{center}
\begin{minipage}{\linewidth}
\begin{verbnobox}[\tiny]
|   funct7  |                                             funct3                                            |
|30|29|27|25|    001    |    101    |    000    |    010    |    011    |    100    |    110    |    111    |
|-----------|-----------------------------------------------------------------------------------------------|
|  -00-0-0  |    SLL    |    SRL    |    ADD    |    SLT^   |    SLTU^  |    XOR^   |     OR^   |    AND^   |
|  -10-0-0  |           |    SRA    |    SUB    |           |           |    XNOR^  |     ORN^  |    ANDN^  |
|-----------|-----------------------------------------------------------------------------------------------|
|  -00-0-1  |   MULH^(2)|   DIVU (2)|    MUL    |   MULHSU^ |   MULHU^  |    DIV    |    REM    |    REMU   |
|  -10-0-1  |        (2)|        (2)|           |           |           |           |           |           |
|-----------|-----------------------------------------------------------------------------------------------|
|  -00-1-0  |   SHFL (4)|   UNSHFL  | ADDU.W (1)|           |   BMATOR^ |    PACK   |    BEXT   |   PACKH^  |
|  -10-1-0  |   SBCLR   |   SBEXT   |           |           |  BMATXOR^ |   PACKU   |    BDEP   |    BFP    |
|-----------|-----------------------------------------------------------------------------------------------|
|  -00-1-1  |  CLMUL^(2)|   MINU^(2)|           |   CLMULR^ |   CLMULH^ |    MIN^   |    MAX^   |    MAXU^  |
|  -10-1-1  |        (2)|        (2)|           |           |           |           |           |           |
|-----------|-----------------------------------------------------------------------------------------------|
|  -01-0-0  |    SLO    |    SRO    |  (SH4ADD) |   SH1ADD  |           |   SH2ADD  |   SH3ADD  |           |
|  -11-0-0  |    ROL (3)|    ROR    |           |           |           |           |           |           |
|-----------|-----------------------------------------------------------------------------------------------|
|  -01-0-1  |        (2)|        (2)|           |           |           |           |           |           |
|  -11-0-1  |        (2)|        (2)|           |           |           |           |           |           |
|-----------|-----------------------------------------------------------------------------------------------|
|  -01-1-0  |   SBSET   |    GORC   |  XPERM.W  |  XPERM.N  |           |  XPERM.B  |  XPERM.H  |           |
|  -11-1-0  |   SBINV   |    GREV   |           |           |           |           |           |           |
|-----------|-----------------------------------------------------------------------------------------------|
|  -01-1-1  |        (2)|        (2)|           |           |           |           |           |           |
|  -11-1-1  |        (2)|        (2)|           |           |           |           |           |           |
|-----------|-----------------------------------------------------------------------------------------------|
Notes:
- funct7 bits not shown: 31,28 unused (always zero), 26 used for ternary instructions
- columns reordered to show shift opcodes on the left
- rows reordered in groups with and without bit 30 set
(1) These instructions only exist in OP-32.
(2) No "shift-immediate" encoding for opcodes with bit 25 set.
(3) All unary instructions use the code space for the non-existing ROLI instruction.
(4) SLLIU.W is encoded in the code space for the non-existing SHFLIW instruction.
^ Instructions marked with ^ have no *W equivalent in OP-32
\end{verbnobox}
\end{minipage}
\end{center}
\caption{OP/OP-32 Binary Instruction Map}
\label{op-op32-bin}
\end{figure}

\begin{figure}[t]
\begin{center}
\begin{minipage}{\linewidth}
\begin{verbnobox}[\tiny]
|   funct7  |                                             funct3                                            |
| RS3|26|25 |    001    |    101    |    000    |    010    |    011    |    100    |    110    |    111    |
|-----------|-----------------------------------------------------------------------------------------------|
|  -----10  |    FSL (2)|    FSR (1)|           |           |           |           |           |           |
|  -----11  |    CMIX   |    CMOV   |           |           |           |           |           |           |
|-----------|-----------------------------------------------------------------------------------------------|
Notes:
- funct7 bits: bits 31:27 RS3, bit 26 always set for ternary instructions
- columns reordered to show shift opcodes on the left
(1) There is also an FSRI instruction in OP-IMM/OP-IMM-32 (except OP-IMM on RV128)
(2) The encoding for the non-existing FSLI instruction is reserved
(RV128 FSRI could be implemented using the reserved FSLI opcodes in OP-IMM-32/OP-IMM-64)
\end{verbnobox}
\end{minipage}
\end{center}
\caption{OP/OP-32 Ternary Instruction Map}
\label{op-op32-tern}
\end{figure}

%%%%%%%%%%%%%%%%%%%%%%%%%%%%%%%%%%%%%%%%%%%%%%%%%%%%%%%%%%%%%%%%%%%%%%%%%%%%%%%%%%%%%%%%%%%

\section{Future compressed instructions}

The RISC-V ISA has no dedicated instructions for bitwise inverse (\texttt{not}).
Instead \texttt{not} is implemented as \texttt{xori\ rd,\ rs,\ -1} and
\texttt{neg} is implemented as \texttt{sub\ rd,\ x0,\ rs}.

In bitmanipulation code \texttt{not} is a very common operation. But there is
no compressed encoding for those operation because there is no \texttt{c.xori}
instruction.

On RV64 (and RV128) {\tt zext.w} and {\tt zext.d} ({\tt pack} and {\tt packw})
are commonly used to zero-extend unsigned values $<$XLEN.

It presumably would make sense for a future revision of the ``C'' extension to
include compressed opcodes for those instructions.

An encoding with the constraint \texttt{rd $=$ rs} would fit nicely in the
reserved space in \texttt{c.addi16sp/c.lui}.

\begin{table}[h]
\begin{small}
\begin{center}
\begin{tabular}{p{0in}p{0.05in}p{0.05in}p{0.05in}p{0.05in}p{0.05in}p{0.05in}p{0.05in}p{0.05in}p{0.05in}p{0.05in}p{0.05in}p{0.05in}p{0.05in}p{0.05in}p{0.05in}p{0.05in}l}
& & & & & & & & & & \\
                      &
\instbit{15} &
\instbit{14} &
\instbit{13} &
\multicolumn{1}{c}{\instbit{12}} &
\instbit{11} &
\instbit{10} &
\instbit{9} &
\instbit{8} &
\instbit{7} &
\instbit{6} &
\multicolumn{1}{c}{\instbit{5}} &
\instbit{4} &
\instbit{3} &
\instbit{2} &
\instbit{1} &
\instbit{0} \\
\cline{2-17}

&
\multicolumn{3}{|c|}{011} &
\multicolumn{1}{c|}{nzimm[9]} &
\multicolumn{5}{c|}{2} &
\multicolumn{5}{c|}{nzimm[4$\vert$6$\vert$8:7$\vert$5]} &
\multicolumn{2}{c|}{01} & C.ADDI16SP {\em \tiny (\sout{RES, nzimm=0})} \\
\cline{2-17}

&
\multicolumn{3}{|c|}{011} &
\multicolumn{1}{c|}{nzimm[17]} &
\multicolumn{5}{c|}{rd$\neq$$\{0,2\}$} &
\multicolumn{5}{c|}{nzimm[16:12]} &
\multicolumn{2}{c|}{01} & C.LUI {\em \tiny (\sout{RES, nzimm=0}; HINT, rd=0)} \\
\cline{2-17}

&
\multicolumn{3}{|c|}{011} &
\multicolumn{1}{c|}{0} &
\multicolumn{2}{c|}{00} &
\multicolumn{3}{c|}{rs1$'$/rd$'$} &
\multicolumn{5}{c|}{0} &
\multicolumn{2}{c|}{01} & C.NOT \\
\cline{2-17}

&
\multicolumn{3}{|c|}{011} &
\multicolumn{1}{c|}{0} &
\multicolumn{2}{c|}{01} &
\multicolumn{3}{c|}{rs1$'$/rd$'$} &
\multicolumn{5}{c|}{0} &
\multicolumn{2}{c|}{01} & C.NEG \\
\cline{2-17}

&
\multicolumn{3}{|c|}{011} &
\multicolumn{1}{c|}{0} &
\multicolumn{2}{c|}{10} &
\multicolumn{3}{c|}{rs1$'$/rd$'$} &
\multicolumn{5}{c|}{0} &
\multicolumn{2}{c|}{01} & C.ZEXT.W {\em \tiny (RV64/128; RV32 RES)} \\
\cline{2-17}

&
\multicolumn{3}{|c|}{011} &
\multicolumn{1}{c|}{0} &
\multicolumn{2}{c|}{11} &
\multicolumn{3}{c|}{rs1$'$/rd$'$} &
\multicolumn{5}{c|}{0} &
\multicolumn{2}{c|}{01} & C.ZEXT.D {\em \tiny (RV128; RV32/64 RES)} \\
\cline{2-17}

\end{tabular}
\end{center}
\end{small}
\end{table}

The entire RVC encoding space is $15.585$~bits wide, the remaining reserved
encoding space in RVC is $11.155$~bits wide, not including space that is only
reserved on RV32/RV64. This means that above encoding would use $0.0065\%$ of
the RVC encoding space, or $1.4\%$ of the remaining reserved RVC encoding
space. Preliminary experiments have shown that NOT instructions alone make up
approximately $1\%$ of bitmanipulation code size.~\cite{Wolf17A}

%%%%%%%%%%%%%%%%%%%%%%%%%%%%%%%%%%%%%%%%%%%%%%%%%%%%%%%%%%%%%%%%%%%%%%%%%%%%%%%%%%%%%%%%%%%

\section{Macro-op fusion patterns}

Some bitmanip operations have been left out of this spec because of lack of a (sensible) way
to encode them in the 32-bit RISC-V encoding space. Instead we present fuse-able sequences
of up to three instructions for those operations, so that high-end processors can implement
them in a single fused macro-op, should they decide to do so.

For this document we only consider sequences fuse-able if they read at most two
registers, only write one register, and contain no branch instructions.

\subsection{Fused {\tt *-bfp} sequences}

The {\tt bfp} instruction is most commonly used in sequences of one the the following forms.

For 32-bit (RV32 or *W instructions on RV64):

\begin{minipage}{\linewidth}
\begin{verbatim}
  addi rd, zero, ...
  pack[w] rd, rs2, rd
  bfp[w] rd, rs1, rd
\end{verbatim}
\end{minipage}

\begin{minipage}{\linewidth}
\begin{verbatim}
  lui rd, ...
  addi rd, rd, ...
  bfp[w] rd, rs1, rd
\end{verbatim}
\end{minipage}

And for 64-bit:

\begin{minipage}{\linewidth}
\begin{verbatim}
  lui rd, ...
  pack rd, rs2, rd
  bfp rd, rs1, rd
\end{verbatim}
\end{minipage}

\subsection{Load-immediate}

For 32-bit code (RV32 or *W instructions on RV64) we recommend to fuse the {\tt
lui+addi} pattern for loading a 32-bit constants:

\begin{minipage}{\linewidth}
\begin{verbatim}
  lui rd, imm
  addi[w] rd, rd, imm
\end{verbatim}
\end{minipage}

Further, for loading 64-bit constants in two macro-ops:

\begin{minipage}{\linewidth}
\begin{verbatim}
  lui rd, imm
  addiw rd, rd, imm
  pack rd, rd, rs
\end{verbatim}
\end{minipage}

\subsection{Fused shift sequences}
\label{postfix-shift-fusion}

Pairs of left and right shifts are common operations for extracting a bit field.

To extract the contiguous bit field starting at {\tt pos} with length {\tt len}
from {\tt rs} (with $\texttt{pos}>0$, $\texttt{len}>0$, and
$\texttt{pos}+\texttt{len}\le\textrm{XLEN}$):

\begin{minipage}{\linewidth}
\begin{verbatim}
  slli rd, rs, (XLEN-len-pos)
  srli rd, rd, (XLEN-len)
\end{verbatim}
\end{minipage}

Using \texttt{srai} instead of \texttt{srli} will sign-extend the extracted bit-field.

Similarly, placing a bit field with length {\tt len} at the position {\tt pos}:

\begin{minipage}{\linewidth}
\begin{verbatim}
  slli rd, rs, (XLEN-len)
  srli rd, rd, (XLEN-len-pos)
\end{verbatim}
\end{minipage}

Note that the postfix right shift instruction can use a compressed encoding,
yielding a 48-bit fused instruction if the left shift is a 32-bit instruction.

More generally, a processor might fuse all destructive shift operations following
any other ALU operation.

We define the following assembler pseudo-ops for {\tt sr[la]i} postfix fusion:

\begin{minipage}{\linewidth}
\begin{verbatim}
  bfext  rd, rs, len, pos    ->   slli rd, rs, (XLEN-len-pos); srai rd, rd, (XLEN-len)
  bfextu rd, rs, len, pos    ->   slli rd, rs, (XLEN-len-pos); srli rd, rd, (XLEN-len)
  bfmak  rd, len, pos        ->   sroi rd, zero, len; srli rd, rd, (XLEN-len-pos)
\end{verbatim}
\end{minipage}

(The names {\tt bfext}, {\tt bfextu}, and {\tt bfmak} are borrowed from m88k,
that had dedicated instructions of those names (without {\tt bf}-prefix) with
equivalent semantics.~\cite[p.~3-28]{m88k})

%%%%%%%%%%%%%%%%%%%%%%%%%%%%%%%%%%%%%%%%%%%%%%%%%%%%%%%%%%%%%%%%%%%%%%%%%%%%%%%%%%%%%%%%%%%

\section{Other micro-architectural considerations}

In addition to macro-op fusion, we issue the following recommendations for
cores that aim at better performance for bitmanipulation tasks.

\subsection{Unaligned memory access}

The base ISA spec requires load/store operations that are not naturally aligned
to succeed in U-mode, but explicitly states that execution may be slow.

For many bitmanipulation tasks it can be of great advantage to be able to
perform load and store operations with arbitrary alignments quickly. Thus we
recommend that cores optimized for bitmanipulation tasks provide fast hardware
support for load/store with arbitrary alignment.

There should be a {\tt getauxval()}-based mechanism as part of the RISC-V Linux
ABI that can be used to query information on support for unaligned memory
access.~\cite{FastLdSt}

\subsection{Fast multiply}

A lot of bitmanipulation tricks rely on multiplication with ``magic numbers'' and
similar tricks involving multiply/divide instructions. Thus, cores optimized for
bitmanipulation tasks should provide reasonably fast implementations of the
``M''-extension multiply/divide instructions.

%%%%%%%%%%%%%%%%%%%%%%%%%%%%%%%%%%%%%%%%%%%%%%%%%%%%%%%%%%%%%%%%%%%%%%%%%%%%%%%%%%%%%%%%%%%

\begin{table}[h]
\begin{center}
\begin{tabular}{l|cc|ccc|}
& \multicolumn{2}{c|}{RV32} & \multicolumn{3}{c|}{RV64} \\
Instruction & {\tt \_rv\_*} & {\tt \_rv32\_*} & {\tt \_rv\_*} & {\tt \_rv32\_*} & {\tt \_rv64\_*} \\
\hline
{\tt clz      } & \ding{52} & \ding{52} & \ding{52} & \ding{52} & \ding{52} \\
{\tt ctz      } & \ding{52} & \ding{52} & \ding{52} & \ding{52} & \ding{52} \\
{\tt pcnt     } & \ding{52} & \ding{52} & \ding{52} & \ding{52} & \ding{52} \\
\hline
{\tt pack     } & \ding{52} & \ding{52} & \ding{52} & \ding{52} & \ding{52} \\
{\tt min      } & \ding{52} & \ding{52} & \ding{52} & \ding{52} & \ding{52} \\
{\tt minu     } & \ding{52} & \ding{52} & \ding{52} & \ding{52} & \ding{52} \\
{\tt max      } & \ding{52} & \ding{52} & \ding{52} & \ding{52} & \ding{52} \\
{\tt maxu     } & \ding{52} & \ding{52} & \ding{52} & \ding{52} & \ding{52} \\
\hline
{\tt sbset    } & \ding{52} & \ding{52} & \ding{52} & \ding{52} & \ding{52} \\
{\tt sbclr    } & \ding{52} & \ding{52} & \ding{52} & \ding{52} & \ding{52} \\
{\tt sbinv    } & \ding{52} & \ding{52} & \ding{52} & \ding{52} & \ding{52} \\
{\tt sbext    } & \ding{52} & \ding{52} & \ding{52} & \ding{52} & \ding{52} \\
\hline
{\tt sll      } & \ding{52} & \ding{52} & \ding{52} & \ding{52} & \ding{52} \\
{\tt srl      } & \ding{52} & \ding{52} & \ding{52} & \ding{52} & \ding{52} \\
{\tt sra      } & \ding{52} & \ding{52} & \ding{52} & \ding{52} & \ding{52} \\
{\tt slo      } & \ding{52} & \ding{52} & \ding{52} & \ding{52} & \ding{52} \\
{\tt sro      } & \ding{52} & \ding{52} & \ding{52} & \ding{52} & \ding{52} \\
{\tt rol      } & \ding{52} & \ding{52} & \ding{52} & \ding{52} & \ding{52} \\
{\tt ror      } & \ding{52} & \ding{52} & \ding{52} & \ding{52} & \ding{52} \\
\hline
{\tt grev     } & \ding{52} & \ding{52} & \ding{52} & \ding{52} & \ding{52} \\
{\tt gorc     } & \ding{52} & \ding{52} & \ding{52} & \ding{52} & \ding{52} \\
{\tt shfl     } & \ding{52} & \ding{52} & \ding{52} & \ding{52} & \ding{52} \\
{\tt unshfl   } & \ding{52} & \ding{52} & \ding{52} & \ding{52} & \ding{52} \\
\hline
{\tt bfp      } & \ding{52} & \ding{52} & \ding{52} & \ding{52} & \ding{52} \\
\hline
{\tt bext     } & \ding{52} & \ding{52} & \ding{52} & \ding{52} & \ding{52} \\
{\tt bdep     } & \ding{52} & \ding{52} & \ding{52} & \ding{52} & \ding{52} \\
\hline
{\tt clmul    } & \ding{52} & \ding{52} & \ding{52} & \ding{52} & \ding{52} \\
{\tt clmulh   } & \ding{52} & \ding{52} & \ding{52} & \ding{52} & \ding{52} \\
{\tt clmulr   } & \ding{52} & \ding{52} & \ding{52} & \ding{52} & \ding{52} \\
\hline
{\tt bmatflip } &           &           & \ding{52} &           & \ding{52} \\
{\tt bmator   } &           &           & \ding{52} &           & \ding{52} \\
{\tt bmatxor  } &           &           & \ding{52} &           & \ding{52} \\
\hline
{\tt fsl      } & \ding{52} & \ding{52} & \ding{52} & \ding{52} & \ding{52} \\
{\tt fsr      } & \ding{52} & \ding{52} & \ding{52} & \ding{52} & \ding{52} \\
\hline
{\tt cmix     } & \ding{52} &           & \ding{52} &           &           \\
{\tt cmov     } & \ding{52} &           & \ding{52} &           &           \\
\hline
{\tt crc32\_b } & \ding{52} &           & \ding{52} &           &           \\
{\tt crc32\_h } & \ding{52} &           & \ding{52} &           &           \\
{\tt crc32\_w } & \ding{52} &           & \ding{52} &           &           \\
{\tt crc32\_d } &           &           & \ding{52} &           &           \\
\hline
{\tt crc32c\_b} & \ding{52} &           & \ding{52} &           &           \\
{\tt crc32c\_h} & \ding{52} &           & \ding{52} &           &           \\
{\tt crc32c\_w} & \ding{52} &           & \ding{52} &           &           \\
{\tt crc32c\_d} &           &           & \ding{52} &           &           \\
\end{tabular}
\end{center}
\caption{C intrinsics defined in {\tt <rvintrin.h>}}
\label{rvintrin}
\end{table}

\section{C intrinsics via {\tt <rvintrin.h>}}

A C header file {\tt <rvintrin.h>} is provided that contains assembler
templates for directly creating assembler instructions from C code.

The header defines {\tt \_rv\_*(...)} functions that operate on the {\tt long}
data type, {\tt \_rv32\_*(...)} functions that operate on the {\tt int32\_t}
data type, and {\tt \_rv64\_*(...)} functions that operate on the {\tt
int64\_t} data type. The {\tt \_rv64\_*(...)} functions are only available on
RV64. See table~\ref{rvintrin} for a complete list of intrinsics defined in
{\tt <rvintrin.h>}.

Usage example:

\begin{minipage}{\linewidth}
\begin{verbatim}
  #include <rvintrin.h>

  int find_nth_set_bit(unsigned int value, int cnt) {
    return _rv32_ctz(_rv32_bdep(1 << cnt, value));
  }
\end{verbatim}
\end{minipage}

Defining {\tt RVINTRIN\_EMULATE} before including {\tt <rvintrin.h>} will
define plain C functions that emulate the behavior of the RISC-V instructions.
This is useful for testing software on non-RISC-V platforms.
