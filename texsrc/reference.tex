% \chapter{Reference Implementations}
% \label{reference}

%%%%%%%%%%%%%%%%%%%%%%%%%%%%%%%%%%%%%%%%%%%%%%%%%%%%%%%%%%%%%%%%%%%%%%%%%%%%%%%%%%%%%%%%%%%

\section{Fast C reference implementations}
\label{fastc}

GCC has intrinsics for the bit counting instructions {\tt clz}, {\tt ctz}, and
{\tt pcnt}.  So a performance-sensitive application (such as an emulator)
should probably just use those:

\input{bextcref-fast-bitcnt}

For processors with BMI2 support GCC has intrinsics for bit extract and bit
deposit instructions (compile with {\tt -mbmi2} and include {\tt <x86intrin.h>}):

\input{bextcref-fast-bext-bmi2}

For other processors we need to provide our own implementations. The following
implementation is a good compromise between code complexity and runtime:

\input{bextcref-fast-bext}

For the other Bitmanip instructions the C reference functions given in Chapter~\ref{bext}
are already reasonably efficient.
