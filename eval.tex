\chapter{Evaluation, Algorithms}

This chapter contains a collection of short code snippets and algorithms using
the XBitmanip extension for evaluation purposes. For the sake of simplicity we
assume RV32 for most examples in this chapter.

Most assembler routines in this chapter are written as if they were ABI functions,
i.e. arguments are passed in a0, a1, \dots\ and results are returned in a0. Registers
t0, t1, \dots\ are used for spilling.

Some of the assembler routines below can not or should not overwrite their
first argument. In those cases the arguments are passed in a1, a2, \dots\ and
results are returned in a0.

% \section{Bit Scanning Operations}
%
% \subsubsection{Count Leading Zeros}
%
% This is trivially just the \texttt{clz} instruction:
%
% \begin{verbatim}
%   clz a0, a0
% \end{verbatim}
%
% \subsubsection{Count Leading Ones}
%
% Simply invert before executing the \texttt{clz} instruction:
%
% \begin{verbatim}
%   not a0, a0
%   clz a0, a0
% \end{verbatim}
%
% \subsubsection{Find index of highest bit set (aka \texttt{ilog2})}
%
% \begin{verbatim}
%   clz a0, a0
%   neg a0, a0
%   addi a0, a0, 32
% \end{verbatim}
%
% \subsubsection{Find index of lowest bit set}
%
% \begin{verbatim}
%   ctz a0, a0
% \end{verbatim}

\section{Emulating other Bit Manipulation ISAs using macro-op fusion}

The following code snippets implement all instructions from the x86 bit manipulation
ISA extensions ABM, BMI1, BMI2, and TBM using RISC-V code that does not spill any
registers and thus could easily be implemented in a single instruction using macro-op
fusion. (Some of them simply map directly to instructions in this spec and so no
macro-op fusion is needed.) Note that shorter RISC-V code sequences are possible if
we allow spilling to temporary registers.

\begin{longtable}[c]{@{}llrrl@{}}
\caption{Emulating other Bit Manipulation ISAs using macro-op fusion}\tabularnewline
\toprule
x86 Ext & x86 Instruction & \multicolumn{2}{c}{Bytes} & RISC-V Code\tabularnewline
        &                 & x86 & RV            &\tabularnewline
\midrule
\endfirsthead
\toprule
x86 Ext & x86 Instruction & \multicolumn{2}{c}{Bytes} & RISC-V Code\tabularnewline
        &                 & x86 & RV            &\tabularnewline
\midrule
\endhead
ABM  & {\tt popcnt}           & 5 &  4 & {\tt pcnt a0, a0}\tabularnewline
\cline{2-5}
     & {\tt lzcnt}            & 5 &  4 & {\tt clz a0, a0}\tabularnewline
\midrule
BMI1 & {\tt andn}             & 5 &  4 & {\tt andc a0, a2, a1}\tabularnewline
\cline{2-5}
     & {\tt bextr} (regs)\footnote{
         The BMI1 {\tt bextr} instruction expects the length and start position packed in one
	 register operand. Our version expects the length in a0, start position in a1, and source
	 value in a2.
                            } & 5 & 12 & {\tt c.add a0, a1}\tabularnewline
     &                        &   &    & {\tt slo a0, zero, a0}\tabularnewline
     &                        &   &    & {\tt c.and a0, a2}\tabularnewline
     &                        &   &    & {\tt srl a0, a0, a1}\tabularnewline
\cline{2-5}
     & {\tt blsi}             & 5 &  6 & {\tt neg a0, a1}\tabularnewline
     &                        &   &    & {\tt c.and a0, a1}\tabularnewline
\cline{2-5}
     & {\tt blsmsk}           & 5 &  6 & {\tt addi a0, a1, -1}\tabularnewline
     &                        &   &    & {\tt c.xor a0, a1}\tabularnewline
\cline{2-5}
     & {\tt blsr}             & 5 &  6 & {\tt addi a0, a1, -1}\tabularnewline
     &                        &   &    & {\tt c.and a0, a1}\tabularnewline
\midrule
BMI2 & {\tt bzhi}             & 5 &  6 & {\tt slo a0, zero, a2}\tabularnewline
     &                        &   &    & {\tt c.and a0, a1}\tabularnewline
\cline{2-5}
     & {\tt mulx}\footnote{
         The \texttt{*x} BMI2 isnstructions just perform the indicated operation without
	 changing any flags. RISC-V does not use flags, so this instructions trivially
	 just map to their regular RISC-V counterparts.
                \label{bmix}} & 5 &  4 & {\tt mul}\tabularnewline
\cline{2-5}
     & {\tt pdep}             & 5 &  4 & {\tt bdep}\tabularnewline
\cline{2-5}
     & {\tt pext}             & 5 &  4 & {\tt bext}\tabularnewline
\cline{2-5}
     & {\tt rorx}$^{\ref{bmix}}$ & 6 &  4 & {\tt rori}\tabularnewline
\cline{2-5}
     & {\tt sarx}$^{\ref{bmix}}$ & 5 &  4 & {\tt sra}\tabularnewline
\cline{2-5}
     & {\tt shrx}$^{\ref{bmix}}$ & 5 &  4 & {\tt srl}\tabularnewline
\cline{2-5}
     & {\tt shlx}$^{\ref{bmix}}$ & 5 &  4 & {\tt sll}\tabularnewline
\midrule
TBM  & {\tt bextr} (imm)      & 7 &  4 & {\tt c.slli a0, (32-START-LEN)}\tabularnewline
     &                        &   &    & {\tt c.srli a0, (32-LEN)}\tabularnewline
\cline{2-5}
     & {\tt blcfill}          & 5 &  6 & {\tt addi a0, a1, 1}\tabularnewline
     &                        &   &    & {\tt c.and a0, a1}\tabularnewline
\cline{2-5}
     & {\tt blci}             & 5 &  8 & {\tt addi a0, a1, 1}\tabularnewline
     &                        &   &    & {\tt c.not a0}\tabularnewline
     &                        &   &    & {\tt c.or a0, a1}\tabularnewline
\cline{2-5}
     & {\tt blcic}            & 5 & 10 & {\tt addi a0, a1, 1}\tabularnewline
     &                        &   &    & {\tt andc a0, a1, a0}\tabularnewline
     &                        &   &    & {\tt c.not a0}\tabularnewline
\cline{2-5}
     & {\tt blcmsk}           & 5 &  6 & {\tt addi a0, a1, 1}\tabularnewline
     &                        &   &    & {\tt c.xor a0, a1}\tabularnewline
\cline{2-5}
     & {\tt blcs}             & 5 &  6 & {\tt addi a0, a1, 1}\tabularnewline
     &                        &   &    & {\tt c.or a0, a1}\tabularnewline
\cline{2-5}
     & {\tt blsfill}          & 5 &  6 & {\tt addi a0, a1, -1}\tabularnewline
     &                        &   &    & {\tt c.or a0, a1}\tabularnewline
\cline{2-5}
     & {\tt blsic}            & 5 & 10 & {\tt addi a0, a1, -1}\tabularnewline
     &                        &   &    & {\tt andc a0, a1, a0}\tabularnewline
     &                        &   &    & {\tt c.not a0}\tabularnewline
\cline{2-5}
     & {\tt t1mskc}           & 5 & 10 & {\tt addi a0, a1, +1}\tabularnewline
     &                        &   &    & {\tt andc a0, a1, a0}\tabularnewline
     &                        &   &    & {\tt c.not a0}\tabularnewline
\cline{2-5}
     & {\tt t1msk}            & 5 &  8 & {\tt addi a0, a1, -1}\tabularnewline
     &                        &   &    & {\tt andc a0, a0, a1}\tabularnewline
\bottomrule
\end{longtable}

There will be a separate RISC-V standard for recommended sequences for macro-op fusion.
The macros listed here are merely for demonstrating that suitable sequences exist. We
do not advocate for any of those sequences to become ``standard sequences'' for macro-op
fusion.

\section{Decoding RISC-V Immediates}

The following code snippets decode the immedate from RISC-V S-type, B-type, J-type,
and CJ-type instructions. They are nice ``nothing up my sleeve''-examples for
real-world bit permutations.

\begin{small}
\begin{center}
\begin{tabular}{p{0in}p{0.4in}p{0.05in}p{0.05in}p{0.05in}p{0.05in}p{0.4in}p{0.6in}p{0.4in}p{0.6in}p{0.7in}l}
& & & & & & & & & & \\
                      &
\multicolumn{1}{l}{\instbit{31}} &
\multicolumn{1}{r}{\instbit{27}} &
\instbit{26} &
\instbit{25} &
\multicolumn{1}{l}{\instbit{24}} &
\multicolumn{1}{r}{\instbit{20}} &
\instbitrange{19}{15} &
\instbitrange{14}{12} &
\instbitrange{11}{7} &
\instbitrange{6}{0} \\
\cline{2-11}

&
\multicolumn{4}{|c|}{imm[11:5]} &
\multicolumn{2}{c|}{} &
\multicolumn{1}{c|}{} &
\multicolumn{1}{c|}{} &
\multicolumn{1}{c|}{imm[4:0]} &
\multicolumn{1}{c|}{} & S-type \\
\cline{2-11}

&
\multicolumn{4}{|c|}{imm[12$\vert$10:5]} &
\multicolumn{2}{c|}{} &
\multicolumn{1}{c|}{} &
\multicolumn{1}{c|}{} &
\multicolumn{1}{c|}{imm[4:1$\vert$11]} &
\multicolumn{1}{c|}{} & B-type \\
\cline{2-11}

&
\multicolumn{8}{|c|}{imm[20$\vert$10:1$\vert$11$\vert$19:12]} &
\multicolumn{1}{c|}{} &
\multicolumn{1}{c|}{} & J-type \\
\cline{2-11}

\end{tabular}

\begin{tabular}{p{0in}p{0.05in}p{0.05in}p{0.05in}p{0.05in}p{0.05in}p{0.05in}p{0.05in}p{0.05in}p{0.05in}p{0.05in}p{0.05in}p{0.05in}p{0.05in}p{0.05in}p{0.05in}p{0.05in}l}
& & & & & & & & & & \\
                      &
\instbit{15} &
\instbit{14} &
\instbit{13} &
\multicolumn{1}{c}{\instbit{12}} &
\instbit{11} &
\instbit{10} &
\instbit{9} &
\instbit{8} &
\instbit{7} &
\instbit{6} &
\multicolumn{1}{c}{\instbit{5}} &
\instbit{4} &
\instbit{3} &
\instbit{2} &
\instbit{1} &
\instbit{0} \\
\cline{2-17}

&
\multicolumn{3}{|c|}{} &
\multicolumn{11}{c|}{imm[11$\vert$4$\vert$9:8$\vert$10$\vert$6$\vert$7$\vert$3:1$\vert$5]} &
\multicolumn{2}{c|}{} & CJ-type \\
\cline{2-17}

\end{tabular}
\end{center}
\end{small}

\begin{verbatim}
  decode_s:
    li t0, 0xfe000f80
    bext a0, a0, t0
    c.slli a0, 20
    c.srai a0, 20
    ret

  decode_b:
    lui t0, 0x51574
    shuffle a0, a0, t0
    li t0, 0xeaa80055
    bext a0, a0, t0
    c.slli a0, 20
    c.srai a0, 19
    ret

  // variant 1 (with shuffle/unshuffle/bext)
  decode_j:
    lui t0, 0x0fff3
    unshuffle a0, a0, t0
    lui t0, 0x34349
    shuffle a0, a0, t0
    lui t0, 0x70fe4
    shuffle a0, a0, t0
    li t0, 0x8ff170fe
    bext a0, a0, t0
    c.slli a0, 12
    c.srai a0, 11
    ret

  // variant 2 (with bext but without shuffle/unshuffle)
  decode_j:
    li t0, 0x800ff000
    li a1, 0x00100000
    bext a2, a0, t0
    c.and a1, a0
    c.slli a0, a0, 1
    c.srli a0, a0, 22
    c.slli a2, 23
    c.slli a1, 2
    c.slli a0, 12
    c.or a0, a2
    c.or a0, a1
    c.srai a0, 11
    ret

  // variant 1 (with shuffle/unshuffle/bext)
  decode_cj:
    rori a0, a0, 24
    lui t0, 0x399cb
    shuffle a0, a0, t0
    lui t0, 0xf2d30
    unshuffle a0, a0, t0
    li t0, 0x538a2046
    bext a0, a0, t0
    c.slli a0, 21
    c.srai a0, 20
    ret

  // variant 2 (without shuffle/unshuffle/bext)
  decode_cj:
    srli a5, a0, 2
    srli a4, a0, 7
    c.andi a4, 16
    slli a3, a0, 3
    c.andi a5, 14
    c.add a5, a4
    andi a3, a3, 32
    srli a4, a0, 1
    c.add a5, a3
    andi a4, a4, 64
    slli a2, a0, 1
    c.add a5, a4
    andi a2, a2, 128
    srli a3, a0, 1
    slli a4, a0, 19
    c.add a5, a2
    andi a3, a3, 768
    c.slli a0, 2
    c.add a5, a3
    andi a0, a0, 1024
    c.srai a4, 31
    c.add a5, a0
    slli a0, a4, 11
    c.add a0, a5
    ret
\end{verbatim}

% \section{Butterfly Operations}
%
% The \texttt{grev}, \texttt{bext}, and \texttt{bdep} instructions are commonly
% implemented using butterfly circuits\footnote{http://programming.sirrida.de/bit\_perm.html\#butterfly}.
% (Butterfly circuits are a powerful tool for implementing bit permutations.)
% However, sometimes one wants to use butterfly operations directly in a program
% and it is not immediately obvious how to use GREVI and BDEP to perform
% butterfly operations.
%
% One can easily implement a single butterfly stage using GREVI in just four
% instructions:
%
% \begin{verbatim}
%   grevi t0, a0, N
%   and t0, t0, a1
%   andc a0, a0, a1
%   or a0, a0, t0
% \end{verbatim}
%
% With {\tt a0} containing the data input on entry and the result on exit. {\tt N} is a
% power of two indicating which butterfly stage to implement, and {\tt a1} contains
% an XLEN bitmask derived from the XLEN/2 butterfly control word.
%
% This bitmask can either be pre-computed or, if it needs to be generated
% on-the-fly, it can easily be calculated with the help of BDEP. For example
% for the first butterfly stage (using a0 as input and output):
%
% \begin{verbatim}
%   li t0, 0x55555555
%   bdep t0, a0, t0
%   slli a0, t0, 1
%   or a0, a0, t0
% \end{verbatim}
%
% Likewise for the second butterfly stage:
%
% \begin{verbatim}
%   li t0, 0x33333333
%   bdep t0, a0, t0
%   slli a0, t0, 2
%   or a0, a0, t0
% \end{verbatim}
%
% For the third stage:
%
% \begin{verbatim}
%   li t0, 0x0f0f0f0f
%   bdep t0, a0, t0
%   slli a0, t0, 4
%   or a0, a0, t0
% \end{verbatim}
%
% For the fourth stage:
%
% \begin{verbatim}
%   li t0, 0x00ff00ff
%   bdep t0, a0, t0
%   slli a0, t0, 8
%   or a0, a0, t0
% \end{verbatim}
%
% And for the fifth stage:
%
% \begin{verbatim}
%   li t0, 0x0000ffff
%   and t0, a0, t0
%   slli a0, t0, 16
%   or a0, a0, t0
% \end{verbatim}
