\chapter{Evaluation}

This chapter contains a collection of short code snippets and algorithms using
the XBitmanip extension for evaluation purposes. For the sake of simplicity we
assume RV32 for most examples in this chapter.

Most assembler routines in this chapter are written as if they were ABI functions,
i.e. arguments are passed in a0, a1,~\dots~and results are returned in a0. Registers
t0, t1,~\dots~and a0, a1,~\dots~are used for spilling.

Some of the assembler routines below can not or should not overwrite their
first argument. In those cases the arguments are passed in a1, a2, \dots\ and
results are returned in a0.

\section{MIX/MUX pattern}

A MIX pattern selects bits from {\tt a0} and {\tt a1} based on the bits in
the control word {\tt a2}.

\begin{verbatim}
  c.and a0, a2
  andc a1, a1, a2
  c.or a0, a1
\end{verbatim}

A MUX operation selects word {\tt a0} or {\tt a1} based on if the control
word {\tt a2} is zero or nonzero, without branching.

\begin{verbatim}
  snez a2, a2
  c.neg a2
  c.and a0, a2
  andc a1, a1, a2
  c.or a0, a1
\end{verbatim}

Or when {\tt a2} is already either 0 or 1:

\begin{verbatim}
  c.neg a2
  c.and a0, a2
  andc a1, a1, a2
  c.or a0, a1
\end{verbatim}

Alternatively, a core might fuse a conditional branch that just skips one
instruction with that instruction to form a fused conditional macro-op.

\section{Bit scanning and counting}

Counting leading ones:

\begin{verbatim}
  c.not a0
  clz a0, a0
\end{verbatim}

Counting trailing ones:

\begin{verbatim}
  c.not a0
  ctz a0, a0
\end{verbatim}

Counting bits cleared:

\begin{verbatim}
  c.not a0
  pcnt a0, a0
\end{verbatim}

(This is better than XLEN-pcnt because RISC-V has no ``reverse-subtract-immediate'' operation, but
the {\tt not} instruction can even utilize compressed instructions if {\tt rd} $=$ {\tt rs}.)

Odd parity:

\begin{verbatim}
  pcnt a0, a0
  c.andi a0, 1
\end{verbatim}

Even parity:

\begin{verbatim}
  pcnt a0, a0
  c.addi a0, 1
  c.andi a0, 1
\end{verbatim}

(Using {\tt addi} here is better than using {\tt xori}, because there is
a compressed opcode for {\tt addi} but none for {\tt xori}.)

\section{Arbitrary bit permutations}

This section lists code snippets for computing arbitrary bit permutations that
are defined by data (as opposed to bit permutations that are known at compile
time and can likely be compiled into shift-and-mask operations and/or a few
instances of bext/bdep).

\subsection{Using butterfly operations}

The following macro performs a stage-{\tt N} butterfly operation on the word in
{\tt a0} using the mask in {\tt a1}.

\begin{verbatim}
  grevi a2, a0, (1 << N)
  c.and a2, a1
  andc a0, a0, a1
  c.or a0, a2
\end{verbatim}

The bitmask in {\tt a1} must be preformatted correctly for the selected butterfly
stage. A butterfly operation only has a XLEN/2 wide control word. The following
macros format the mask assuming those XLEN/2 bits in the lower half of {\tt a1}
on entry (preformatted mask in {\tt a1} on exit):

\begin{verbatim}
bfly_msk_0:
  zip a1, a1
  slli a2, a1, 1
  c.or a1, a2

bfly_msk_1:
  zip2 a1, a1
  slli a2, a1, 2
  c.or a1, a2

bfly_msk_2:
  zip4 a1, a1
  slli a2, a1, 4
  c.or a1, a2

...
\end{verbatim}

A sequence of $2\cdot{}log_2(\textrm{XLEN})-1$ butterfly operations can perform any
arbitrary bit permutation (Bene{\v s} network):

\begin{verbatim}
  butterfly(LOG2_XLEN-1)
  butterfly(LOG2_XLEN-2)
  ...
  butterfly(0)
  ...
  butterfly(LOG2_XLEN-2)
  butterfly(LOG2_XLEN-1)
\end{verbatim}


Many permutations arising from real-world applications can be implemented
using shorter sequences. For example, any sheep-and-goats operation with either
the sheep or the goats bit reversed can be implemented in $log_2(\textrm{XLEN})$
butterfly operations.

Reversing a permutation implemented using butterfly operations is as simple as
reversing the order of butterfly operations.

% References
% http://www.princeton.edu/~rblee/PUpapers/xiao_spie00.pdf
% https://www.lirmm.fr/arith18/papers/hilewitz-PerformingBitManipulations.pdf
% https://pdfs.semanticscholar.org/bcd0/8fdccf3d5ab959fd81162bd811706ba1676a.pdf

\subsection{Using omega-flip networks}

The omega operation is a stage-0 butterfly preceeded by a zip operation:

\begin{verbatim}
  zip a0, a0
  grevi a2, a0, 1
  c.and a2, a1
  andc a0, a0, a1
  c.or a0, a2
\end{verbatim}

The flip operation is a stage-0 butterfly followed by an unzip operation:

\begin{verbatim}
  grevi a2, a0, 1
  c.and a2, a1
  andc a0, a0, a1
  c.or a0, a2
  unzip a0, a0
\end{verbatim}

A sequence of $log_2(\textrm{XLEN})$ omega operations followed by
$log_2(\textrm{XLEN})$ flip operations can implement any arbitrary 32 bit
permutation.

As for butterfly networks, permutations arising from real-world applications
can often be implemented using a shorter sequence.

% References
% https://ieeexplore.ieee.org/document/878264/
% https://www.princeton.edu/~rblee/ELE572Papers/lee_slideshotchips2002.pdf

\subsection{Using baseline networks}

Another way of implementing arbitrary 32 bit permutations is using a
baseline network followed by an inverse baseline network.

A baseline network is a sequence of $log_2(\textrm{XLEN})$ butterfly(0)
operations interleaved with unzip operations. For example, a 32-bit
baseline network:

\begin{verbatim}
  butterfly(0)
  unzip
  butterfly(0)
  unzip.h
  butterfly(0)
  unzip.b
  butterfly(0)
  unzip.n
  butterfly(0)
\end{verbatim}

An inverse baseline network is a sequence of $log_2(\textrm{XLEN})$ butterfly(0)
operations interleaved with zip operations. The order is opposite to the
order in a baseline network. For example, a 32-bit inverse baseline network:

\begin{verbatim}
  butterfly(0)
  zip.n
  butterfly(0)
  zip.b
  butterfly(0)
  zip.h
  butterfly(0)
  zip
  butterfly(0)
\end{verbatim}

A baseline network followed by an inverse baseline network can implement
any arbitrary bit permutation.

% References
% https://dl.acm.org/citation.cfm?id=1311797

\subsection{Using sheep-and-goats}

The Sheep-and-goats (SAG) operation is a common operation for bit permutations.
It moves all the bits selected by a mask (goats) to the LSB end of the word
and all the remaining bits (sheep) to the MSB end of the word, without changing
the order of sheep or goats.

The SAG operation can easily be performed using {\tt bext} (data in {\tt a0} and
mask in {\tt a1}):

\begin{verbatim}
  bext a2, a0, a1
  c.not a1
  bext a0, a0, a1
  pcnt a1, a1
  ror a0, a0, a1
  c.or a0, a2
\end{verbatim}

Any arbitrary bit permutation can be implemented in $log_2(\textrm{XLEN})$ SAG
operations.

{\it The Hacker's Delight} describes an optimized standard C implementation of
the SAG operation. Their algorithm takes 254 instructions (for 32 bit) or 340
instructions (for 64 bit) on their reference RISC instruction
set.~\cite[p.~152f,~162f]{Seander05}

% References
% Knuth
% Hackers Delight, Chapter 7-7

\section{Emulating x86 Bit Manipulation ISAs}

The following code snippets implement all instructions from the x86 bit manipulation
ISA extensions ABM, BMI1, BMI2, and TBM using RISC-V code that does not spill any
registers and thus could easily be implemented in a single instruction using macro-op
fusion. (Some of them simply map directly to instructions in this spec and so no
macro-op fusion is needed.) Note that shorter RISC-V code sequences are possible if
we allow spilling to temporary registers.

\begin{longtable}[c]{@{}llrrl@{}}
\caption{Emulating other Bit Manipulation ISAs using macro-op fusion}\tabularnewline
\toprule
x86 Ext & x86 Instruction & \multicolumn{2}{c}{Bytes} & RISC-V Code\tabularnewline
        &                 & x86 & RV            &\tabularnewline
\midrule
\endfirsthead
\toprule
x86 Ext & x86 Instruction & \multicolumn{2}{c}{Bytes} & RISC-V Code\tabularnewline
        &                 & x86 & RV            &\tabularnewline
\midrule
\endhead
ABM  & {\tt popcnt}           & 5 &  4 & {\tt pcnt a0, a0}\tabularnewline
\cline{2-5}
     & {\tt lzcnt}            & 5 &  4 & {\tt clz a0, a0}\tabularnewline
\midrule
BMI1 & {\tt andn}             & 5 &  4 & {\tt andc a0, a2, a1}\tabularnewline
\cline{2-5}
     & {\tt bextr} (regs)\footnote{
         The BMI1 {\tt bextr} instruction expects the length and start position packed in one
	 register operand. Our version expects the length in a0, start position in a1, and source
	 value in a2.
                            } & 5 & 12 & {\tt c.add a0, a1}\tabularnewline
     &                        &   &    & {\tt slo a0, zero, a0}\tabularnewline
     &                        &   &    & {\tt c.and a0, a2}\tabularnewline
     &                        &   &    & {\tt srl a0, a0, a1}\tabularnewline
\cline{2-5}
     & {\tt blsi}             & 5 &  6 & {\tt neg a0, a1}\tabularnewline
     &                        &   &    & {\tt c.and a0, a1}\tabularnewline
\cline{2-5}
     & {\tt blsmsk}           & 5 &  6 & {\tt addi a0, a1, -1}\tabularnewline
     &                        &   &    & {\tt c.xor a0, a1}\tabularnewline
\cline{2-5}
     & {\tt blsr}             & 5 &  6 & {\tt addi a0, a1, -1}\tabularnewline
     &                        &   &    & {\tt c.and a0, a1}\tabularnewline
\midrule
BMI2 & {\tt bzhi}             & 5 &  6 & {\tt slo a0, zero, a2}\tabularnewline
     &                        &   &    & {\tt c.and a0, a1}\tabularnewline
\cline{2-5}
     & {\tt mulx}\footnote{
         The \texttt{*x} BMI2 isnstructions just perform the indicated operation without
	 changing any flags. RISC-V does not use flags, so this instructions trivially
	 just map to their regular RISC-V counterparts.
                \label{bmix}} & 5 &  4 & {\tt mul}\tabularnewline
\cline{2-5}
     & {\tt pdep}             & 5 &  4 & {\tt bdep}\tabularnewline
\cline{2-5}
     & {\tt pext}             & 5 &  4 & {\tt bext}\tabularnewline
\cline{2-5}
     & {\tt rorx}$^{\ref{bmix}}$ & 6 &  4 & {\tt rori}\tabularnewline
\cline{2-5}
     & {\tt sarx}$^{\ref{bmix}}$ & 5 &  4 & {\tt sra}\tabularnewline
\cline{2-5}
     & {\tt shrx}$^{\ref{bmix}}$ & 5 &  4 & {\tt srl}\tabularnewline
\cline{2-5}
     & {\tt shlx}$^{\ref{bmix}}$ & 5 &  4 & {\tt sll}\tabularnewline
\midrule
TBM  & {\tt bextr} (imm)      & 7 &  4 & {\tt c.slli a0, (32-START-LEN)}\tabularnewline
     &                        &   &    & {\tt c.srli a0, (32-LEN)}\tabularnewline
\cline{2-5}
     & {\tt blcfill}          & 5 &  6 & {\tt addi a0, a1, 1}\tabularnewline
     &                        &   &    & {\tt c.and a0, a1}\tabularnewline
\cline{2-5}
     & {\tt blci}             & 5 &  8 & {\tt addi a0, a1, 1}\tabularnewline
     &                        &   &    & {\tt c.not a0}\tabularnewline
     &                        &   &    & {\tt c.or a0, a1}\tabularnewline
\cline{2-5}
     & {\tt blcic}            & 5 & 10 & {\tt addi a0, a1, 1}\tabularnewline
     &                        &   &    & {\tt andc a0, a1, a0}\tabularnewline
     &                        &   &    & {\tt c.not a0}\tabularnewline
\cline{2-5}
     & {\tt blcmsk}           & 5 &  6 & {\tt addi a0, a1, 1}\tabularnewline
     &                        &   &    & {\tt c.xor a0, a1}\tabularnewline
\cline{2-5}
     & {\tt blcs}             & 5 &  6 & {\tt addi a0, a1, 1}\tabularnewline
     &                        &   &    & {\tt c.or a0, a1}\tabularnewline
\cline{2-5}
     & {\tt blsfill}          & 5 &  6 & {\tt addi a0, a1, -1}\tabularnewline
     &                        &   &    & {\tt c.or a0, a1}\tabularnewline
\cline{2-5}
     & {\tt blsic}            & 5 & 10 & {\tt addi a0, a1, -1}\tabularnewline
     &                        &   &    & {\tt andc a0, a1, a0}\tabularnewline
     &                        &   &    & {\tt c.not a0}\tabularnewline
\cline{2-5}
     & {\tt t1mskc}           & 5 & 10 & {\tt addi a0, a1, +1}\tabularnewline
     &                        &   &    & {\tt andc a0, a1, a0}\tabularnewline
     &                        &   &    & {\tt c.not a0}\tabularnewline
\cline{2-5}
     & {\tt t1msk}            & 5 &  8 & {\tt addi a0, a1, -1}\tabularnewline
     &                        &   &    & {\tt andc a0, a0, a1}\tabularnewline
\bottomrule
\end{longtable}

There will be a separate RISC-V standard for recommended sequences for macro-op fusion.
The macros listed here are merely for demonstrating that suitable sequences exist. We
do not advocate for any of those sequences to become ``standard sequences'' for macro-op
fusion.

\section{Emulating RI5CY Bit Manipulation ISA}

TBD

\section{Decoding RISC-V Immediates}

The following code snippets decode and sign-extend the immedate from RISC-V
S-type, B-type, J-type, and CJ-type instructions. They are nice ``nothing up my
sleeve''-examples for real-world bit permutations.

\begin{small}
\begin{center}
\begin{tabular}{p{0in}p{0.4in}p{0.05in}p{0.05in}p{0.05in}p{0.05in}p{0.4in}p{0.6in}p{0.4in}p{0.6in}p{0.7in}l}
& & & & & & & & & & \\
                      &
\multicolumn{1}{l}{\instbit{31}} &
\multicolumn{1}{r}{\instbit{27}} &
\instbit{26} &
\instbit{25} &
\multicolumn{1}{l}{\instbit{24}} &
\multicolumn{1}{r}{\instbit{20}} &
\instbitrange{19}{15} &
\instbitrange{14}{12} &
\instbitrange{11}{7} &
\instbitrange{6}{0} \\
\cline{2-11}

&
\multicolumn{4}{|c|}{imm[11:5]} &
\multicolumn{2}{c|}{} &
\multicolumn{1}{c|}{} &
\multicolumn{1}{c|}{} &
\multicolumn{1}{c|}{imm[4:0]} &
\multicolumn{1}{c|}{} & S-type \\
\cline{2-11}

&
\multicolumn{4}{|c|}{imm[12$\vert$10:5]} &
\multicolumn{2}{c|}{} &
\multicolumn{1}{c|}{} &
\multicolumn{1}{c|}{} &
\multicolumn{1}{c|}{imm[4:1$\vert$11]} &
\multicolumn{1}{c|}{} & B-type \\
\cline{2-11}

&
\multicolumn{8}{|c|}{imm[20$\vert$10:1$\vert$11$\vert$19:12]} &
\multicolumn{1}{c|}{} &
\multicolumn{1}{c|}{} & J-type \\
\cline{2-11}

\end{tabular}

\begin{tabular}{p{0in}p{0.05in}p{0.05in}p{0.05in}p{0.05in}p{0.05in}p{0.05in}p{0.05in}p{0.05in}p{0.05in}p{0.05in}p{0.05in}p{0.05in}p{0.05in}p{0.05in}p{0.05in}p{0.05in}l}
& & & & & & & & & & \\
                      &
\instbit{15} &
\instbit{14} &
\instbit{13} &
\multicolumn{1}{c}{\instbit{12}} &
\instbit{11} &
\instbit{10} &
\instbit{9} &
\instbit{8} &
\instbit{7} &
\instbit{6} &
\multicolumn{1}{c}{\instbit{5}} &
\instbit{4} &
\instbit{3} &
\instbit{2} &
\instbit{1} &
\instbit{0} \\
\cline{2-17}

&
\multicolumn{3}{|c|}{} &
\multicolumn{11}{c|}{imm[11$\vert$4$\vert$9:8$\vert$10$\vert$6$\vert$7$\vert$3:1$\vert$5]} &
\multicolumn{2}{c|}{} & CJ-type \\
\cline{2-17}

\end{tabular}
\end{center}
\end{small}

\begin{multicols}{2}
\begin{verbatim}
  decode_s:
    li t0, 0xfe000f80
    bext a0, a0, t0
    c.slli a0, 20
    c.srai a0, 20
    ret

  decode_b:
    li t0, 0xeaa800aa
    rori a0, a0, 8
    grevi a0, a0, 8
    gzip a0, a0, 14
    bext a0, a0, t0
    c.slli a0, 20
    c.srai a0, 19
    ret

  decode_j:
    li t0, 0x800003ff
    li t1, 0x800ff000
    bext a1, a0, t1
    c.slli a1, 23
    rori a0, a0, 21
    bext a0, a0, t0
    c.slli a0, 12
    c.or a0, a1
    c.srai a0, 11
    ret

  // variant 1 (with XBitmanip)
  decode_cj:
    li t0, 0x28800001
    li t1, 0x000016b8
    li t2, 0xb4e00000
    li t3, 0x4b000000
    bext a1, a0, t1
    bdep a1, a1, t2
    rori a0, a0, 11
    bext a0, a0, t0
    bdep a0, a0, t3
    c.or a0, a1
    c.srai a0, 20
    ret

  // variant 2 (without XBitmanip)
  decode_cj:
    srli a5, a0, 2
    srli a4, a0, 7
    c.andi a4, 16
    slli a3, a0, 3
    c.andi a5, 14
    c.add a5, a4
    andi a3, a3, 32
    srli a4, a0, 1
    c.add a5, a3
    andi a4, a4, 64
    slli a2, a0, 1
    c.add a5, a4
    andi a2, a2, 128
    srli a3, a0, 1
    slli a4, a0, 19
    c.add a5, a2
    andi a3, a3, 768
    c.slli a0, 2
    c.add a5, a3
    andi a0, a0, 1024
    c.srai a4, 31
    c.add a5, a0
    slli a0, a4, 11
    c.add a0, a5
    ret
\end{verbatim}
\end{multicols}
