\chapter{RISC-V XBitmanip Extension}

In the proposals provided in this section, the C code examples are for
illustration purposes. They are not optimal implementations, but are
intended to specify the desired functionality.

The sections on encodings are mere placeholders.

%%%%%%%%%%%%%%%%%%%%%%%%%%%%%%%%%%%%%%%%%%%%%%%%%%%%%%%%%%%%%%%%%%%%%%%%%%%%%%%%%%%%%%%%%%%

\section{Count Leading/Trailing Zeros (\texttt{clz, ctz})}

The {\tt clz} operation counts the number of 0 bits at the MSB end of the
argument.  That is, the number of 0 bits before the first 1 bit counting from
the most significant bit. If the input is 0, the output is XLEN. If the input
is -1, the output is 0.

The {\tt ctz} operation counts the number of 0 bits at the LSB end of the
argument. If the input is 0, the output is XLEN. If the input is -1, the
output is 0.

\input{bextcref-clz-ctz}

\vspace{-0.3in}
\begin{center}
\begin{tabular}{M@{}R@{}S@{}R@{}O}
\\
\instbitrange{31}{20} &
\instbitrange{19}{15} &
\instbitrange{14}{12} &
\instbitrange{11}{7} &
\instbitrange{6}{0} \\
\hline
\multicolumn{1}{|c|}{imm[11:0]} &
\multicolumn{1}{c|}{rs1} &
\multicolumn{1}{c|}{funct3} &
\multicolumn{1}{c|}{rd} &
\multicolumn{1}{c|}{opcode} \\
\hline
12 & 5 & 3 & 5 & 7 \\
???????????? & src & CLZ  & dest & OP-IMM \\
???????????? & src & CTZ  & dest & OP-IMM \\
???????????? & src & CLZW  & dest & OP-IMM-32 \\
???????????? & src & CTZW  & dest & OP-IMM-32 \\
\end{tabular}
\end{center}


One possible encoding for \texttt{clz} and \texttt{ctz} is as standard I-type opcodes
somewhere in the brownfield surrounding the shift-immediate instructions.

% \subsection{References}
%
% https://en.wikipedia.org/wiki/Find\_first\_set\#CLZ
%
% https://fgiesen.wordpress.com/2013/10/18/bit-scanning-equivalencies/

%%%%%%%%%%%%%%%%%%%%%%%%%%%%%%%%%%%%%%%%%%%%%%%%%%%%%%%%%%%%%%%%%%%%%%%%%%%%%%%%%%%%%%%%%%%

\section{Count Bits Set (\texttt{pcnt})}

This instruction counts the number of 1 bits in a register. This operations is known as
population count, popcount, sideways sum, bit summation, or Hamming weight.~\cite{HammingWeight,Warren12}

\input{bextcref-pcnt}

\vspace{-0.4in}
\begin{center}
\begin{tabular}{M@{}R@{}S@{}R@{}O}
\\
\instbitrange{31}{20} &
\instbitrange{19}{15} &
\instbitrange{14}{12} &
\instbitrange{11}{7} &
\instbitrange{6}{0} \\
\hline
\multicolumn{1}{|c|}{imm[11:0]} &
\multicolumn{1}{c|}{rs1} &
\multicolumn{1}{c|}{funct3} &
\multicolumn{1}{c|}{rd} &
\multicolumn{1}{c|}{opcode} \\
\hline
12 & 5 & 3 & 5 & 7 \\
???????????? & src & PCNTW  & dest & OP-IMM-32 \\
???????????? & src & PCNT  & dest & OP-IMM \\
\end{tabular}
\end{center}


One possible encoding for \texttt{pcnt} is as a standard I-type opcode somewhere
in the brownfield surrounding the shift-immediate instructions.

%%%%%%%%%%%%%%%%%%%%%%%%%%%%%%%%%%%%%%%%%%%%%%%%%%%%%%%%%%%%%%%%%%%%%%%%%%%%%%%%%%%%%%%%%%%

\section{And-with-complement (\texttt{andc})}

This instruction implements the and-with-complement operation.

\input{bextcref-andc}

Other with-complement operations ({\tt orc, nand, nor}, etc) can be implemented
by combining {\tt not} ({\tt c.not}) with the base ALU operation. (Which can
fit in 32 bit when using two compressed instructions.) Only and-with-complement
occurs frequently enough to warrant a dedicated instruction.

\vspace{-0.4in}
\begin{center}
\begin{tabular}{S@{}R@{}R@{}S@{}R@{}O}
\\
\instbitrange{31}{25} &
\instbitrange{24}{20} &
\instbitrange{19}{15} &
\instbitrange{14}{12} &
\instbitrange{11}{7} &
\instbitrange{6}{0} \\
\hline
\multicolumn{1}{|c|}{funct7} &
\multicolumn{1}{c|}{rs2} &
\multicolumn{1}{c|}{rs1} &
\multicolumn{1}{c|}{funct3} &
\multicolumn{1}{c|}{rd} &
\multicolumn{1}{c|}{opcode} \\
\hline
7 & 5 & 5 & 3 & 5 & 7 \\
??????? & src2 & src1 & ANDC  & dest & OP    \\
??????? & src2 & src1 & ANDCW & dest & OP-32 \\
\end{tabular}
\end{center}


% \subsection{Justification}
%
% http://svn.clifford.at/handicraft/2017/bitcode/
%
% \subsection{References}
%
% http://www.hackersdelight.org/basics2.pdf

%%%%%%%%%%%%%%%%%%%%%%%%%%%%%%%%%%%%%%%%%%%%%%%%%%%%%%%%%%%%%%%%%%%%%%%%%%%%%%%%%%%%%%%%%%%

\section{Shift Ones (Left/Right) (\texttt{slo,\ sloi,\ sro,\ sroi})}

These instructions are similar to shift-logical operations from the base
spec, except instead of shifting in zeros, they shifts in ones.

\input{bextcref-sxo}

\vspace{-0.2in}
\begin{center}
\begin{tabular}{S@{}R@{}R@{}S@{}R@{}O}
\\
\instbitrange{31}{25} &
\instbitrange{24}{20} &
\instbitrange{19}{15} &
\instbitrange{14}{12} &
\instbitrange{11}{7} &
\instbitrange{6}{0} \\
\hline
\multicolumn{1}{|c|}{imm[11:5]} &
\multicolumn{1}{c|}{imm[4:0]} &
\multicolumn{1}{c|}{rs1} &
\multicolumn{1}{c|}{funct3} &
\multicolumn{1}{c|}{rd} &
\multicolumn{1}{c|}{opcode} \\
\hline
7 & 5 & 5 & 3 & 5 & 7 \\
??????? & shamt & src & SLOIW & dest & OP-IMM-32 \\
??????? & shamt & src & SROIW & dest & OP-IMM-32 \\
\end{tabular}
\end{center}

\vspace{-0.2in}
\begin{center}
\begin{tabular}{S@{}R@{}R@{}S@{}R@{}O}
\\
\instbitrange{31}{27} &
\instbitrange{26}{20} &
\instbitrange{19}{15} &
\instbitrange{14}{12} &
\instbitrange{11}{7} &
\instbitrange{6}{0} \\
\hline
\multicolumn{1}{|c|}{imm[11:5]} &
\multicolumn{1}{c|}{imm[4:0]} &
\multicolumn{1}{c|}{rs1} &
\multicolumn{1}{c|}{funct3} &
\multicolumn{1}{c|}{rd} &
\multicolumn{1}{c|}{opcode} \\
\hline
5 & 7 & 5 & 3 & 5 & 7 \\
????? & shamt & src & SLOI & dest & OP-IMM \\
????? & shamt & src & SROI & dest & OP-IMM \\
\end{tabular}
\end{center}

\vspace{-0.2in}
\begin{center}
\begin{tabular}{S@{}R@{}R@{}S@{}R@{}O}
\\
\instbitrange{31}{25} &
\instbitrange{24}{20} &
\instbitrange{19}{15} &
\instbitrange{14}{12} &
\instbitrange{11}{7} &
\instbitrange{6}{0} \\
\hline
\multicolumn{1}{|c|}{funct7} &
\multicolumn{1}{c|}{rs2} &
\multicolumn{1}{c|}{rs1} &
\multicolumn{1}{c|}{funct3} &
\multicolumn{1}{c|}{rd} &
\multicolumn{1}{c|}{opcode} \\
\hline
7 & 5 & 5 & 3 & 5 & 7 \\
??????? & src2 & src1 & SRO    & dest & OP    \\
??????? & src2 & src1 & SLO    & dest & OP    \\
??????? & src2 & src1 & SROW   & dest & OP-32 \\
??????? & src2 & src1 & SLOW   & dest & OP-32 \\
\end{tabular}
\end{center}


\texttt{s(l/r)o(i)} is encoded similarly to the logical shifts in the
base spec. However, the spec of the entire family of instructions is
changed so that the high bit of the instruction indicates the value to
be inserted during a shift. This means that a \texttt{sloi} instruction
can be encoded similarly to an \texttt{slli} instruction, but with a 1
in the highest bit of the encoded instruction. This encoding is
backwards compatible with the definition for the shifts in the base
spec, but allows for simple addition of a ones-insert.

When implementing this circuit, the only change in the ALU over a
standard logical shift is that the value shifted in is not zero, but is
a 1-bit register value that has been forwarded from the high bit of the
instruction decode. This creates the desired behavior on both logical
zero-shifts and logical ones-shifts.

%%%%%%%%%%%%%%%%%%%%%%%%%%%%%%%%%%%%%%%%%%%%%%%%%%%%%%%%%%%%%%%%%%%%%%%%%%%%%%%%%%%%%%%%%%%

\section{Rotate (Left/Right) (\texttt{rol,\ ror,\ rori})}

These instructions are similar to shift-logical operations from the base
spec, except they shift in the values from the opposite side of the
register, in order. This is also called `circular shift'.

\input{bextcref-rox}

\vspace{-0.4in}
\begin{center}
\begin{tabular}{S@{}R@{}R@{}S@{}R@{}O}
\\
\instbitrange{31}{25} &
\instbitrange{24}{20} &
\instbitrange{19}{15} &
\instbitrange{14}{12} &
\instbitrange{11}{7} &
\instbitrange{6}{0} \\
\hline
\multicolumn{1}{|c|}{imm[11:5]} &
\multicolumn{1}{c|}{imm[4:0]} &
\multicolumn{1}{c|}{rs1} &
\multicolumn{1}{c|}{funct3} &
\multicolumn{1}{c|}{rd} &
\multicolumn{1}{c|}{opcode} \\
\hline
7 & 5 & 5 & 3 & 5 & 7 \\
11????? & shamt & src & RORIW & dest & OP-IMM-32 \\
\end{tabular}
\end{center}

\vspace{-0.4in}
\begin{center}
\begin{tabular}{S@{}R@{}R@{}S@{}R@{}O}
\\
\instbitrange{31}{27} &
\instbitrange{26}{20} &
\instbitrange{19}{15} &
\instbitrange{14}{12} &
\instbitrange{11}{7} &
\instbitrange{6}{0} \\
\hline
\multicolumn{1}{|c|}{imm[11:7]} &
\multicolumn{1}{c|}{imm[6:0]} &
\multicolumn{1}{c|}{rs1} &
\multicolumn{1}{c|}{funct3} &
\multicolumn{1}{c|}{rd} &
\multicolumn{1}{c|}{opcode} \\
\hline
5 & 7 & 5 & 3 & 5 & 7 \\
11??? & shamt & src & RORI & dest & OP-IMM \\
\end{tabular}
\end{center}

\vspace{-0.4in}
\begin{center}
\begin{tabular}{S@{}R@{}R@{}S@{}R@{}O}
\\
\instbitrange{31}{25} &
\instbitrange{24}{20} &
\instbitrange{19}{15} &
\instbitrange{14}{12} &
\instbitrange{11}{7} &
\instbitrange{6}{0} \\
\hline
\multicolumn{1}{|c|}{funct7} &
\multicolumn{1}{c|}{rs2} &
\multicolumn{1}{c|}{rs1} &
\multicolumn{1}{c|}{funct3} &
\multicolumn{1}{c|}{rd} &
\multicolumn{1}{c|}{opcode} \\
\hline
7 & 5 & 5 & 3 & 5 & 7 \\
11????? & src2 & src1 & RORW   & dest & OP-32 \\
11????? & src2 & src1 & ROLW   & dest & OP-32 \\
11????? & src2 & src1 & ROR    & dest & OP    \\
11????? & src2 & src1 & ROL    & dest & OP    \\
\end{tabular}
\end{center}


Rotate shift is implemented very similarly to the other shift
instructions. One possible way to encode it is to re-use the way that
bit 30 in the instruction encoding selects `arithmetic shift' when bit
31 is zero (signalling a logical-zero shift). We can re-use this so that
when bit 31 is set (signalling a logical-ones shift), if bit 31 is also
set, then we are doing a rotate. The following table summarizes the
behavior. The generalized reverse instructions can be encoded using the
bit pattern that would otherwise encode an ``Arithmetic Left Shift''
(which is an operation that does not exist). Likewise, the generalized zip
instruction can be encoded using the bit pattern that would otherwise
encode an ``Rotate left immediate''.

\begin{longtable}[c]{@{}lll@{}}
\caption{Rotate Encodings}\tabularnewline
\toprule
Bit 31 & Bit 30 & Meaning\tabularnewline
\midrule
\endfirsthead
\toprule
Bit 31 & Bit 30 & Meaning\tabularnewline
\midrule
\endhead
0 & 0 & Logical Shift-Zeros\tabularnewline
0 & 1 & Arithmetic Shift\tabularnewline
1 & 0 & Logical Shift-Ones\tabularnewline
1 & 1 & Rotate\tabularnewline
\bottomrule
\end{longtable}

%%%%%%%%%%%%%%%%%%%%%%%%%%%%%%%%%%%%%%%%%%%%%%%%%%%%%%%%%%%%%%%%%%%%%%%%%%%%%%%%%%%%%%%%%%%

\begin{figure}[t]
\begin{center}
\input{bextcref-printperm-ror}
\end{center}
\caption{\texttt{ror} permutation network}
\label{permnet-ror}
\end{figure}

\section{Generalized Reverse (\texttt{grev,\ grevi})}
\label{grev}

This instruction provides a single hardware instruction that can implement all
of byte-order swap, bitwise reversal, short-order-swap, word-order-swap
(RV64), nibble-order swap, bitwise reversal in a byte, etc, all from a single
hardware instruction. It takes in a single register value and an immediate that
controls which function occurs, through controlling the levels in the recursive
tree at which reversals occur.

This operation iteratively checks each bit $i$ in rs2 from $i=0$ to
$\textrm{XLEN}-1$, and if the corresponding bit is set, swaps each adjacent
pair of $2^i$ bits.

\begin{figure}[t]
\begin{center}
\input{bextcref-printperm-grev}
\end{center}
\caption{\texttt{grev} permutation network}
\label{permnet-grev}
\end{figure}

\input{bextcref-grev}

The above pattern should be intuitive to understand in order to extend
this definition in an obvious manner for RV128.

\begin{table}[h]
\begin{small}
\begin{center}
\begin{tabular}{l l p{0.5in} l l p{0.3in} l l}

\multicolumn{2}{c}{RV32} & &
\multicolumn{5}{c}{RV64} \\

\cline{1-2}
\cline{4-8}

\multicolumn{1}{c}{shamt} & Instruction & &
\multicolumn{1}{c}{shamt} & Instruction & &
\multicolumn{1}{c}{shamt} & Instruction \\

\cline{1-2}
\cline{4-5}
\cline{7-8}

00000 & ---           &   &   000000 & ---           &   &   100000 & {\tt wswap} \\
00001 & {\tt brev.p}  &   &   000001 & {\tt brev.p}  &   &   100001 & ---         \\
00010 & {\tt pswap.n} &   &   000010 & {\tt pswap.n} &   &   100010 & ---         \\
00011 & {\tt brev.n}  &   &   000011 & {\tt brev.n}  &   &   100011 & ---         \\
00100 & {\tt nswap.b} &   &   000100 & {\tt nswap.b} &   &   100100 & ---         \\
00101 & ---           &   &   000101 & ---           &   &   100101 & ---         \\
00110 & {\tt pswap.b} &   &   000110 & {\tt pswap.b} &   &   100110 & ---         \\
00111 & {\tt brev.b}  &   &   000111 & {\tt brev.b}  &   &   100111 & ---         \\

\cline{1-2}
\cline{4-5}
\cline{7-8}

01000 & {\tt bswap.h} &   &   001000 & {\tt bswap.h} &   &   101000 & ---         \\
01001 & ---           &   &   001001 & ---           &   &   101001 & ---         \\
01010 & ---           &   &   001010 & ---           &   &   101010 & ---         \\
01011 & ---           &   &   001011 & ---           &   &   101011 & ---         \\
01100 & {\tt nswap.h} &   &   001100 & {\tt nswap.h} &   &   101100 & ---         \\
01101 & ---           &   &   001101 & ---           &   &   101101 & ---         \\
01110 & {\tt pswap.h} &   &   001110 & {\tt pswap.h} &   &   101110 & ---         \\
01111 & {\tt brev.h}  &   &   001111 & {\tt brev.h}  &   &   101111 & ---         \\

\cline{1-2}
\cline{4-5}
\cline{7-8}

10000 & {\tt hswap}   &   &   010000 & {\tt hswap.w} &   &   110000 & {\tt hswap} \\
10001 & ---           &   &   010001 & ---           &   &   110001 & ---         \\
10010 & ---           &   &   010010 & ---           &   &   110010 & ---         \\
10011 & ---           &   &   010011 & ---           &   &   110011 & ---         \\
10100 & ---           &   &   010100 & ---           &   &   110100 & ---         \\
10101 & ---           &   &   010101 & ---           &   &   110101 & ---         \\
10110 & ---           &   &   010110 & ---           &   &   110110 & ---         \\
10111 & ---           &   &   010111 & ---           &   &   110111 & ---         \\

\cline{1-2}
\cline{4-5}
\cline{7-8}

11000 & {\tt bswap}   &   &   011000 & {\tt bswap.w} &   &   111000 & {\tt bswap} \\
11001 & ---           &   &   011001 & ---           &   &   111001 & ---         \\
11010 & ---           &   &   011010 & ---           &   &   111010 & ---         \\
11011 & ---           &   &   011011 & ---           &   &   111011 & ---         \\
11100 & {\tt nswap}   &   &   011100 & {\tt nswap.w} &   &   111100 & {\tt nswap} \\
11101 & ---           &   &   011101 & ---           &   &   111101 & ---         \\
11110 & {\tt pswap}   &   &   011110 & {\tt pswap.w} &   &   111110 & {\tt pswap} \\
11111 & {\tt brev}    &   &   011111 & {\tt brev.w}  &   &   111111 & {\tt brev}  \\
\end{tabular}
\end{center}
\end{small}
\caption{Pseudo-instructions for {\tt grevi} instruction}
\label{grevi-modes}
\end{table}

The {\tt grev} operation can easily be implemented using a permutation
network with $log_2(\textrm{XLEN})$ stages. Figure~\ref{permnet-ror}
shows the permutation network for {\tt ror} for reference.
Figure~\ref{permnet-grev} shows the permutation network for {\tt grev}.

\vspace{-0.4in}
\begin{center}
\begin{tabular}{S@{}R@{}R@{}S@{}R@{}O}
\\
\instbitrange{31}{25} &
\instbitrange{24}{20} &
\instbitrange{19}{15} &
\instbitrange{14}{12} &
\instbitrange{11}{7} &
\instbitrange{6}{0} \\
\hline
\multicolumn{1}{|c|}{funct7} &
\multicolumn{1}{c|}{rs2} &
\multicolumn{1}{c|}{rs1} &
\multicolumn{1}{c|}{funct3} &
\multicolumn{1}{c|}{rd} &
\multicolumn{1}{c|}{opcode} \\
\hline
7 & 5 & 5 & 3 & 5 & 7 \\
??????? & src2 & src1 & GREV   & dest & OP    \\
??????? & src2 & src1 & GREVW  & dest & OP-32 \\
\end{tabular}
\end{center}

\vspace{-0.4in}
\begin{center}
\begin{tabular}{R@{}R@{}R@{}S@{}R@{}O}
\\
\instbitrange{31}{27} &
\instbitrange{26}{20} &
\instbitrange{19}{15} &
\instbitrange{14}{12} &
\instbitrange{11}{7} &
\instbitrange{6}{0} \\
\hline
\multicolumn{1}{|c|}{imm[11:7]} &
\multicolumn{1}{c|}{imm[6:0]} &
\multicolumn{1}{c|}{rs1} &
\multicolumn{1}{c|}{funct3} &
\multicolumn{1}{c|}{rd} &
\multicolumn{1}{c|}{opcode} \\
\hline
5 & 7 & 5 & 3 & 5 & 7 \\
????? & mode & src & GREVI  & dest & OP-IMM \\
\end{tabular}
\end{center}

\vspace{-0.4in}
\begin{center}
\begin{tabular}{R@{}R@{}R@{}S@{}R@{}O}
\\
\instbitrange{31}{25} &
\instbitrange{24}{20} &
\instbitrange{19}{15} &
\instbitrange{14}{12} &
\instbitrange{11}{7} &
\instbitrange{6}{0} \\
\hline
\multicolumn{1}{|c|}{imm[11:5]} &
\multicolumn{1}{c|}{imm[4:0]} &
\multicolumn{1}{c|}{rs1} &
\multicolumn{1}{c|}{funct3} &
\multicolumn{1}{c|}{rd} &
\multicolumn{1}{c|}{opcode} \\
\hline
7 & 5 & 5 & 3 & 5 & 7 \\
??????? & mode & src & GREVIW  & dest & OP-IMM-32 \\
\end{tabular}
\end{center}


\texttt{grev} is encoded as standard R-type opcode and \texttt{grevi} is
encoded as standard I-type opcode. \texttt{grev} and \texttt{grevi} can
use the instruction encoding for ``arithmetic shift left''.

% \subsection{References}
%
% Hackers Delight, Chapter 7.1, ``Generalized Bit Reversal'' in
%
% https://books.google.com/books?id=iBNKMspIlqEC\&lpg=PP1\&pg=RA1-SL20-PA2\#v=onepage\&q\&f=false
%
% http://hackersdelight.org/

%%%%%%%%%%%%%%%%%%%%%%%%%%%%%%%%%%%%%%%%%%%%%%%%%%%%%%%%%%%%%%%%%%%%%%%%%%%%%%%%%%%%%%%%%%%

\section{Generalized zip/unzip (\texttt{gzip})}
\label{gzip}

{\tt gzip} is the third bit permutation instruction in XBitmanip, after {\tt
rori} and {\tt grevi}. It implements a generalization of the operation commonly
known as perfect outer shuffle and its inverse (shuffle/unshuffle), also known
as zip/unzip or interlace/uninterlace.

This set of operation is best understood as operation on the bit indices. For
reference, rotate shift adds {\tt shamt} to the bit index (modulo XLEN), and
generalized reversed performs an XOR between the bit index and {\tt shamt}.
Generalized zip/unzip performs a roate left shift (zip) or rotate right shift (unzip)
on a continous range of bits in the bit index. Every arbitrary permutation
of the bit index bits can be achieved by applying {\tt gzip} repeatedly.

The {\tt gzip} instruction uses an I-type encoding similar to {\tt grevi}.
There are XLEN different generalized zip operations, some of which are reserved
because they are no-ops, or equivalent to other modes, or encode for obscure
combinations of other modes. The bit pattern for the non-reserved modes
match the regular expression {\tt /\^{}0*(10+|11+0*[01])\$/}, or in words:
Bits {\tt mode[LOG2\_XLEN-1:1]} must only contain one continous range of
1 bits, and if only one bit in {\tt mode[LOG2\_XLEN-1:1]} is set then
{\tt mode[0]} must be cleared. See Tables~\ref{gzip32-modes}~and~\ref{gzip64-modes}.

Reserving modes that encode for ``obscure combinations of other modes'' can help
implementations that use different base permutations (or completely different
mechanisms) to implement the {\tt gzip} instruction. The reserved modes can be
used to encode unary functions such as {\tt ctz}, {\tt clz}, and {\tt pcnt}.

\begin{table}[h]
\begin{small}
\begin{center}
\begin{tabular}{c l l}
      mode    & Bit index rotations              & Pseudo-Instruction     \\

\hline

\sout{0000 0} & no-op                            & {\it reserved}         \\
\sout{0000 1} & no-op                            & {\it reserved}         \\
      0001 0  & {\tt i[1] -> i[0]}               & {\tt zip.n, unzip.n}   \\
\sout{0001 1} & {\it equivalent to 0001 0}       & {\it reserved}         \\
      0010 0  & {\tt i[2] -> i[1]}               & {\tt zip2.b, unzip2.b} \\
\sout{0010 1} & {\it equivalent to 0010 0}       & {\it reserved}         \\
      0011 0  & {\tt i[2] -> i[0]}               & {\tt zip.b}            \\
      0011 1  & {\tt i[2] <- i[0]}               & {\tt unzip.b}          \\

\hline

      0100 0  & {\tt i[3] -> i[2]}               & {\tt zip4.h, unzip4.h} \\
\sout{0100 1} & {\it equivalent to 0100 0}       & {\it reserved}         \\
\sout{0101 0} & {\tt i[3] -> i[2], i[1] -> i[0]} & {\it reserved}         \\
\sout{0101 1} & {\it equivalent to 0101 0}       & {\it reserved}         \\
      0110 0  & {\tt i[3] -> i[1]}               & {\tt zip2.h}           \\
      0110 1  & {\tt i[3] <- i[1]}               & {\tt unzip2.h}         \\
      0111 0  & {\tt i[3] -> i[0]}               & {\tt zip.h}            \\
      0111 1  & {\tt i[3] <- i[0]}               & {\tt unzip.h}          \\

\hline

      1000 0  & {\tt i[4] -> i[3]}               & {\tt zip8, unzip8}     \\
\sout{1000 1} & {\it equivalent to 1000 0}       & {\it reserved}         \\
\sout{1001 0} & {\tt i[4] -> i[3], i[1] -> i[0]} & {\it reserved}         \\
\sout{1001 1} & {\it equivalent to 1001 0}       & {\it reserved}         \\
\sout{1010 0} & {\tt i[4] -> i[3], i[2] -> i[1]} & {\it reserved}         \\
\sout{1010 1} & {\it equivalent to 1010 0}       & {\it reserved}         \\
\sout{1011 0} & {\tt i[4] -> i[3], i[2] -> i[0]} & {\it reserved}         \\
\sout{1011 1} & {\tt i[4] <- i[3], i[2] <- i[0]} & {\it reserved}         \\

\hline

      1100 0  & {\tt i[4] -> i[2]}               & {\tt zip4}             \\
      1100 1  & {\tt i[4] <- i[2]}               & {\tt unzip4}           \\
\sout{1101 0} & {\tt i[4] -> i[2], i[1] -> i[0]} & {\it reserved}         \\
\sout{1101 1} & {\tt i[4] <- i[2], i[1] <- i[0]} & {\it reserved}         \\
      1110 0  & {\tt i[4] -> i[1]}               & {\tt zip2}             \\
      1110 1  & {\tt i[4] <- i[1]}               & {\tt unzip2}           \\
      1111 0  & {\tt i[4] -> i[0]}               & {\tt zip}              \\
      1111 1  & {\tt i[4] <- i[0]}               & {\tt unzip}            \\
\end{tabular}
\end{center}
\end{small}
\caption{RV32 modes and pseudo-instructions for {\tt gzip} instruction}
\label{gzip32-modes}
\end{table}

\begin{table}[h]
\begin{small}
\begin{center}
\begin{tabular}{c l p{1in} c l}
      mode     & Pseudo-Instruction       &   &         mode     & Pseudo-Instruction      \\

\cline{1-2}
\cline{4-5}

\sout{00000 0} & {\it reserved}           &   &   \sout{10000 0} & {\tt zip16, unzip16}    \\
\sout{00000 1} & {\it reserved}           &   &   \sout{10000 1} & {\it reserved}          \\
      00001 0  & {\tt zip.n, unzip.n}     &   &   \sout{10001 0} & {\it reserved}          \\
\sout{00001 1} & {\it reserved}           &   &   \sout{10001 1} & {\it reserved}          \\
      00010 0  & {\tt zip2.b, unzip2.b}   &   &   \sout{10010 0} & {\it reserved}          \\
\sout{00010 1} & {\it reserved}           &   &   \sout{10010 1} & {\it reserved}          \\
      00011 0  & {\tt zip.b}              &   &   \sout{10011 0} & {\it reserved}          \\
      00011 1  & {\tt unzip.b}            &   &   \sout{10011 1} & {\it reserved}          \\

\cline{1-2}
\cline{4-5}

      00100 0  & {\tt zip4.h, unzip4.h}   &   &   \sout{10100 0} & {\it reserved}          \\
\sout{00100 1} & {\it reserved}           &   &   \sout{10100 1} & {\it reserved}          \\
\sout{00101 0} & {\it reserved}           &   &   \sout{10101 0} & {\it reserved}          \\
\sout{00101 1} & {\it reserved}           &   &   \sout{10101 1} & {\it reserved}          \\
      00110 0  & {\tt zip2.h}             &   &   \sout{10110 0} & {\it reserved}          \\
      00110 1  & {\tt unzip2.h}           &   &   \sout{10110 1} & {\it reserved}          \\
      00111 0  & {\tt zip.h}              &   &   \sout{10111 0} & {\it reserved}          \\
      00111 1  & {\tt unzip.h}            &   &   \sout{10111 1} & {\it reserved}          \\

\cline{1-2}
\cline{4-5}

      01000 0  & {\tt zip8.w, unzip8.w}   &   &         11000 0  & {\tt zip8, unzip8}      \\
\sout{01000 1} & {\it reserved}           &   &   \sout{11000 1} & {\it reserved}          \\
\sout{01001 0} & {\it reserved}           &   &   \sout{11001 0} & {\it reserved}          \\
\sout{01001 1} & {\it reserved}           &   &   \sout{11001 1} & {\it reserved}          \\
\sout{01010 0} & {\it reserved}           &   &   \sout{11010 0} & {\it reserved}          \\
\sout{01010 1} & {\it reserved}           &   &   \sout{11010 1} & {\it reserved}          \\
\sout{01011 0} & {\it reserved}           &   &   \sout{11011 0} & {\it reserved}          \\
\sout{01011 1} & {\it reserved}           &   &   \sout{11011 1} & {\it reserved}          \\

\cline{1-2}
\cline{4-5}

      01100 0  & {\tt zip4.w}             &   &         11100 0  & {\tt zip4}              \\
      01100 1  & {\tt unzip4.w}           &   &         11100 1  & {\tt unzip4}            \\
\sout{01101 0} & {\it reserved}           &   &   \sout{11101 0} & {\it reserved}          \\
\sout{01101 1} & {\it reserved}           &   &   \sout{11101 1} & {\it reserved}          \\
      01110 0  & {\tt zip2.w}             &   &         11110 0  & {\tt zip2}              \\
      01110 1  & {\tt unzip2.w}           &   &         11110 1  & {\tt unzip2}            \\
      01111 0  & {\tt zip.w}              &   &         11111 0  & {\tt zip}               \\
      01111 1  & {\tt unzip.w}            &   &         11111 1  & {\tt unzip}             \\
\end{tabular}
\end{center}
\end{small}
\caption{RV64 modes and pseudo-instructions for {\tt gzip} instruction}
\label{gzip64-modes}
\end{table}

\begin{figure}[t]
\begin{center}
\input{bextcref-printperm-gzip-noflip}
\end{center}
\caption{\texttt{gzip} permutation network without ``flip'' stages}
\label{permnet-gzip-noflip}
\end{figure}

Like GREV and rotate shift, the {\tt gzip} instruction can be implemented using a short
sequence of atomic permutations, that are enabled or disabled by the mode (shamt)
bits. But zip has one stage fewer than GREV and the LSB bit of mode controls the order
in which the stages are applied:

\input{bextcref-gzip32}

Alternatively {\tt gzip} can be implemented in a single network with one more
stage than GREV, with the additional first and last stage executing a
permutation that effectively reverses the order of the inner stages. However,
since the inner stages only mux half of the bits in the word each, a hardware
implementation using this additional ``flip'' stages might actually be more
expensive than simply creating two networks.

\input{bextcref-gzip32-alt}

Figure~\ref{permnet-gzip-flip} shows the {\tt gzip} permutation network with
``flip'' stages and Figure~\ref{permnet-gzip-noflip} shows the {\tt gzip}
permutation network without ``flip'' stages.

\begin{figure}[t]
\begin{center}
\input{bextcref-printperm-gzip-flip}
\end{center}
\caption{\texttt{gzip} permutation network with ``flip'' stages}
\label{permnet-gzip-flip}
\end{figure}

The \texttt{zip} instruction with the upper half of its input cleared performs
the commonly needed ``fan-out'' operation. (Equivalent to {\tt bdep} with a
0x55555555 mask.) The \texttt{zip} instruction applied twice fans out the bits
in the lower quarter of the input word by a spacing of 4 bits.

For example, the following code calculates the bitwise prefix sum of the bits
in the lower byte of a 32 bit word on RV32:

\begin{verbatim}
  andi a0, a0, 0xff
  zip a0, a0
  zip a0, a0
  slli a1, a0, 4
  c.add a0, a1
  slli a1, a0, 8
  c.add a0, a1
  slli a1, a0, 16
  c.add a0, a1
\end{verbatim}

The final prefix sum is stored in the 8 nibbles of the {\tt a0} output word.

\vspace{-0.3in}
\begin{center}
\begin{tabular}{R@{}R@{}R@{}S@{}R@{}O}
\\
\instbitrange{31}{27} &
\instbitrange{26}{20} &
\instbitrange{19}{15} &
\instbitrange{14}{12} &
\instbitrange{11}{7} &
\instbitrange{6}{0} \\
\hline
\multicolumn{1}{|c|}{imm[11:7]} &
\multicolumn{1}{c|}{imm[6:0]} &
\multicolumn{1}{c|}{rs1} &
\multicolumn{1}{c|}{funct3} &
\multicolumn{1}{c|}{rd} &
\multicolumn{1}{c|}{opcode} \\
\hline
5 & 7 & 5 & 3 & 5 & 7 \\
????? & mode & src & GZIP & dest & OP-IMM \\
\end{tabular}
\end{center}

% \vspace{-0.4in}
% \begin{center}
% \begin{tabular}{R@{}R@{}R@{}S@{}R@{}O}
% \\
% \instbitrange{31}{25} &
% \instbitrange{24}{20} &
% \instbitrange{19}{15} &
% \instbitrange{14}{12} &
% \instbitrange{11}{7} &
% \instbitrange{6}{0} \\
% \hline
% \multicolumn{1}{|c|}{imm[11:5]} &
% \multicolumn{1}{c|}{imm[4:0]} &
% \multicolumn{1}{c|}{rs1} &
% \multicolumn{1}{c|}{funct3} &
% \multicolumn{1}{c|}{rd} &
% \multicolumn{1}{c|}{opcode} \\
% \hline
% 7 & 5 & 5 & 3 & 5 & 7 \\
% ??????? & mode & src & GZIPW  & dest & OP-IMM-32 \\
% \end{tabular}
% \end{center}


There is no R-type instruction for {\tt gzip}. It is an I-type only instruction.
\texttt{gzip} can use the instruction encoding for ``rotate left immediate''.

%%%%%%%%%%%%%%%%%%%%%%%%%%%%%%%%%%%%%%%%%%%%%%%%%%%%%%%%%%%%%%%%%%%%%%%%%%%%%%%%%%%%%%%%%%%

\section{Bit Extract/Deposit (\texttt{bext,\ bdep})}

This instructions implement the generic bit extract and bit deposit functions.
This operation is also referred to as bit gather/scatter, bit pack/unpack,
parallel extract/deposit, compress/expand, or right\_compress/right\_expand.

\texttt{bext} collects LSB justified bits to rd from rs1 using extract mask in rs2.

\texttt{bdep} writes LSB justified bits from rs1 to rd using deposit mask in rs2.

\input{bextcref-bext}

Implementations may choose to use smaller multi-cycle implementations of
\texttt{bext} and \texttt{bdep}, or even emulate the instructions in software.

Even though multi-cycle \texttt{bext} and \texttt{bdep} often are not fast
enough to outperform algortihms that use sequences of shifts and bit masks,
dedicated instructions for those operations can still be of great advantage in
cases where the mask argument is not constant.

For example, the following code efficiently calculates the index of the tenth
set bit in {\tt a0} using \texttt{bdep}:

\begin{verbatim}
  li a1, 0x00000200
  bdep a0, a1, a0
  ctz a0, a0
\end{verbatim}

For cases with a constant mask an optimizing compiler would decide when to use
\texttt{bext} or \texttt{bdep} based on the optimization profile for the
concrete processor it is optimizing for. This is similar to the decision
whether to use MUL or DIV with a constant, or to perform the same operation
using a longer sequence of much simpler operations.

\vspace{-0.3in}
\begin{center}
\begin{tabular}{S@{}R@{}R@{}S@{}R@{}O}
\\
\instbitrange{31}{25} &
\instbitrange{24}{20} &
\instbitrange{19}{15} &
\instbitrange{14}{12} &
\instbitrange{11}{7} &
\instbitrange{6}{0} \\
\hline
\multicolumn{1}{|c|}{funct7} &
\multicolumn{1}{c|}{rs2} &
\multicolumn{1}{c|}{rs1} &
\multicolumn{1}{c|}{funct3} &
\multicolumn{1}{c|}{rd} &
\multicolumn{1}{c|}{opcode} \\
\hline
7 & 5 & 5 & 3 & 5 & 7 \\
??????? & src2 & src1 & BEXT   & dest & OP       \\
??????? & src2 & src1 & BDEP   & dest & OP       \\
??????? & src2 & src1 & BEXTW  & dest & OP-32    \\
??????? & src2 & src1 & BDEPW  & dest & OP-32    \\
\end{tabular}
\end{center}


% \subsection{Justification}
%
% http://svn.clifford.at/handicraft/2017/permsyn/
%
% \subsection{References}
%
% http://programming.sirrida.de/bit\_perm.html\#gather\_scatter
%
% Hackers Delight, Chapter 7.1, ``Compress, Generalized Extract'' in
%
% https://books.google.com/books?id=iBNKMspIlqEC\&lpg=PP1\&pg=RA1-SL20-PA2\#v=onepage\&q\&f=false
%
% http://hackersdelight.org/
%
% https://github.com/cliffordwolf/bextdep
%
% http://palms.ee.princeton.edu/system/files/Hilewitz_JSPS_08.pdf

%%%%%%%%%%%%%%%%%%%%%%%%%%%%%%%%%%%%%%%%%%%%%%%%%%%%%%%%%%%%%%%%%%%%%%%%%%%%%%%%%%%%%%%%%%%

\section{Compressed instructions (\texttt{c.not,\ c.neg,\ c.brev})}

The RISC-V ISA has no dedicated instructions for bitwise inverse (\texttt{not})
and arithmetic inverse (\texttt{neg}). Instead \texttt{not} is implemented as
\texttt{xori\ rd,\ rs,\ -1} and \texttt{neg} is implemented as
\texttt{sub\ rd,\ x0,\ rs}.

In bitmanipulation code \texttt{not} and \texttt{neg} are very common operations. But
there are no compressed encodings for those operations because there is no \texttt{c.xori}
instruction and \texttt{c.sub} can not operate on \texttt{x0}.

Many bit manipulation operations that have dedicated opcodes in other ISAs
must be constructed from smaller atoms in RISC-V XBitmanip code. But
implementations might choose to implement them in a single micro-op using
macro-op-fusion. For this it can be helpful when the fused sequences are short.
\texttt{not} and \texttt{neg} are good candidates for macro-op-fusion, so
it can be helpful to have compressed opcodes for them.

Likewise \texttt{brev} (an alias for \texttt{grevi\ rd,\ rs,\ -1}, i.e. bitwise
reversal) is also a very common atom for building bit manipulation operations. So it
is helpful to have a compressed opcode for this instruction as well.

The compressed instructions \texttt{c.not,\ c.neg,\ c.brev} must be supported by
all implementations that support the C extension and XBitmanip.

\begin{table}[h]
\begin{small}
\begin{center}
\begin{tabular}{p{0in}p{0.05in}p{0.05in}p{0.05in}p{0.05in}p{0.05in}p{0.05in}p{0.05in}p{0.05in}p{0.05in}p{0.05in}p{0.05in}p{0.05in}p{0.05in}p{0.05in}p{0.05in}p{0.05in}l}
& & & & & & & & & & \\
                      &
\instbit{15} &
\instbit{14} &
\instbit{13} &
\multicolumn{1}{c}{\instbit{12}} &
\instbit{11} &
\instbit{10} &
\instbit{9} &
\instbit{8} &
\instbit{7} &
\instbit{6} &
\multicolumn{1}{c}{\instbit{5}} &
\instbit{4} &
\instbit{3} &
\instbit{2} &
\instbit{1} &
\instbit{0} \\
\cline{2-17}

&
\multicolumn{3}{|c|}{011} &
\multicolumn{1}{c|}{nzimm[17]} &
\multicolumn{5}{c|}{rd$\neq$$\{0,2\}$} &
\multicolumn{5}{c|}{nzimm[16:12]} &
\multicolumn{2}{c|}{01} & C.LUI {\em \tiny (\sout{RES, nzimm=0}; HINT, rd=0)} \\
\cline{2-17}

&
\multicolumn{3}{|c|}{011} &
\multicolumn{1}{c|}{0} &
\multicolumn{2}{c|}{00} &
\multicolumn{3}{c|}{rs2$'$/rd$'$} &
\multicolumn{5}{c|}{0} &
\multicolumn{2}{c|}{01} & C.NEG \\
\cline{2-17}

&
\multicolumn{3}{|c|}{011} &
\multicolumn{1}{c|}{0} &
\multicolumn{2}{c|}{01} &
\multicolumn{3}{c|}{rs1$'$/rd$'$} &
\multicolumn{5}{c|}{0} &
\multicolumn{2}{c|}{01} & C.NOT \\
\cline{2-17}

&
\multicolumn{3}{|c|}{011} &
\multicolumn{1}{c|}{0} &
\multicolumn{2}{c|}{10} &
\multicolumn{3}{c|}{rs1$'$/rd$'$} &
\multicolumn{5}{c|}{0} &
\multicolumn{2}{c|}{01} & C.CLZ \\
\cline{2-17}

&
\multicolumn{3}{|c|}{011} &
\multicolumn{1}{c|}{0} &
\multicolumn{2}{c|}{11} &
\multicolumn{3}{c|}{rs1$'$/rd$'$} &
\multicolumn{5}{c|}{0} &
\multicolumn{2}{c|}{01} & C.PCNT \\
\cline{2-17}
  

\end{tabular}
\end{center}
\end{small}
\end{table}


This three instructions fit nicely in the reserved space in C.LUI/C.ADDI16SP.
They only occupy $0.1\%$ of the $\approx15.6$ bits wide RVC encoding space.

%%%%%%%%%%%%%%%%%%%%%%%%%%%%%%%%%%%%%%%%%%%%%%%%%%%%%%%%%%%%%%%%%%%%%%%%%%%%%%%%%%%%%%%%%%%

\section{Bit-field extract pseudo instruction ({\tt bfext})}

Extract the continous bit field starting at {\tt pos} with length {\tt len}
from {\tt rs} (with $\texttt{pos}>0$, $\texttt{len}>0$, and
$\texttt{pos}+\texttt{len}\le\textrm{XLEN}$).

\begin{verbatim}
  bfext rd, rs, pos, len   ->   slli rd, rs, (XLEN-pos-len)
                                srli rd, rd, (XLEN-len)
\end{verbatim}

If possible, an implementation should fuse this two shift operations into a single
macro-op. (Some implementors have raised concerns about the lack of a dedicated
bit field extract instruction with large immediate, especially for implementations
that can not fuse instructions into macro-ops and/or implementations that do
not support compressed instructions. See Chapter~\ref{bfxp} for a brief discussion
of why no such instruction is part of XBitmanip, and a short separate extension
proposal that adds such an instruction.)
