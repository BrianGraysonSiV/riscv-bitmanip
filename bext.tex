\chapter{RISC-V XBitmanip Extension}

In the proposals provided in this section, the C code examples are for
illustration purposes. They are not optimal implementations, but are
intended to specify the desired functionality.

The sections on encodings are mere placeholders.

%%%%%%%%%%%%%%%%%%%%%%%%%%%%%%%%%%%%%%%%%%%%%%%%%%%%%%%%%%%%%%%%%%%%%%%%%%%%%%%%%%%%%%%%%%%

\section{Count Leading Zeros (\texttt{clz})}

This operation counts the number of 0 bits before the first 1 bit
(counting from the most significant bit) in the source register. This is
related to the ``integer logarithm''. It takes a single register as
input and operates on the entire register. If the input is 0, the output is
XLEN. If the input is \textasciitilde{}0, the output is 0.

\begin{verbatim}
uint_xlen_t clz(uint_xlen_t rs1)
{
    for (int count = 0; count < XLEN; count++)
        if ((rs1 << count) >> (XLEN-1))
            return count;
    return XLEN;
}
\end{verbatim}

\vspace{-0.3in}
\begin{center}
\begin{tabular}{M@{}R@{}S@{}R@{}O}
\\
\instbitrange{31}{20} &
\instbitrange{19}{15} &
\instbitrange{14}{12} &
\instbitrange{11}{7} &
\instbitrange{6}{0} \\
\hline
\multicolumn{1}{|c|}{imm[11:0]} &
\multicolumn{1}{c|}{rs1} &
\multicolumn{1}{c|}{funct3} &
\multicolumn{1}{c|}{rd} &
\multicolumn{1}{c|}{opcode} \\
\hline
12 & 5 & 3 & 5 & 7 \\
???????????? & src & CLZ  & dest & OP-IMM \\
???????????? & src & CTZ  & dest & OP-IMM \\
???????????? & src & CLZW  & dest & OP-IMM-32 \\
???????????? & src & CTZW  & dest & OP-IMM-32 \\
\end{tabular}
\end{center}


One possible encoding for \texttt{clz} is as a standard I-type opcode somewhere
in the brownfield surrounding the shift-immediate instructions.

% \subsection{Related Pseudo-instructions}
%
% The common operation of `finding the index of the highest bit set' (also called
% \texttt{ilog2} or \texttt{bsr} or \texttt{find\ last\ set}) is computed as
% \texttt{XLEN-clz(rs1)}.
%
% Counting trailing zeros is easily implemented by combining \texttt{clz} with
% the \texttt{grevi} instruction.
%
% \subsection{Criteria}
%
% The criteria evaluation is shown below:
%
% \begin{itemize}
% \item
%   These operations all fit easily into the RISCV instruction encoding
%   and philosophy.
% \item
%   The hardware to implement them is fairly simple, and can be done in a
%   logarithmic number of stages in parallel.
% \item
%   They all have current compiler and standard library support, and are
%   standardized intrinsics in many of those.
% \item
%   The threshold criteria gets a little complicated: Every one of these
%   operations can be implemented in 2-3 instructions in terms of
%   each-other, so we only need one of them in hardware, and the rest can
%   be pseudo-instructions.
% \end{itemize}
%
% However, implementing any of these instructions can also be done with a
% combination of \texttt{ctz} and a bitwise reverse (\texttt{brev}) in 1-2
% instructions. Similarly, \texttt{ctz} can be implemented in terms of
% these functions with a \texttt{brev} in 1-2 instructions.
%
% Without \texttt{brev} AND \texttt{ctz}, however, there is no simple way
% to construct this without a lot of instructions. The algorithm above in
% C is a nearly asymptotically optimal implementation. It seems to pass
% the threshold criteria in this case.
%
% \paragraph{Benchmarks and Applications}
%
% TBD
%
% \subsection{References}
%
% https://en.wikipedia.org/wiki/Find\_first\_set\#CLZ
%
% https://fgiesen.wordpress.com/2013/10/18/bit-scanning-equivalencies/

%%%%%%%%%%%%%%%%%%%%%%%%%%%%%%%%%%%%%%%%%%%%%%%%%%%%%%%%%%%%%%%%%%%%%%%%%%%%%%%%%%%%%%%%%%%

\section{Count Bits Set (\texttt{pcnt})}

The purpose of this instruction is to compute the number of 1 bits in a
register. It takes a single register as input and operates on the entire
register.

This operation counts the total number of set bits in the register.

\begin{verbatim}
uint_xlen_t pcnt(uint_xlen_t rs1)
{
    int count = 0;
    for(int index=0; index < XLEN; index++)
        count += (rs1 >> index) & 1;
    return count;
}
\end{verbatim}

\vspace{-0.4in}
\begin{center}
\begin{tabular}{M@{}R@{}S@{}R@{}O}
\\
\instbitrange{31}{20} &
\instbitrange{19}{15} &
\instbitrange{14}{12} &
\instbitrange{11}{7} &
\instbitrange{6}{0} \\
\hline
\multicolumn{1}{|c|}{imm[11:0]} &
\multicolumn{1}{c|}{rs1} &
\multicolumn{1}{c|}{funct3} &
\multicolumn{1}{c|}{rd} &
\multicolumn{1}{c|}{opcode} \\
\hline
12 & 5 & 3 & 5 & 7 \\
???????????? & src & PCNTW  & dest & OP-IMM-32 \\
???????????? & src & PCNT  & dest & OP-IMM \\
\end{tabular}
\end{center}


One possible encoding for \texttt{pcnt} is as a standard I-type opcode somewhere
in the brownfield surrounding the shift-immediate instructions.

% Implementation Commentary
%
% \texttt{pcnt} can be implemented in parallel, similarly to an adder.
%
% \subsection{Related Pseudo-Instructions}
%
% \begin{verbatim}
% //odd parity
% parity rOut,rIn:
%     pcnt rPopulation,rIn
%     andi rOut,rPopulation,0x1
% \end{verbatim}
%
% \subsection{Extensions: \texttt{pcnt.w} (R64I+), \texttt{pcnt.d} (R128I+)}
%
% On RV64B+, the appropriate versions of pcnt that operate on sub-portions
% of the register are defined. On these platforms, the input source is set
% to zero for the high portions of the register. See \texttt{addi.w} and
% \texttt{addi.d} in the R64I/R128I specifications.
%
% \subsection{Justification}
%
% This operation is one of the `core' bitwise instructions that is nearly
% universal on other ISAs. It is supported as an intrinsic on GCC, LLVM,
% and MSVC, and is supported in hardware on a plethora of platforms.
%
% \subsection{References}
%
% https://en.wikipedia.org/wiki/Hamming\_weight
%
% https://graphics.stanford.edu/\textasciitilde{}seander/bithacks.html

%%%%%%%%%%%%%%%%%%%%%%%%%%%%%%%%%%%%%%%%%%%%%%%%%%%%%%%%%%%%%%%%%%%%%%%%%%%%%%%%%%%%%%%%%%%

\section{Generalized Reverse (\texttt{grev,\ grevi})}

The purpose of this instruction is to provide a single hardware
instruction that can implement all of byte-order swap, bitwise reversal,
short-order-swap, word-order-swap (RV64I), nibble-order swap, bitwise
reversal in a byte, etc, all from a single hardware instruction. It
takes in a single register value and an immediate that controls which
function occurs, through controlling the levels in the recursive tree at
which reversals occur.

This operation iteratively checks each bit immed\_i from i=0 to XLEN-1,
in XLEN stages, and if the corresponding bit of the `function\_select'
immediate is true for the current stage, swaps each adjacent pair of
2\^{}i bits in the register.

\texttt{grevi} `butterfly' implementation in C on various architectures

\begin{verbatim}
uint32_t grevi32(uint32_t rs1,int12_t immed)
{
    uint32_t x = rs1;
    uint5_t k = (uint12_t)immed;
    if (k &  1) x = ((x & 0x55555555) << 1) | ((x & 0xAAAAAAAA) >> 1);
    if (k &  2) x = ((x & 0x33333333) << 2) | ((x & 0xCCCCCCCC) >> 2);
    if (k &  4) x = ((x & 0x0F0F0F0F) << 4) | ((x & 0xF0F0F0F0) >> 4);
    if (k &  8) x = ((x & 0x00FF00FF) << 8) | ((x & 0xFF00FF00) >> 8);
    if (k & 16) x = ((x & 0x0000FFFF) <<16) | ((x & 0xFFFF0000) >> 16);
    return x;
}

uint64_t grevi64(uint32_t rs1,int12_t immed)
{
    uint64_t x = rs1;
    uint6_t k = (uint12_t)immed;
    if (k &  1) x = ((x & 0x5555555555555555) << 1) | ((x & 0xAAAAAAAAAAAAAAAA) >> 1);
    if (k &  2) x = ((x & 0x3333333333333333) << 2) | ((x & 0xCCCCCCCCCCCCCCCC) >> 2);
    if (k &  4) x = ((x & 0x0F0F0F0F0F0F0F0F) << 4) | ((x & 0xF0F0F0F0F0F0F0F0) >> 4);
    if (k &  8) x = ((x & 0x00FF00FF00FF00FF) << 8) | ((x & 0xFF00FF00FF00FF00) >> 8);
    if (k & 16) x = ((x & 0x0000FFFF0000FFFF) <<16) | ((x & 0xFFFF0000FFFF0000) >> 16);
    if (k & 32) x = ((x & 0x00000000FFFFFFFF) <<32) | ((x & 0xFFFFFFFF00000000) >> 32);
    return x;
}
\end{verbatim}

The above pattern should be intuitive to understand in order to extend
this definition in an obvious manner for RV128+.

\vspace{-0.4in}
\begin{center}
\begin{tabular}{S@{}R@{}R@{}S@{}R@{}O}
\\
\instbitrange{31}{25} &
\instbitrange{24}{20} &
\instbitrange{19}{15} &
\instbitrange{14}{12} &
\instbitrange{11}{7} &
\instbitrange{6}{0} \\
\hline
\multicolumn{1}{|c|}{funct7} &
\multicolumn{1}{c|}{rs2} &
\multicolumn{1}{c|}{rs1} &
\multicolumn{1}{c|}{funct3} &
\multicolumn{1}{c|}{rd} &
\multicolumn{1}{c|}{opcode} \\
\hline
7 & 5 & 5 & 3 & 5 & 7 \\
??????? & src2 & src1 & GREV   & dest & OP    \\
??????? & src2 & src1 & GREVW  & dest & OP-32 \\
\end{tabular}
\end{center}

\vspace{-0.4in}
\begin{center}
\begin{tabular}{R@{}R@{}R@{}S@{}R@{}O}
\\
\instbitrange{31}{27} &
\instbitrange{26}{20} &
\instbitrange{19}{15} &
\instbitrange{14}{12} &
\instbitrange{11}{7} &
\instbitrange{6}{0} \\
\hline
\multicolumn{1}{|c|}{imm[11:7]} &
\multicolumn{1}{c|}{imm[6:0]} &
\multicolumn{1}{c|}{rs1} &
\multicolumn{1}{c|}{funct3} &
\multicolumn{1}{c|}{rd} &
\multicolumn{1}{c|}{opcode} \\
\hline
5 & 7 & 5 & 3 & 5 & 7 \\
????? & mode & src & GREVI  & dest & OP-IMM \\
\end{tabular}
\end{center}

\vspace{-0.4in}
\begin{center}
\begin{tabular}{R@{}R@{}R@{}S@{}R@{}O}
\\
\instbitrange{31}{25} &
\instbitrange{24}{20} &
\instbitrange{19}{15} &
\instbitrange{14}{12} &
\instbitrange{11}{7} &
\instbitrange{6}{0} \\
\hline
\multicolumn{1}{|c|}{imm[11:5]} &
\multicolumn{1}{c|}{imm[4:0]} &
\multicolumn{1}{c|}{rs1} &
\multicolumn{1}{c|}{funct3} &
\multicolumn{1}{c|}{rd} &
\multicolumn{1}{c|}{opcode} \\
\hline
7 & 5 & 5 & 3 & 5 & 7 \\
??????? & mode & src & GREVIW  & dest & OP-IMM-32 \\
\end{tabular}
\end{center}


\texttt{grev} is encoded as standard R-type opcode and \texttt{grevi} is
encoded as standard I-type opcode.

% \subsection{Related Pseudo-Instructions}
%
% Related Pseudoinstruction List
%
% \begin{verbatim}
% // reverse the bits in a register (this works on all platforms)
% rev rOut,rIn:
%     grevi rOut,rIn,-1
%
% // reverse the bits in each byte, but leave the bytes in the same order.
% rev.b rOut,rIn:
%     grevi rOut,rIn,7
%
% // reverse the bits in each halfword, but leave the halfwords in the same order
% rev.h rOut,rIn:
%     grevi rOut,rIn,15
%
% // reverse the bits in each word, but leave the words in the same order (e.g. RV64IB+)
% rev.w rOut,rIn:
%     grevi rOut,rIn,31
%
% // reverse the bits in each dword, but leave the dwords in the same order (e.g. RV128IB+)
% rev.d rOut,rIn:
%     grevi rOut,rIn,63
%
% // reverse the byte order of an entire register
% bswap rOut,rIn:
%     grevi rOut,rIn,-8
%
% // reverse the byte order of each halfword in the register, but leave them in the same
% // order (e.g, on RV32IB, this reverses the byte order of the high and lo halfwords)
% bswap.h rOut,rIn:
%     grevi rOut,rIn,8
%
% // reverse the byte order each word in the register (e.g, on RV64IB/RV128IB, this
% // reverses the byte order of each of the 2/4 words)
% bswap.w rOut,rIn:
%     grevi rOut,rIn,24
%
% // reverse the byte order of each dword in the register (e.g, on RV128IB, reverse the
% // byte order of the high and lo words)
% bswap.d rOut,rIn:
%     grevi rOut,rIn,56
%
% // reverse the halfword order of an entire register
% hswap rOut,rIn:
%     grevi rOut,rIn,-16
%
% // reverse the halfword order of each word in the register
% hswap.w rOut,rIn:
%     grevi rOut,rIn,16
%
% // reverse the halfword order of each dword in the register (e.g. RV128IB+, there are 2
% // dwords)
% hswap.d rOut,rIn:
%     grevi rOut,rIn,48
%
% // reverse the word order of an entire register (only meaningful on RV64IB+)
% wswap rOut,rIn:
%     grevi rOut,rIn,-32
%
% // reverse the word order of each dword in the register (e.g. RV128IB+, there are 2 dwords)
% wswap.d rOut,rIn:
%     grevi rOut,rIn,32
%
% // swap the nibbles of each byte in the whole register (useful for converting to/from
% // bigendian hex)
% nswap.b rOut,rIn:
%     grevi rOut,rIn,4
%
% // Etc.  A total of XLEN different possible operations can be constructed.
% \end{verbatim}
%
% \subsection{Justification}
%
% Quoted from page 102 of Hacker's Delight:
%
% https://books.google.com/books?id=iBNKMspIlqEC\&lpg=PP1\&pg=RA1-SL20-PA2\#v=onepage\&q\&f=false
%
% \begin{quote}
% For k=31, this operation reverses the bits in a word. For k=24, it
% reverses the bytes in a word. For k=7, it reverses the bits in each
% byte, without changing the positions of the bytes. For k=16, it swaps
% the left and right halfwords of a word, and so on. In general, it moves
% the bit at positing m to position m XOR k. It can be implemented in
% hardware very similarly to the way a rotate shifter is usually
% implemented (five stages of MUX, with each stage controlled by a bit of
% the shift amount k)
%
% This strongly suggests that this generic instruction could allow us to
% use a great deal of other bitwise operations as pseudo-instructions.
% \end{quote}
%
% \subsection{References}
%
% Hackers Delight, Chapter 7.1, ``Generalized Bit Reversal'' in
%
% https://books.google.com/books?id=iBNKMspIlqEC\&lpg=PP1\&pg=RA1-SL20-PA2\#v=onepage\&q\&f=false
%
% http://hackersdelight.org/

%%%%%%%%%%%%%%%%%%%%%%%%%%%%%%%%%%%%%%%%%%%%%%%%%%%%%%%%%%%%%%%%%%%%%%%%%%%%%%%%%%%%%%%%%%%

\section{Shift Ones (Left/Right) (\texttt{slo,\ sloi,\ sro,\ sroi})}

These instructions are similar to shift-logical operations from the base
spec, except instead of shifting in zeros, it shifts in ones. This can
be used in mask creation or bit-field insertions, for example.

These instructions are exactly the same as the equivalent logical shift
operations, except the shift shifts in ones values.

\begin{verbatim}
uint_xlen_t slo(uint_xlen_t rs1, uint_xlen_t rs2)
{
    int12_t amount=(rs2 & (XLEN-1));
    uint_xlen_t mask=((1 << amount)-1);
    return (rs1 << amount) | mask;
}

uint_xlen_t sloi(uint_xlen_t rs1, int12_t immed)
{
    int12_t amount=(immed & (XLEN-1));
    uint_xlen_t mask=((1 << amount)-1);
    return (rs1 << amount) | mask;
}

uint_xlen_t sro(uint_xlen_t rs1, uint_xlen_t rs2)
{
    int12_t amount=(rs2 & (XLEN-1));
    uint_xlen_t mask=~((1 << (XLEN-amount))-1);
    return (rs1 >> amount) | mask;
}

uint_xlen_t sroi(uint_xlen_t rs1, int12_t immed)
{
    int12_t amount=(immed & (XLEN-1));
    uint_xlen_t mask=~((1 << (XLEN-amount))-1);
    return (rs1 >> amount) | mask;
}
\end{verbatim}

\vspace{-0.2in}
\begin{center}
\begin{tabular}{S@{}R@{}R@{}S@{}R@{}O}
\\
\instbitrange{31}{25} &
\instbitrange{24}{20} &
\instbitrange{19}{15} &
\instbitrange{14}{12} &
\instbitrange{11}{7} &
\instbitrange{6}{0} \\
\hline
\multicolumn{1}{|c|}{imm[11:5]} &
\multicolumn{1}{c|}{imm[4:0]} &
\multicolumn{1}{c|}{rs1} &
\multicolumn{1}{c|}{funct3} &
\multicolumn{1}{c|}{rd} &
\multicolumn{1}{c|}{opcode} \\
\hline
7 & 5 & 5 & 3 & 5 & 7 \\
??????? & shamt & src & SLOIW & dest & OP-IMM-32 \\
??????? & shamt & src & SROIW & dest & OP-IMM-32 \\
\end{tabular}
\end{center}

\vspace{-0.2in}
\begin{center}
\begin{tabular}{S@{}R@{}R@{}S@{}R@{}O}
\\
\instbitrange{31}{27} &
\instbitrange{26}{20} &
\instbitrange{19}{15} &
\instbitrange{14}{12} &
\instbitrange{11}{7} &
\instbitrange{6}{0} \\
\hline
\multicolumn{1}{|c|}{imm[11:5]} &
\multicolumn{1}{c|}{imm[4:0]} &
\multicolumn{1}{c|}{rs1} &
\multicolumn{1}{c|}{funct3} &
\multicolumn{1}{c|}{rd} &
\multicolumn{1}{c|}{opcode} \\
\hline
5 & 7 & 5 & 3 & 5 & 7 \\
????? & shamt & src & SLOI & dest & OP-IMM \\
????? & shamt & src & SROI & dest & OP-IMM \\
\end{tabular}
\end{center}

\vspace{-0.2in}
\begin{center}
\begin{tabular}{S@{}R@{}R@{}S@{}R@{}O}
\\
\instbitrange{31}{25} &
\instbitrange{24}{20} &
\instbitrange{19}{15} &
\instbitrange{14}{12} &
\instbitrange{11}{7} &
\instbitrange{6}{0} \\
\hline
\multicolumn{1}{|c|}{funct7} &
\multicolumn{1}{c|}{rs2} &
\multicolumn{1}{c|}{rs1} &
\multicolumn{1}{c|}{funct3} &
\multicolumn{1}{c|}{rd} &
\multicolumn{1}{c|}{opcode} \\
\hline
7 & 5 & 5 & 3 & 5 & 7 \\
??????? & src2 & src1 & SRO    & dest & OP    \\
??????? & src2 & src1 & SLO    & dest & OP    \\
??????? & src2 & src1 & SROW   & dest & OP-32 \\
??????? & src2 & src1 & SLOW   & dest & OP-32 \\
\end{tabular}
\end{center}


\texttt{s(l/r)o(i)} is encoded similarly to the logical shifts in the
base spec. However, the spec of the entire family of instructions is
changed so that the high bit of the instruction indicates the value to
be inserted during a shift. This means that a \texttt{sloi} instruction
can be encoded similarly to an \texttt{slli} instruction, but with a 1
in the highest bit of the encoded instruction. This encoding is
backwards compatible with the definition for the shifts in the base
spec, but allows for simple addition of a ones-insert.

When implementing this circuit, the only change in the ALU over a
standard logical shift is that the value shifted in is not zero, but is
a 1-bit register value that has been forwarded from the high bit of the
instruction decode. This creates the desired behavior on both logical
zero-shifts and logical ones-shifts.

% \subsection{Extensions: \texttt{s(l/r)o(i).w} (R64I+), \texttt{s(l/r)o(i).d} (R128I+)}
%
% On RV64B+, the appropriate versions of shift-ones that operate on
% sub-portions of the register are defined. On these platforms, the input
% source is set to zero for the high portions of the register, and the
% shift only occurs on the lower portions of the register. See
% \texttt{slli.w} and \texttt{slli.d} in the R64I/R128I specifications.
% See \texttt{slli.w} and \texttt{slli.d} in the R64I/R128I
% specifications.
%
% \subsection{Related Pseudo-Instructions}
%
% \begin{verbatim}
% // builds a mask in the low-order bits up to a certain point
% maski rOut,iWidth:
%     sloi rOut,r0,iWidth
%
% // builds a mask in the low-order bits up to a certain point
% mask rOut,rIn:
%     slo rOut,r0,rIn
%
% // extracts a bitfield of length width from offset and moves it down to the bottom
% bfextracti rOut,rIn,iWidth,iOffset:
%     slli rTop,rIn,XLEN-iWidth-iOffset
%     srli rOut,rTop,XLEN-iWidth
%
% // updates a bitfield of length width at offset in rCurrent with the value from rIn
% bfupdatei rOut,rCurrent,rIn,iWidth,iOffset:
%     sloi rTop,rIn,<XLEN-iWidth>
%     sroi rField,rTop,<XLEN-iWidth-iOffset>
%     and rOut,rCurrent,rField
%
% // sets a bitfield of length-width at offset to 1
% bfseti rOut,rCurrent,iWidth,iOffset:
%     sloi rMask,r0,iWidth
%     slli rMask,rMask,iOffset
%     or rOut,rCurrent,rMask
%
% // sets a bitfield of length-width at offset to 0
% bfcleari rOut,rCurrent,iWidth,iOffset:
%     sloi rMask,r0,iWidth
%     slli rMask,rMask,iOffset
%     andc rOut,rCurrent,rMask
% \end{verbatim}
%
% \subsection{Justification}
%
% This instruction can be used to create masks, which is an incredibly
% common operation for modifying the bitfield structures.
%
% \subsection{References}

%%%%%%%%%%%%%%%%%%%%%%%%%%%%%%%%%%%%%%%%%%%%%%%%%%%%%%%%%%%%%%%%%%%%%%%%%%%%%%%%%%%%%%%%%%%

\section{Rotate (Left/Right) (\texttt{rol,\ ror,\ rori})}

These instructions are similar to shift-logical operations from the base
spec, except they shift in the values from the opposite side of the
register, in order. This is also called `circular shift'.

\begin{verbatim}
uint_xlen_t rol(uint_xlen_t rs1, uint_xlen_t rs2)
{
    int12_t amount=(rs2 & (XLEN-1));
    return (rs1 << amount) | (rsl >> (XLEN-amount));
}

uint_xlen_t ror(uint_xlen_t rs1, uint_xlen_t rs2)
{
    int12_t amount=(rs2 & (XLEN-1));
    return (rs1 >> amount) | (rsl << (XLEN-amount));
}

int_xlen_t rori(uint_xlen_t rs1, int12_t immed)
{
    int12_t amount=(immed & (XLEN-1));
    return (rs1 >> amount) | (rs1 << (XLEN-amount));
}
\end{verbatim}

\vspace{-0.4in}
\begin{center}
\begin{tabular}{S@{}R@{}R@{}S@{}R@{}O}
\\
\instbitrange{31}{25} &
\instbitrange{24}{20} &
\instbitrange{19}{15} &
\instbitrange{14}{12} &
\instbitrange{11}{7} &
\instbitrange{6}{0} \\
\hline
\multicolumn{1}{|c|}{imm[11:5]} &
\multicolumn{1}{c|}{imm[4:0]} &
\multicolumn{1}{c|}{rs1} &
\multicolumn{1}{c|}{funct3} &
\multicolumn{1}{c|}{rd} &
\multicolumn{1}{c|}{opcode} \\
\hline
7 & 5 & 5 & 3 & 5 & 7 \\
11????? & shamt & src & RORIW & dest & OP-IMM-32 \\
\end{tabular}
\end{center}

\vspace{-0.4in}
\begin{center}
\begin{tabular}{S@{}R@{}R@{}S@{}R@{}O}
\\
\instbitrange{31}{27} &
\instbitrange{26}{20} &
\instbitrange{19}{15} &
\instbitrange{14}{12} &
\instbitrange{11}{7} &
\instbitrange{6}{0} \\
\hline
\multicolumn{1}{|c|}{imm[11:7]} &
\multicolumn{1}{c|}{imm[6:0]} &
\multicolumn{1}{c|}{rs1} &
\multicolumn{1}{c|}{funct3} &
\multicolumn{1}{c|}{rd} &
\multicolumn{1}{c|}{opcode} \\
\hline
5 & 7 & 5 & 3 & 5 & 7 \\
11??? & shamt & src & RORI & dest & OP-IMM \\
\end{tabular}
\end{center}

\vspace{-0.4in}
\begin{center}
\begin{tabular}{S@{}R@{}R@{}S@{}R@{}O}
\\
\instbitrange{31}{25} &
\instbitrange{24}{20} &
\instbitrange{19}{15} &
\instbitrange{14}{12} &
\instbitrange{11}{7} &
\instbitrange{6}{0} \\
\hline
\multicolumn{1}{|c|}{funct7} &
\multicolumn{1}{c|}{rs2} &
\multicolumn{1}{c|}{rs1} &
\multicolumn{1}{c|}{funct3} &
\multicolumn{1}{c|}{rd} &
\multicolumn{1}{c|}{opcode} \\
\hline
7 & 5 & 5 & 3 & 5 & 7 \\
11????? & src2 & src1 & RORW   & dest & OP-32 \\
11????? & src2 & src1 & ROLW   & dest & OP-32 \\
11????? & src2 & src1 & ROR    & dest & OP    \\
11????? & src2 & src1 & ROL    & dest & OP    \\
\end{tabular}
\end{center}


\texttt{ror(i),rol(i)} is implemented very similarly to the other shift
instructions. One possible way to encode it is to re-use the way that
bit 30 in the instruction encoding selects `arithmetic shift' when bit
31 is zero (signalling a logical-zero shift). We can re-use this so that
when bit 31 is set (signalling a logical-ones shift), if bit 31 is also
set, then we are doing a rotate. The following table summarizes the
behavior. The generalized reverse opcodes can be encoded using the
bit pattern that would otherwise encode an ``Arithmetic Left Shift''
(which is an operation that does not exist).

\begin{longtable}[c]{@{}lll@{}}
\caption{Rotate Encodings}\tabularnewline
\toprule
Bit 31 & Bit 30 & Meaning\tabularnewline
\midrule
\endfirsthead
\toprule
Bit 31 & Bit 30 & Meaning\tabularnewline
\midrule
\endhead
0 & 0 & Logical Shift-Zeros\tabularnewline
0 & 1 & Arithmetic Shift\tabularnewline
1 & 0 & Logical Shift-Ones\tabularnewline
1 & 1 & Rotate\tabularnewline
\bottomrule
\end{longtable}

% Implementation Commentary
%
% \subsection{Extensions: \texttt{ro(r/l)(i).w} (R64I+),\texttt{ro(r/l)(i).d} (R128I+)}
%
% On RV64B+, the appropriate versions of rotate that operate on
% sub-portions of the register are defined. On these platforms, the input
% source is set to zero for the high portions of the register, and the
% shift only occurs on the lower portions of the register. See
% \texttt{slli.w} and \texttt{slli.d} in the R64I/R128I specifications.
%
% \subsection{Justification}
%
% This instruction is very useful for cryptography, hashing, and other
% operations.
%
% \subsection{References}

%%%%%%%%%%%%%%%%%%%%%%%%%%%%%%%%%%%%%%%%%%%%%%%%%%%%%%%%%%%%%%%%%%%%%%%%%%%%%%%%%%%%%%%%%%%

\section{And-with-complement (\texttt{andc})}

\vspace{-0.4in}
\begin{center}
\begin{tabular}{S@{}R@{}R@{}S@{}R@{}O}
\\
\instbitrange{31}{25} &
\instbitrange{24}{20} &
\instbitrange{19}{15} &
\instbitrange{14}{12} &
\instbitrange{11}{7} &
\instbitrange{6}{0} \\
\hline
\multicolumn{1}{|c|}{funct7} &
\multicolumn{1}{c|}{rs2} &
\multicolumn{1}{c|}{rs1} &
\multicolumn{1}{c|}{funct3} &
\multicolumn{1}{c|}{rd} &
\multicolumn{1}{c|}{opcode} \\
\hline
7 & 5 & 5 & 3 & 5 & 7 \\
??????? & src2 & src1 & ANDC  & dest & OP    \\
??????? & src2 & src1 & ANDCW & dest & OP-32 \\
\end{tabular}
\end{center}


This is an R-type instruction that implements the and-with-complement
operation \texttt{x=(a\ \&\ \textasciitilde{}b)} with an assembly format
of \texttt{andc\ ro,r1,r2}.

Additionally, the four positive bitwise operations can all be
implemented with the same ALU area (e.g.~as a single microcode
instruction) using a scheme described in the implementation commentary,
and fused versions can be implemented as well.

\begin{verbatim}
uint_xlen_t andc(uint_xlen_t rs1, uint_xlen_t rs2)
{
    return rs1 & ~rs2;
}
\end{verbatim}

This instruction should be encoded similarly to the instruction in the
base spec. The exact instruction encoding is to be decided, however.

% \subsection{Related Pseudo-Instructions}
%
% \begin{verbatim}
% // this selects the corresponding bit from the two arguments A and B, based on
% // whether or not Control is high.
% // out[i] = Control[i] ? A[i] : B[i]
% mix rOut,rControl,rA,rB:
%     and rTmp,rA,rControl
%     andc rTmp2,rB,rControl
%     or rOut,rTmp,rTmp2
%
% // same as MIX but only use if the first bit of C is true (as in from a comparator)
% // C can only be zero or 1, so if you need a SGT to implement nonzero comparator
% // its 5 instructions in total.
% mux rOut,rControl,rA,rB:
%     neg rC2,rControl
%     and rTmp,rA,rC2
%     andc rTmp2,rB,rC2
%     or rOut,rTmp,rTmp2
% \end{verbatim}
%
% \subsection{Justification}
%
% http://svn.clifford.at/handicraft/2017/bitcode/
%
% \subsection{References}
%
% http://www.hackersdelight.org/basics2.pdf

%%%%%%%%%%%%%%%%%%%%%%%%%%%%%%%%%%%%%%%%%%%%%%%%%%%%%%%%%%%%%%%%%%%%%%%%%%%%%%%%%%%%%%%%%%%

\section{Bit Extract/Deposit (\texttt{bext,\ bdep})}

This is an R-type instruction that implements the generic bit extract
and bit deposit functions. Similar implementation can be referred to as
gather/scatter or pack/unpack.

\texttt{BEXT[W] rd,rs1,rs2} collects LSB justified bits to rd from
rs1 using extract mask in rs2.

\texttt{BDEP[W] rd,rs1,rs2} writes LSB justified bits from rs1 to rd using
deposit mask in rs2.

\begin{verbatim}
uint_xlen_t bext(uint_xlen_t v, uint_xlen_t mask)
{
    uint32_t c = 0, m = 1;
    while (mask) {
        uint_xlen_t b = mask & -mask;
        if (v & b)
            c |= m;
        mask -= b;
        m <<= 1;
    }
    return c;
}

uint_xlen_t bdep32(uint_xlen_t v, uint_xlen_t mask)
{
    uint_xlen_t c = 0, m = 1;
    while (mask) {
        uint_xlen_t b = mask & -mask;
        if (v & m)
            c |= b;
        mask -= b;
        m <<= 1;
    }
    return c;
}
\end{verbatim}

\vspace{-0.3in}
\begin{center}
\begin{tabular}{S@{}R@{}R@{}S@{}R@{}O}
\\
\instbitrange{31}{25} &
\instbitrange{24}{20} &
\instbitrange{19}{15} &
\instbitrange{14}{12} &
\instbitrange{11}{7} &
\instbitrange{6}{0} \\
\hline
\multicolumn{1}{|c|}{funct7} &
\multicolumn{1}{c|}{rs2} &
\multicolumn{1}{c|}{rs1} &
\multicolumn{1}{c|}{funct3} &
\multicolumn{1}{c|}{rd} &
\multicolumn{1}{c|}{opcode} \\
\hline
7 & 5 & 5 & 3 & 5 & 7 \\
??????? & src2 & src1 & BEXT   & dest & OP       \\
??????? & src2 & src1 & BDEP   & dest & OP       \\
??????? & src2 & src1 & BEXTW  & dest & OP-32    \\
??????? & src2 & src1 & BDEPW  & dest & OP-32    \\
\end{tabular}
\end{center}


This instruction should be encoded similarly to the instruction in the
base spec. The exact instruction encoding is to be decided, however.

% \subsection{Related Pseudo-Instructions}
%
% TBD
%
% \subsection{Justification}
%
% http://svn.clifford.at/handicraft/2017/permsyn/
%
% \subsection{References}
%
% http://programming.sirrida.de/bit\_perm.html\#gather\_scatter
%
% Hackers Delight, Chapter 7.1, ``Compress, Generalized Extract'' in
%
% https://books.google.com/books?id=iBNKMspIlqEC\&lpg=PP1\&pg=RA1-SL20-PA2\#v=onepage\&q\&f=false
%
% http://hackersdelight.org/
%
% https://github.com/cliffordwolf/bextdep

%%%%%%%%%%%%%%%%%%%%%%%%%%%%%%%%%%%%%%%%%%%%%%%%%%%%%%%%%%%%%%%%%%%%%%%%%%%%%%%%%%%%%%%%%%%

\section{Compressed instructions (\texttt{c.not,\ c.neg,\ c.clz,\ c.brev})}

The RISC-V ISA has no dedicated instructions for bitwise inverse (\texttt{not})
and arithmetic inverse (\texttt{neg}). Instead \texttt{not} is implemented as
\texttt{xori\ rd,\ rs,\ -1} and \texttt{neg} is implemented as
\texttt{sub\ rd,\ x0,\ rs}.

In bitmanipulation code \texttt{not} and \texttt{neg} are very common operations. But
there are no compressed encodings for those operations because there is no \texttt{c.xori}
instructions and \texttt{c.sub} can not operate on \texttt{x0}.

Many bit manipulation operations that have dedicated opecodes in other ISAs
must be constructed from smaller atoms in RISC-V XBitmanip code. But
implementations might chose to implement them in a single micro-op using
macro-op-fusion. For this it can be helpful when the fused sequences are short.
\texttt{not} and \texttt{neg} are good candidates for macro-op-fusion, so
it can be helpful to have compressed opcodes for them.

Likewise, \texttt{clz} and \texttt{brev} (an alias for \texttt{grevi\ rd,\ rs,\ -1},
i.e. bitwise reversal) are very common atoms for building common bit
manipulation operations. So it is helpful to have compressed opcodes for them
as well.

For example, the equivalent to the x86 instruction \texttt{BSR} (find index of
highest set bit) can be implemented using the following 64 bit sequence that does not
spill other registers (and thus is a prime candidate for macro-op-fusion):

\begin{verbatim}
  bsr rd, rs:
    c.clz rd, rs
    c.neg rd
    addi rd, rd, 32
\end{verbatim}

Likewise, the AMD TBM extension adds 10 x86 bit manipulation instructions. The equivalent
RISC-V code sequences can utilize \texttt{c.not} in half of the cases.

The compressed instructions \texttt{c.not,\ c.neg,\ c.clz,\ c.brev} must be supported by
all implementations that support the C extension and XBitmanip.

\begin{table}[h]
\begin{small}
\begin{center}
\begin{tabular}{p{0in}p{0.05in}p{0.05in}p{0.05in}p{0.05in}p{0.05in}p{0.05in}p{0.05in}p{0.05in}p{0.05in}p{0.05in}p{0.05in}p{0.05in}p{0.05in}p{0.05in}p{0.05in}p{0.05in}l}
& & & & & & & & & & \\
                      &
\instbit{15} &
\instbit{14} &
\instbit{13} &
\multicolumn{1}{c}{\instbit{12}} &
\instbit{11} &
\instbit{10} &
\instbit{9} &
\instbit{8} &
\instbit{7} &
\instbit{6} &
\multicolumn{1}{c}{\instbit{5}} &
\instbit{4} &
\instbit{3} &
\instbit{2} &
\instbit{1} &
\instbit{0} \\
\cline{2-17}

&
\multicolumn{3}{|c|}{011} &
\multicolumn{1}{c|}{nzimm[17]} &
\multicolumn{5}{c|}{rd$\neq$$\{0,2\}$} &
\multicolumn{5}{c|}{nzimm[16:12]} &
\multicolumn{2}{c|}{01} & C.LUI {\em \tiny (\sout{RES, nzimm=0}; HINT, rd=0)} \\
\cline{2-17}

&
\multicolumn{3}{|c|}{011} &
\multicolumn{1}{c|}{0} &
\multicolumn{2}{c|}{00} &
\multicolumn{3}{c|}{rs2$'$/rd$'$} &
\multicolumn{5}{c|}{0} &
\multicolumn{2}{c|}{01} & C.NEG \\
\cline{2-17}

&
\multicolumn{3}{|c|}{011} &
\multicolumn{1}{c|}{0} &
\multicolumn{2}{c|}{01} &
\multicolumn{3}{c|}{rs1$'$/rd$'$} &
\multicolumn{5}{c|}{0} &
\multicolumn{2}{c|}{01} & C.NOT \\
\cline{2-17}

&
\multicolumn{3}{|c|}{011} &
\multicolumn{1}{c|}{0} &
\multicolumn{2}{c|}{10} &
\multicolumn{3}{c|}{rs1$'$/rd$'$} &
\multicolumn{5}{c|}{0} &
\multicolumn{2}{c|}{01} & C.CLZ \\
\cline{2-17}

&
\multicolumn{3}{|c|}{011} &
\multicolumn{1}{c|}{0} &
\multicolumn{2}{c|}{11} &
\multicolumn{3}{c|}{rs1$'$/rd$'$} &
\multicolumn{5}{c|}{0} &
\multicolumn{2}{c|}{01} & C.PCNT \\
\cline{2-17}
  

\end{tabular}
\end{center}
\end{small}
\end{table}


This four instructions fit nicely in the reserved space in C.LUI/C.ADDI16SP.
They only occupy $0.13\%$ of the $\approx15.6$ bits wide RVC encoding space.
