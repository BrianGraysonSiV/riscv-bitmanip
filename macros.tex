\chapter{Pseudo-Ops, Macros, Algorithms}

This chapter contains a collection of psudo-ops, macros, and algorithms using
the XBitmanip extension. For the sake of simplicity we assume RV32 for most
examples in this chapter.

Most assembler routines in this chapter are written as if they were ABI functions,
i.e. arguments are passed in a0, a1, \dots\ and results are returned in a0. Registers
t0, t1, \dots\ are used for spilling.

\section{GREVI Pseudo-Ops}

Tables~\ref{pseudos-grevi32}~and~\ref{pseudos-grevi64} list assembler
pseudo-ops for commonly used {\tt grevi} commands.

\begin{table}[h]
\begin{small}
\begin{center}
\begin{tabular}{l l l}
Pseudoinstruction & Base Instruction(s) & Meaning \\ \hline

{\tt brev rd, rs} & {\tt grevi rd, rs, 31} & bitwise reverse \\
{\tt bswap rd, rs} & {\tt grevi rd, rs, 24} & reverse the byte order \\
{\tt bswap.h rd, rs} & {\tt grevi rd, rs, 8} & reverse the byte order in each half word \\
{\tt hswap rd, rs} & {\tt grevi rd, rs, 16} & reverse the order of half words \\

\hline

\end{tabular}
\end{center}
\end{small}
\caption{RV32 GREVI Pseudoinstructions.}
\label{pseudos-grevi32}
\end{table}

\begin{table}[h]
\begin{small}
\begin{center}
\begin{tabular}{l l l}
Pseudoinstruction & Base Instruction(s) & Meaning \\ \hline

{\tt brev rd, rs} & {\tt grevi rd, rs, 63} & bitwise reverse \\
{\tt bswap rd, rs} & {\tt grevi rd, rs, 56} & reverse the byte order \\
{\tt bswap.h rd, rs} & {\tt grevi rd, rs, 8} & reverse the byte order in each half word \\
{\tt bswap.w rd, rs} & {\tt grevi rd, rs, 24} & reverse the byte order in each word \\
{\tt hswap rd, rs} & {\tt grevi rd, rs, 48} & reverse the order of half words \\
{\tt hswap.w rd, rs} & {\tt grevi rd, rs, 16} & reverse the order of half words in each word \\
{\tt wswap rd, rs} & {\tt grevi rd, rs, 32} & reverse the order of words \\
\hline

\end{tabular}
\end{center}
\end{small}
\caption{RV64 GREVI Pseudoinstructions.}
\label{pseudos-grevi64}
\end{table}

\section{Bit Scanning Operations}

\subsubsection{Count Leading Zeros}

This is trivially just the \texttt{clz} instruction:

\begin{verbatim}
  clz a0, a0
\end{verbatim}

\subsubsection{Count Leading Ones}

Simply invert before executing the \texttt{clz} instruction:

\begin{verbatim}
  not a0, a0
  clz a0, a0
\end{verbatim}

\subsubsection{Find index of highest bit set (aka \texttt{ilog2})}

\begin{verbatim}
  clz a0, a0
  neg a0, a0
  addi a0, a0, 32
\end{verbatim}

\subsubsection{Find index of lowest bit set}

\begin{verbatim}
  brev a0, a0
  clz a0, a0
\end{verbatim}

\section{Butterfly Operations}

The \texttt{grev}, \texttt{bext}, and \texttt{bdep} instructions are commonly
implemented using butterfly circuits\footnote{http://programming.sirrida.de/bit\_perm.html\#butterfly}.
(Butterfly circuits are a powerful tool for implementing bit permutations.)
However, sometimes one wants to use butterfly operations directly in a program
and it is not immediately obvious how to use GREVI and BDEP to perform
butterfly operations.

One can easily implement a single butterfly stage using GREVI in just four
instructions:

\begin{verbatim}
  grevi t0, a0, N
  and t0, t0, a1
  andc t1, a0, a1
  or a0, t0, t1
\end{verbatim}

With {\tt a0} containing the data input on entry and the result on exit. {\tt N} is a
power of two indicating which butterfly stage to implement, and {\tt a1} contains
an XLEN bitmask derived from the XLEN/2 butterfly control word.

This bitmask can either be pre-computed or, if it needs to be generated
on-the-fly, it can easily be calculated with the help of BDEP. For example
for the first butterfly stage (using a0 as input and output):

\begin{verbatim}
  li t0, 0x55555555
  bdep t0, a0, t0
  slli a0, t0, 1
  or a0, a0, t0
\end{verbatim}

Likewise for the second butterfly stage:

\begin{verbatim}
  li t0, 0x33333333
  bdep t0, a0, t0
  slli a0, t0, 2
  or a0, a0, t0
\end{verbatim}

For the third stage:

\begin{verbatim}
  li t0, 0x0f0f0f0f
  bdep t0, a0, t0
  slli a0, t0, 4
  or a0, a0, t0
\end{verbatim}

For the fourth stage:

\begin{verbatim}
  li t0, 0x00ff00ff
  bdep t0, a0, t0
  slli a0, t0, 8
  or a0, a0, t0
\end{verbatim}

And for the fifth stage:

\begin{verbatim}
  li t0, 0x0000ffff
  and t0, a0, t0
  slli a0, t0, 16
  or a0, a0, t0
\end{verbatim}
