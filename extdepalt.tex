\chapter{Alternatives to \texttt{bext} and \texttt{bdep}}

There are issues with \texttt{bext} and \texttt{bdep}. One is the high hardware cost
of the unit performing the operations. Another is the open question of patents potentially
preventing implementations to use the best-known single-cycle implementations.

Various multi-cycle implementations have been proposed, but preliminary evaluations of
real-world bit manipulation tasks show that multi-cycle \texttt{bext} and \texttt{bdep}
often are not fast enough to outperform algortihms that use sequences of shifts and bit
masks.

However, some operations that would benefit from \texttt{bext} and \texttt{bdep} would
be very hard to implement without them. Some examples are Bit scatter/gather
(compress/uncompress) operations, goats-and-sheeps, and calulating the index of the Nth
set bit in a word. For example, the following code efficiently calculates the
index of the tenth set bit in {\tt a0} using \texttt{bdep}:

\begin{verbatim}
  li a1, 0x00000200
  bdep a0, a1, a0
  brev a0, a0
  clz a0, a0
\end{verbatim}

The instruction encoding space used by \texttt{bext} and \texttt{bdep} is
minimal, i.e. not much is gained by preserving the opcode space occupied
by the two instructions. So one possible solution would be to make \texttt{bext}
and \texttt{bdep} optional features that may be emulated by software, and add
fast implementations of the much simpler {\tt shuffle} and {\tt unshuffle} instructions
described below. Those instructions simplify some of the most common cases that
would otherwise benefit from a fast \texttt{bext} and \texttt{bdep} unit.

For small cores this would be an immense simplification because they would not
need to implement \texttt{bext} and \texttt{bdep} in hardware. For large cores
that provide \texttt{bext} and \texttt{bdep} hardware, adding the instructions
below is only a minor additional effort. But this additional effort would not
be in vain because the instructions below are somewhat orthoganal to
\texttt{bext} and \texttt{bdep}, so that overall the combined feature set of
\texttt{bext} and \texttt{bdep} and the instructions below would be even more
powerful than \texttt{bext} and \texttt{bdep} alone.

A compiler would decide to use \texttt{bext} or \texttt{bdep} based on the optimization
profile for the concrete processor it is optimizing for, similar to the decision whether
to use MUL or DIV with a constant, or to perform the same operation using a
longer sequence of much simpler operations.

%%%%%%%%%%%%%%%%%%%%%%%%%%%%%%%%%%%%%%%%%%%%%%%%%%%%%%%%%%%%%%%%%%%%%%%%%%%%%%%%%%%%%%%%%%%

\section{Generalized Bit Permutations (\texttt{shuffle,\ unshuffle})}

This instructions perform a bit permutation on the value in rs1. Which bit permutation
is performed is defined by the control word in rs2 (Table~\ref{shuffle-unshuffle-ctrl}).

\begin{table}[h]
\begin{small}
\begin{center}
\begin{tabular}{p{0.5in}p{0.5in}p{0.5in}p{0.5in}p{0.5in}p{0.5in}p{0.2in}p{0.2in}p{0.05in}p{0.05in}l}
& & & & & & & & & & \\
\multicolumn{1}{l}{\instbit{63}} &
\multicolumn{1}{r}{\instbit{48}} &
\multicolumn{1}{l}{\instbit{47}} &
\multicolumn{1}{r}{\instbit{32}} &
\multicolumn{1}{l}{\instbit{31}} &
\multicolumn{1}{r}{\instbit{16}} &
\multicolumn{1}{l}{\instbit{15}} &
\multicolumn{1}{r}{\instbit{12}} &
\multicolumn{1}{l}{\instbit{11}} &
\multicolumn{1}{r}{\instbit{0}} & \\
\cline{1-10}

\multicolumn{2}{|c|}{unused} &
\multicolumn{4}{c|}{mask} &
\multicolumn{2}{c|}{mode} &
\multicolumn{2}{c|}{command} & RV64 \\
\cline{1-10}

& & & &
\multicolumn{2}{|c|}{mask} &
\multicolumn{2}{c|}{mode} &
\multicolumn{2}{c|}{command} & RV32 \\
\cline{5-10}

\end{tabular}
\end{center}
\end{small}
\caption{\texttt{shuffle,\,unshuffle} control word}
\label{shuffle-unshuffle-ctrl}
\end{table}

This spec only defines command $=$ 0. Non-zero command values are reserved for
future use.  An implementation that does not support a given command must
return 0 in rd. Support for command 0 is mandatory. Command values 48-127 are
reserved for non-standard extensions (NSE).

Command 0 implements functions that are required for computing zip and unzip
operations, butterfly stages or entire butterfly networks, omega stages or networks, flip
stages or network, and similar operations. Tables~\ref{shuffle-modes} and~\ref{unshuffle-modes}
list the operations performed by command 0. Note that this functions can all use the
existing butterfly network that implements the GREV instructions.

\begin{table}[h]
\begin{small}
\begin{center}
\begin{tabular}{l l}
Mode & Description \\ \hline

0000 & zip + butterfly(mask, 0) \\
0001 & zip + butterfly(mask, 1) \\
0010 & zip + butterfly(mask, 2) \\
0011 & zip + butterfly(mask, 3) \\
0100 & zip + butterfly(mask, 4) \\
0101 & zip + butterfly(mask, 5) (RV64/128; NSE) \\
0110 & zip + butterfly(mask, 6) (RV128; NSE) \\
0111 & reserved for future standard extensions \\

\hline

1000 & butterfly(mask, 0) \\
1001 & butterfly(mask, 1) \\
1010 & butterfly(mask, 2) \\
1011 & butterfly(mask, 3) \\
1100 & butterfly(mask, 4) \\
1101 & butterfly(mask, 5) (RV64/128; NSE) \\
1110 & butterfly(mask, 6) (RV128; NSE) \\
1111 & reserved for future standard extensions \\

\hline

\end{tabular}
\end{center}
\end{small}
\caption{Modes for shuffle command 0}
\label{shuffle-modes}
\end{table}

\begin{table}[h]
\begin{small}
\begin{center}
\begin{tabular}{l l}
Mode & Description \\ \hline

0000 & butterfly(mask, 0) + unzip \\
0001 & butterfly(mask, 1) + unzip \\
0010 & butterfly(mask, 2) + unzip \\
0011 & butterfly(mask, 3) + unzip \\
0100 & butterfly(mask, 4) + unzip \\
0101 & butterfly(mask, 5) + unzip (RV64/128; NSE) \\
0110 & butterfly(mask, 6) + unzip (RV128; NSE) \\
0111 & reserved for future standard extensions \\

\hline

1000 & reserved for future standard extensions \\
1001 & reserved for future standard extensions \\
1010 & reserved for future standard extensions \\
1011 & reserved for future standard extensions \\
1100 & reserved for future standard extensions \\
1101 & reserved for future standard extensions \\
1110 & reserved for future standard extensions \\
1111 & reserved for future standard extensions \\

\hline

\end{tabular}
\end{center}
\end{small}
\caption{Modes for unshuffle command 0}
\label{unshuffle-modes}
\end{table}

Modes that are reserved for future standard or non-standard extensions must return
0 in rd on implementations that do not support those future extensions.

\texttt{shuffle} with rs2$=$0 (x0) implements just a zip operation (butterfly is
disabled because mask$=$0), and \texttt{unshuffle} with rs2$=$0 implements just
an unzip operation.

The \texttt{zip} (aka ``shuffle'') operation interleaves the bits of the lower and upper
half of its argument. The \texttt{unzip} (aka ``unshuffle'') instruction performs the inverse.

In other words, \texttt{zip} performs a rotate left shift on the bit indices,
and \texttt{unzip} performs a rotate right shift on the bit indices.
Performing \texttt{zip} $log_2(\textrm{XLEN})$ times is the identity. Performing it
$log_2(\textrm{XLEN})-1$ times is equivalent to one execution of \texttt{unzip}.

\begin{verbatim}
uint_xlen_t zip(uint_xlen_t rs1)
{
    uint_xlen_t x = 0;
    for (int i = 0; i < XLEN/2; i++) {
        x |= ((rs1 >> i) & 1) << (2*i);
        x |= ((rs1 >> (i+XLEN/2)) & 1) << (2*i+1);
    }
    return x;
}

uint_xlen_t unzip(uint_xlen_t rs1)
{
    uint_xlen_t x = 0;
    for (int i = 0; i < XLEN/2; i++) {
        x |= ((rs1 >> (2*i)) & 1) << i;
        x |= ((rs1 >> (2*i+1)) & 1) << (i+XLEN/2);
    }
    return x;
}
\end{verbatim}

The butterfly operation performs a single butterfly stage N (i.e. the {\tt grev} operation
with argument $2^N$), which performs $\textrm{XLEN}/2$ pairwise bit swaps. But unlike
{\tt grev} the individual bit swaps are conditional and the $\textrm{XLEN}/2$
bits in mask determine which bit swaps are taken.

\begin{verbatim}
uint_xlen_t swapbits(uint_xlen_t x, int p, int q)
{
    assert(p < q);
    x = x ^ ((x & (1 << p)) << (q-p));
    x = x ^ ((x & (1 << q)) >> (q-p));
    x = x ^ ((x & (1 << p)) << (q-p));
    return x;
}

uint_xlen_t butterfly(uint_xlen_t x, uint_xlen_t mask, int N)
{
    int a = 1 << N, b = 2*a;
    for (int i = 0; i < XLEN/2; i++) {
        int p = b*(i/a) + i%a, q = p + a;
        if ((mask >> i) & 1)
            x = swapbits(x, p, q);
    }
    return x;
}
\end{verbatim}

Putting it all together:

\begin{verbatim}
uint_xlen_t shuffle(uint_xlen_t x, uint_xlen_t ctrl)
{
    uint_xlen_t mask = ctrl >> 16;
    bool dozip = !((ctrl >> 15) & 1);
    int stage = (ctrl >> 12) & 7;
    int cmd = ctrl & 0xfff;

    if (cmd != 0 || stage >= LOG2_XLEN)
        return 0;

    x = dozip ? zip(x) : x;
    x = butterfly(x, mask, stage);
    return x;
}

uint_xlen_t unshuffle(uint_xlen_t x, uint_xlen_t ctrl)
{
    uint_xlen_t mask = ctrl >> 16;
    bool dounzip = !((ctrl >> 15) & 1);
    int stage = (ctrl >> 12) & 7;
    int cmd = ctrl & 0xfff;

    if (cmd != 0 || stage >= LOG2_XLEN || !dounzip)
        return 0;

    x = butterfly(x, mask, stage);
    x = unzip(x);
    return x;
}
\end{verbatim}

On RV32, a control word for command 0 can be loaded using a single \texttt{lui} instruction.
At most $2\cdot{}log_2(\textrm{XLEN})-1$ shuffle or unshuffle operations are required to
perform an arbitrary bit permutation. Most bit permutations that arise from real-world
applications can be implemented in shorter sequences.

Commands in the range 1-2047 with the upper bits (mask/mode) set to zero can be
loaded with a single \texttt{li} instruction. Note that there is no
requirement for future non-zero commands to perform bit-permutations or even
reversible operations. But if they do, and if they are not their own inverse,
then {\tt shuffle} and {\tt unshuffle} on the same control word should
implement inverse operatations. (This is the case for command $0$. Note that a
butterfly operation without zip or unzip is its own inverse.)

\vspace{-0.4in}
\begin{center}
\begin{tabular}{S@{}R@{}R@{}S@{}R@{}O}
\\
\instbitrange{31}{25} &
\instbitrange{24}{20} &
\instbitrange{19}{15} &
\instbitrange{14}{12} &
\instbitrange{11}{7} &
\instbitrange{6}{0} \\
\hline
\multicolumn{1}{|c|}{funct7} &
\multicolumn{1}{c|}{rs2} &
\multicolumn{1}{c|}{rs1} &
\multicolumn{1}{c|}{funct3} &
\multicolumn{1}{c|}{rd} &
\multicolumn{1}{c|}{opcode} \\
\hline
7 & 5 & 5 & 3 & 5 & 7 \\
??????? & src2 & src1 & SHUFFLE     & dest & OP       \\
??????? & src2 & src1 & SHUFFLEW    & dest & OP-32    \\
\end{tabular}
\end{center}


This instruction should be encoded similarly to the R-type instructions in the
base spec. The exact instruction encoding is to be decided, however.

\subsubsection{Pseudo instruction {\tt bfly}}

{\tt shuffle} with mode[3]$=$1 (and mode[2:0] $<$ $log_2(\textrm{XLEN})$) performs
a butterfly operation. The assembler should provide a {\tt bfly} pseudo-instruction
for rd $\neq$ rs and constant {\tt mask} and {\tt N} (if {\tt N} is omitted then
{\tt N}$=$0 is assumed):

\begin{verbatim}
  bfly rd, rs, mask[, N]     ->     lui rd, ((mask << 4) | 8 | N)
                                    shuffle rd, rs, rd
\end{verbatim}

(On RV64, longer sequences are required instead of {\tt lui} to load the full
32 bit mask into {\tt rd}.)

For example, an arbitrary RV32 bit permutation using a complete butterfly network:

\begin{verbatim}
  bfly a1, a0, <maskA>, 4
  bfly a2, a1, <maskB>, 3
  bfly a0, a2, <maskC>, 2
  bfly a1, a0, <maskD>, 1
  bfly a2, a1, <maskE>, 0
  bfly a0, a2, <maskF>, 1
  bfly a1, a0, <maskG>, 2
  bfly a2, a1, <maskH>, 3
  bfly a0, a2, <maskI>, 4
\end{verbatim}

Permutations arising from real-world applications can often be implemented using
shorter sequences.

\subsubsection{Pseudo instructions {\tt omega} and {\tt flip}}

A zip operation followed by a butterfly(0) is commonly known as an {\it omega stage}.

A butterfly(0) operation followed by unzip (equivalent to an unzip operation followed
by butterfly($\log_2(\textrm{XLEN})-1$)) is commonly
known as a {\it flip stage}. The assember should provide appropiate
pseudo-instructions for rd $\neq$ rs and constant {\tt mask} and {\tt N} (if {\tt N}
is omitted then {\tt N}$=$0 is assumed):

\begin{verbatim}
  omega rd, rs, mask[, N]     ->     lui rd, ((mask << 4) | N)
                                     shuffle rd, rs, rd

  flip  rd, rs, mask[, N]     ->     lui rd, ((mask << 4) | N)
                                     unshuffle rd, rs, rd
\end{verbatim}

For example, an arbitrary RV32 bit permutation using a complete omega-flip network:

\begin{verbatim}
  omega a1, a0, <maskA>
  omega a0, a1, <maskB>
  omega a1, a0, <maskC>
  omega a0, a1, <maskD>
  omega a1, a0, <maskE>
  flip  a0, a1, <maskF>
  flip  a1, a0, <maskG>
  flip  a0, a1, <maskH>
  flip  a1, a0, <maskI>
  flip  a0, a1, <maskJ>
\end{verbatim}

(Either {\tt <maskE>} or {\tt <maskF>} can be zero, effectively turning either
the last {\tt omega} into a {\tt zip} or the first {\tt flip} into an {\tt unzip}.)

As for butterfly networks, permutations arising from real-world applications
can often be implemented using a much shorter sequence. Especially if {\tt bfly},
{\tt omega} and {\tt flip} can be mixed arbitrarily.

\subsubsection{Pseudo instructions {\tt zip} and {\tt unzip}}

With a zero value as control word, {\tt shuffle} and {\tt unshuffle} perform
simple zip or unzip operations. The assembler should provide pseudo-instructions
for those cases (Table~\ref{pseudos-zip-unzip}).

\begin{table}[h]
\begin{small}
\begin{center}
\begin{tabular}{l l l}
Pseudoinstruction & Base Instruction(s) \\ \hline
{\tt zip rd, rs} & {\tt shuffle rd, rs, zero} \\
{\tt unzip rd, rs} & {\tt unshuffle rd, rs, zero} \\
\hline

\end{tabular}
\end{center}
\end{small}
\caption{Pseudo instructions {\tt zip} and {\tt unzip}}
\label{pseudos-zip-unzip}
\end{table}

The \texttt{zip} instruction with the upper half of its input cleared performs
the commonly needed ``fan-out'' operation. (Equivalent to {\tt bdep} with a
0x55555555 mask.) The \texttt{zip} instruction applied twice fans out the bits
in the lower quarter of the input word by a spacing of 4 bits.

For example, the following code calculates the bitwise prefix sum of the bits
in the lower byte of a 32 bit word on RV32:

\begin{verbatim}
  andi a0, a0, 0xff
  zip a0, a0
  zip a0, a0
  slli a1, a0, 4
  add a0, a1
  slli a1, a0, 8
  add a0, a1
  slli a1, a0, 16
  add a0, a1
\end{verbatim}

The final prefix sum is stored in the 8 nibbles of the {\tt a0} output word.

\subsubsection{Implementation notes}

Even though this instruction looks expensive, it is actually quite simple to
implement.  The butterfly operation can just reuse the butterfly circuit that
is already present to support the {\tt grev} instruction, and zip and unzip are
very cheap to implement (just one additional word-wide mux each). The ``return
0'' part for nonzero commands and reserved modes is also very cheap.

Since future non-zero commands are optional they do not add to the complexity
requirements of the command. A simple implementation can simply ignore them
and return 0 in rd.
