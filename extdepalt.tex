\chapter{Alternatives to \texttt{bext} and \texttt{bdep}}

There are issues with \texttt{bext} and \texttt{bdep}. One is the high hardware cost
of the unit performing the operations. Another is the open question of patents potentially
preventing implementations to use the best-known single-cycle implementations.

Various multi-cycle implementations have been proposed, but preliminary evaluations of
real-world bit manipulation tasks show that multi-cycle \texttt{bext} and \texttt{bdep}
often are not fast enough to outperform algortihms that use sequences of shifts and bit
masks.

However, some operations that would benefit from \texttt{bext} and \texttt{bdep} would
be very hard to implement without them. Some examples are Bit scatter/gather
(compress/uncompress) operations, goats-and-sheeps, and calulating the index of the Nth
set bit in a word. For example, the following code efficiently calculates the
index of the twelfth set bit in {\tt a0} using \texttt{bdep}:

\begin{verbatim}
  li a1, 0x00000800
  bdep a0, a0, a1
  brev a0, a0
  clz a0, a0
\end{verbatim}

The instruction encoding space used by \texttt{bext} and \texttt{bdep} is
minimal, i.e. not much is gained by preserving the opcode space occupied
by the two instructions. So one possible solution would be to make \texttt{bext}
and \texttt{bdep} optional features that may be emulated by software, and add
fast implementations of the much simpler instructions below that simplify some
of the most common cases that would otherwise benefit from a fast \texttt{bext}
and \texttt{bdep} unit.

For small cores this would be an immense simplification because they would not
need to implement \texttt{bext} and \texttt{bdep} in hardware. For large cores
that provide \texttt{bext} and \texttt{bdep} hardware, adding the instructions
below is only a minor additional effort. But this additional effort would not
be in vain because the instructions below are somewhat orthoganal to
\texttt{bext} and \texttt{bdep}, so that overall the combined feature set of
\texttt{bext} and \texttt{bdep} and the instructions below would be even more
powerful than \texttt{bext} and \texttt{bdep} alone.

A compiler would decide to use \texttt{bext} or \texttt{bdep} based on the optimization
profile for the concrete processor it is optimizing for, similar to the decision whether
to use MUL or DIV with a constant, or to perform the same operation using a
longer sequence of much simpler operations.

%%%%%%%%%%%%%%%%%%%%%%%%%%%%%%%%%%%%%%%%%%%%%%%%%%%%%%%%%%%%%%%%%%%%%%%%%%%%%%%%%%%%%%%%%%%

\section{Generalized Reverse Masked (\texttt{grevm0, grevm1, ...})}

The \texttt{grevm}$N$ instructions performs the \texttt{grev} operation
with argument $2^N$, followed by a mask uperation using the lower
$\textrm{XLEN}/2$ bits of the second argument, one mask bit for each pair of bits
that may be swapped by the \texttt{grev} operation. If the mask bit
is set then the swap is taken, otherwise the bit values stay in their
position.

\begin{verbatim}
uint_xlen_t swapbits(uint_xlen_t x, int p, int q)
{
    assert(p < q);
    x = x ^ ((x & (1 << p)) << (q-p));
    x = x ^ ((x & (1 << q)) >> (q-p));
    x = x ^ ((x & (1 << p)) << (q-p));
    return x;
}

uint_xlen_t grevmN(uint_xlen_t rs1, uint_xlen_t rs2, int N)
{
    int a = 1 << N, b = 2*a;
    uint_xlen_t x = rs1;
    for (int i = 0; i < XLEN/2; i++) {
        int p = b*(i/a) + i%a, q = p + a;
        if (rs2 & (1 << i))
            x = swapbits(x, p, q);
    }
    return x;
}
\end{verbatim}

The assembler should implement those instructions using a generic \texttt{grevm}
mnemonic with a fourth constant parameter for $N$, and also provide
\texttt{grevm}$N$ pseudo-ops.

With the help of \texttt{grevm}$N$ an arbitrary bit permutation can be computed
using $4(log_2\textrm{XLEN})-2$ instructions and $(log_2\textrm{XLEN})-0.5$ data words using a
{\it Bene\v{s} network}. For example on RV32:

\begin{verbatim}
  lhu a2, 0(a1)
  grevm a0, a0, a2, 4

  lhu a2, 2(a1)
  grevm a0, a0, a2, 3

  lhu a2, 4(a1)
  grevm a0, a0, a2, 2

  lhu a2, 6(a1)
  grevm a0, a0, a2, 1

  lhu a2, 8(a1)
  grevm a0, a0, a2, 0

  lhu a2, 10(a1)
  grevm a0, a0, a2, 1

  lhu a2, 12(a1)
  grevm a0, a0, a2, 2

  lhu a2, 14(a1)
  grevm a0, a0, a2, 3

  lhu a2, 16(a1)
  grevm a0, a0, a2, 4
\end{verbatim}

Similarly, a butterfly network or inverse butterfly network can be implemented
in $2(log_2\textrm{XLEN})$ instructions and $(log_2\textrm{XLEN})/2$ data words.

%%%%%%%%%%%%%%%%%%%%%%%%%%%%%%%%%%%%%%%%%%%%%%%%%%%%%%%%%%%%%%%%%%%%%%%%%%%%%%%%%%%%%%%%%%%

\section{Bit Shuffle (\texttt{zip,\ unzip})}

The \texttt{zip} instruction interleaves the bits of the lower and upper
half of its argument. The \texttt{unzip} instruction performs the inverse.

In other words, \texttt{zip} performs a rotate left shift on the bit indices,
and \texttt{unzip} performs a rotate right shift on the bit indices.

Performing \texttt{zip} $log_2\textrm{XLEN}$ times is the identity. Performing it
$(log_2\textrm{XLEN})-1$ times is equivalent to one execution of \texttt{unzip}.

\begin{verbatim}
uint_xlen_t zip(uint_xlen_t rs1)
{
    uint_xlen_t x = 0;
    for (int i = 0; i < XLEN/2; i++) {
        x |= ((rs1 >> i) & 1) << (2*i);
        x |= ((rs1 >> (i+XLEN/2)) & 1) << (2*i+1);
    }
    return x;
}

uint_xlen_t unzip(uint_xlen_t rs1)
{
    uint_xlen_t x = 0;
    for (int i = 0; i < XLEN/2; i++) {
        x |= ((rs1 >> (2*i)) & 1) << i;
        x |= ((rs1 >> (2*i+1)) & 1) << (i+XLEN/2);
    }
    return x;
}
\end{verbatim}

The \texttt{zip} instruction followed by a \texttt{grevm0} instruction
implements a single stage of an {\it omega network}.

The \texttt{unzip} instruction followed by a \texttt{grevm4} instruction (or
\texttt{grevm5} on RV64) implements a single stage of a {\it flip network}.

The \texttt{zip} instruction with the upper half of its input cleared performs
the commonly needed ``fan-out'' operation. (Equivalent to {\tt bdep} with a
0x55555555 mask.) The \texttt{zip} instruction applied twice fans out the bits
in the lower quarter of the input word by a spacing of 4 bits.

For example, calculating the bitwise prefix sum of the bits in the lower byte
of a 32 bit word on RV32:

\begin{verbatim}
  andi a0, a0, 0xff
  zip a0, a0
  zip a0, a0
  slli a1, a0, 4
  add a0, a1
  slli a1, a0, 8
  add a0, a1
  slli a1, a0, 16
  add a0, a1
\end{verbatim}

The final prefix sum is stored in the 8 nibbles of the {\tt a0} output word.
